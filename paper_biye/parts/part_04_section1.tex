
% =====================================================================
%  第1章  前言
% =====================================================================
\section{前言}
\label{sec:intro}

\subsection{登革热全球与中国流行概况}

登革热(Dengue Fever)是由登革病毒(Dengue Virus, DENV)引起的一种急性蚊媒传染病,属于黄病毒科黄病毒属。登革病毒包括四个血清型(DENV-1至DENV-4),彼此之间仅提供短暂的交叉保护免疫,因此个体可能在一生中多次感染不同血清型。该疾病主要经埃及伊蚊(\textit{Aedes aegypti})和白纹伊蚊(\textit{Aedes albopictus})叮咬传播,临床表现从无症状感染到轻型登革热(发热、头痛、肌肉关节痛、皮疹),乃至重型登革热(登革出血热和登革休克综合征),后者病死率可达5--20\%。世界卫生组织(WHO)将登革热列为全球十大公共卫生威胁之一\citep{who2024}。

从全球疾病负担角度看,登革热的流行规模令人瞩目。Bhatt等\citep{bhatt2013}基于全球大规模数据分析估计,全球每年约有3.9亿例登革病毒感染事件,其中约9600万例出现临床症状,导致约2.1万例死亡。值得注意的是,由于亚临床感染比例高(约75\%),实际感染规模远超官方报告数据。Messina等\citep{messina2019}通过耦合气候模型和人口数据的全球风险制图研究预测,到2050年全球将有约60亿人口面临登革热感染风险,到2080年这一数字将进一步增至63亿,较当前增加约22亿。这一趋势主要由全球变暖导致的蚊虫适宜栖息地扩张和城市化进程加速共同驱动。

中国是亚太地区登革热的重要流行区之一。与东南亚热带国家不同,中国大陆的登革热以输入性病例为主要传播起点,经本地蚊媒扩增后形成暴发。广东省因其热带--亚热带季风气候特征、高度城镇化的人口结构(常住人口超过1.26亿)以及白纹伊蚊在城市环境中的广泛孳生,长期居于全国登革热报告病例数首位\citep{lai2015}。Lai等的系统分析显示,1990--2014年间广东省报告病例数占全国总量的比例常年超过60\%,部分年份高达90\%以上。

Yue等\citep{yue2021}对2005--2019年中国大陆登革热时空格局的综合分析进一步揭示了以下规律:(1)空间聚集性——病例高度集中于珠三角地区,其中广州市和佛山市是报告最为密集的两个热点城市;(2)时间波动性——年际间发病率变异极大,暴发年与非暴发年之间可相差百倍以上;(3)季节性——全国病例的85\%以上发生在8--11月,峰值出现在9--10月。Cheng等\citep{cheng2016}对2014年广州暴发的深入分析发现,该年暴发的气候驱动因素包括夏季持续高温和适量降水创造的理想蚊媒繁殖条件,以及暑期旅行带来的大量东南亚输入性病例。2014年,广东省暴发了有记录以来最大规模的登革热疫情,全省报告超过45,000例,仅广州市即报告超过37,000例,较往年均值增长了约20倍,引起了国内外学术界和公共卫生部门的高度关注。

从全球视野来看,登革热的地理扩张速度正在加快。Kraemer等\citep{kraemer2019}的生态位模型预测显示,过去50年间登革热的全球发病率增长了30倍,目前已在100多个国家和地区呈地方性流行或周期性暴发态势。Brady等\citep{brady2013}利用全球多源数据集估算了伊蚊分布的空间概率图,发现约有39亿人(占全球人口的52\%)生活在登革热传播风险区域内。Lowe等\citep{lowe2021}在一项综合评估中强调,气候变化、城市化和国际旅行的协同作用正在加速登革热的地理扩张。对于中国南方地区而言,全球变暖可能导致白纹伊蚊的适宜栖息地向北扩展,使长江流域及以北地区也面临本地传播风险\citep{dennington2025}。这些趋势对传统的登革热监测、预警和防控体系提出了前所未有的挑战,迫切需要发展基于气候数据的定量预测工具。

\subsection{气候因素与蚊媒传播}

气候因素是调控蚊媒生物学特性和登革病毒传播效率的核心驱动力。从生物学机制上看,温度、降水和湿度通过多条途径影响蚊虫的发育速率、存活率、叮咬频率以及病毒在蚊体内的外潜伏期(Extrinsic Incubation Period, EIP),最终决定了登革热的基本再生数$R_0$的时空变异。理解这些气候因素的定量效应,是构建登革热传播效率公式的科学基础。

\textbf{温度效应。}温度是影响蚊媒传播效率最为重要的气候变量,其作用贯穿蚊虫生活史的各个阶段。从幼虫期看,温度调控着卵的孵化时间、幼虫的发育速率和化蛹率;从成蚊期看,温度影响叮咬频率(Biting Rate)、存活率(Survival Probability)和飞行活性;从病毒学角度看,温度决定了登革病毒在蚊体内完成复制所需的外潜伏期(EIP),EIP越短,感染性蚊虫越早具备传播能力。Mordecai等\citep{mordecai2019}通过汇集全球多项实验和流行病学研究的元分析,系统量化了登革热传播效率与温度的非线性关系:传播适宜温度范围为17.8--34.6$^{\circ}$C,最优传播温度约为29$^{\circ}$C。在此温度下,蚊虫叮咬率、存活率和病毒复制速率达到最优平衡,传播效率最大化。

Col\'{o}n-Gonz\'{a}lez等\citep{colon2018}基于全球多国登革热发病数据和气象观测数据,利用统计模型估算了温度效应的全球平均弹性系数:温度每升高1$^{\circ}$C,登革热发病率约增加12--17\%。然而,这一效应在不同温度基线下呈现非线性特征——在低温区(20$^{\circ}$C以下)增温的边际效应最大,而在高温区(30$^{\circ}$C以上)增温反而抑制传播。Shapiro等\citep{shapiro2017}和Lambrechts等\citep{lambrechts2011}的实验室控制实验进一步揭示了温度对白纹伊蚊EIP的U型调控关系:在20--25$^{\circ}$C范围内,EIP随温度升高从约14天显著缩短至约7天,这意味着病毒在蚊体内达到感染性所需时间缩短了一半;当温度超过30$^{\circ}$C时,EIP继续缩短但蚊虫死亡率开始上升;当超过35$^{\circ}$C后,蚊虫死亡率急剧升高,使得大多数感染蚊虫在病毒完成复制之前即已死亡,传播链断裂。Kamiya等\citep{kamiya2020}利用全球多源数据集拟合了温度依赖的传播效率综合曲线,发现其形态近似于不对称钟形(右侧下降更陡),这与蚊虫在高温下的热致死效应一致。Huber等\citep{huber2018}进一步研究了温度日变异幅度对传播效率的影响,发现较大的日温差会降低平均传播效率。

\textbf{降水效应。}降水通过两条相反的生态途径影响蚊媒传播\citep{nosrat2021}。第一条途径为"促进路径":适量降水为伊蚊提供了产卵和幼虫发育所需的静水环境。伊蚊属蚊虫的孳生地偏好小型人工容器(如废旧轮胎、花盆托盘、排水沟、废弃瓶罐等),降水可直接补充这些微小容器中的积水量,为卵的孵化和幼虫的发育创造条件\citep{roiz2015}。第二条途径为"抑制路径":暴雨和持续强降水会冲刷已有的幼虫栖息地,将卵和幼虫从容器中冲走,导致蚊虫种群骤降。此外,暴雨还可能破坏成蚊的活动空间,降低其叮咬频率。

Cheng等\citep{chengq2023}针对广州市2005--2018年的研究发现,累积降水量与布雷图指数呈正相关($r=0.48$),但当周降水量超过200\,mm后蚊媒密度反而下降,呈现出先升后降的"倒U型"关系。这种非线性饱和效应在流行病学建模中常以指数饱和函数"$1-\exp(-kR)$"加以参数化\citep{polrob2025},其中$k$为饱和常数,$R$为降水量。Zhou等\citep{zhou2025}的全球元分析进一步量化了降水的阈值效应:月降水量在100--200\,mm范围内登革热传播风险最高,超过300\,mm后风险显著下降。Li等\citep{lic2023}对亚太地区多国数据的元分析得出了类似的结论,并指出降水效应的滞后期通常为4--8周,反映了降水-蚊虫繁殖-病毒传播的生态级联过程。

\textbf{湿度效应。}相对湿度(Relative Humidity, RH)主要通过调控成蚊存活率影响登革热传播效率。高湿度环境可减缓蚊虫体表水分蒸发,延长其存活时间,从而增加每只感染蚊虫在其生命期内叮咬宿主的次数。Wu等\citep{wu2018}基于广东省2005--2015年逐月数据的非线性暴露-反应分析发现,当月均相对湿度低于76\%时,登革热发病率显著下降;而RH超过80\%后,其正向边际效应逐渐减弱。高湿度环境不仅有利于成蚊存活,还可延长其感染期,使单只蚊虫的传播能力增强\citep{dacosta2025}。Liyanage等\citep{liyanage2016}在斯里兰卡的研究中发现,湿度对蚊虫飞行活性和吸血行为也有显著的正向调节作用——RH每增加10个百分点,蚊虫叮咬率约增加8\%。DeSouza等\citep{desouza2024}利用多种机器学习方法对全球登革热预测因子进行了重要性排序,发现在包含温度、降水、湿度、风速等多个气象变量的预测模型中,相对湿度的变量重要性仅次于温度,位列第二。

\textbf{多因素交互。}实际的登革热传播过程中,温度、降水和湿度的影响并非独立叠加,而是存在复杂的协同或拮抗效应。Leung等\citep{leung2023}基于全球83个国家的面板数据分析发现,高温叠加高降水量时登革热暴发风险显著高于两个因素单独作用时预期值的简单相加,表现出超加性(Supra-additive)交互效应。其机制解释为:高温加速蚊虫发育和病毒复制,而同期充足降水提供了丰富的孳生水体,两者形成正反馈循环。Liu等\citep{liuk2020}利用分布滞后非线性模型(DLNM)对广东省21个地级市的面板数据分析,揭示了温度与湿度的交互作用在登革热传播中的显著贡献——高温高湿组合的风险比值是低温低湿的4.2倍。Cheng等\citep{chengj2021}在越南河内的研究中也发现了类似的温度-湿度协同效应。这些发现一致表明,仅考虑单一气候变量的线性效应将严重低估登革热传播风险,亟需发展能够捕捉多因素非线性交互效应的建模方法。

\subsection{模型研究现状}

\subsubsection{统计模型}

统计模型是登革热预测领域最早采用且至今应用最广的方法类别。其核心思想是利用历史数据拟合气候因素与发病率之间的统计关联,而不显式假设传播的生物学机制。广义加性模型(Generalized Additive Model, GAM)和分布滞后非线性模型(Distributed Lag Non-linear Model, DLNM)是目前最具代表性的两类统计方法。

Luo等\citep{luo2025}利用GAM框架系统分析了广州市温度、降水与登革热发病率的滞后效应结构,发现温度效应在滞后2--4周时最为显著,降水效应在滞后4--6周时达到峰值,与蚊虫从卵到成蚊的发育周期一致。Sehi等\citep{sehi2025}在科特迪瓦的研究中采用DLNM量化了气候因素的非线性滞后效应,发现温度-发病率关系在25$^{\circ}$C处存在明显的拐点效应。Cheng等\citep{chengy2025}通过城市级别的时间序列回归模型揭示了广东省不同城市间登革热气候敏感性的异质性——珠三角城市的温度敏感性显著高于粤东和粤北城市。

统计模型的优势在于其成熟的统计推断框架和良好的可解释性,可以直接输出置信区间和显著性检验结果。然而,这类模型存在根本性局限:(1)其本质上是"相关性"建模,无法揭示传播的因果机制;(2)对训练数据范围以外的气候条件缺乏外推能力;(3)难以整合蚊虫动力学等生物学过程知识。

\subsubsection{机制模型}

机制模型(Mechanistic Model / Compartmental Model)从传染病传播的生物学过程出发,将人群(和蚊虫种群)划分为不同状态的仓室,以常微分方程组(ODE)刻画各仓室间的转移动力学。经典的Ross--Macdonald模型及其SEIR扩展形式是蚊媒传染病建模的理论基石。Smith等\citep{smith2012}对Ross--Macdonald模型的百年发展史进行了全面回顾,阐述了该模型在疟疾和登革热研究中的核心地位及其数学性质。

Mordecai等\citep{mordecai2017}通过将温度依赖的蚊虫生物学参数(叮咬率、存活率、EIP等)逐一整合入向量-宿主传播模型,实现了对$R_0$随温度变化的机制性刻画,预测最优传播温度为29$^{\circ}$C。Li等\citep{li2019pnas}在PNAS上发表的工作具有里程碑意义:他们在SIR框架下将传播效率$\beta(t)$参数化为温度的自然样条函数(B-spline),通过轨迹匹配方法反演出美洲和东南亚多国(包括墨西哥、巴西、泰国等8个国家)登革热的温度依赖性传播曲线。该研究发现:(1)所有国家的$\beta(T)$曲线形态一致,均在29$^{\circ}$C附近达到峰值;(2)温度变异可以解释登革热发病率年际变化的40--60\%。然而,Li等的方法存在两个重要局限:(a)仅考虑了温度单因素,未纳入降水和湿度的调控效应;(b)使用的是非参数样条函数(无显式解析表达式),难以进行物理解读和跨域迁移。

Guo等\citep{guo2024}在广州登革热建模中引入了完整的SEI-SEIR耦合结构,显式建模蚊虫种群的易感-暴露-感染(SEI)动力学和人群的SEIR动力学,提升了模型的生物学真实性。Zhu等\citep{zhu2016}和Liu等\citep{liuy2023}分别在Ross--Macdonald框架下引入了随机过程(用于刻画传播的随机性)和空间异质性(用于刻画城市内不同区域的传播差异)。Din等\citep{din2021}从数学角度分析了分数阶SEIR模型的Lyapunov稳定性和Hopf分支行为。机制模型的核心优势在于物理可解释性强、模型参数具有明确的生物学含义(如潜伏期、恢复期等),但其参数估计的高维性和不确定性以及对详细数据的依赖限制了其在复杂环境下的实际应用。

\subsubsection{人工智能融合方法}

近年来,深度学习和人工智能方法在传染病建模中的应用日益增多,形成了"数据驱动+物理约束"的融合范式。Chen等\citep{chen2018node}提出的神经常微分方程(Neural Ordinary Differential Equation, Neural ODE)通过将神经网络嵌入微分方程的右端项,实现了连续时间动力系统与深度学习的有机结合,为机制建模与数据驱动方法的融合提供了坚实的理论基础。

物理信息神经网络(Physics-Informed Neural Network, PINN)是这一融合范式的代表性方法,其核心思想是将微分方程的残差作为正则化项加入神经网络的损失函数,使训练后的网络既拟合观测数据又满足物理定律约束。Caldwell等\citep{caldwell2021}将PINN应用于SIR模型的参数反演,利用有限的病例时间序列成功估计了时变传播率$\beta(t)$和恢复率$\gamma$,展示了PINN在传染病动力学参数估计中的巨大潜力。

在纯数据驱动方向,Baker等\citep{baker2022}利用长短期记忆网络(Long Short-Term Memory, LSTM)融合气象变量和社会经济指标,对东南亚多个城市的登革热周病例数进行了逐步预测,在1--4周预测窗口内取得了优于ARIMA和GAM等传统方法的预测精度。Mills等\citep{mills2024}和Ahmad等\citep{ahman2025}系统比较了LSTM、Transformer、时间卷积网络(TCN)等深度学习架构在传染病预测中的表现,发现Transformer架构在捕捉长程依赖方面具有独特优势。然而,纯数据驱动的深度学习方法存在固有的"黑箱"问题——模型内部参数缺乏生物学意义,预测结果难以给出流行病学上的因果解释,在极端事件和分布偏移条件下的鲁棒性也令人担忧。

Zhang等\citep{zhangs2021}和Yang等\citep{yang2023}尝试将图神经网络(Graph Neural Network, GNN)引入登革热的空间传播建模,利用城市间的交通网络和人口流动数据构建空间邻接图,取得了优于独立建模的空间预测精度。Li等\citep{lir2024}将注意力机制与时序模型结合,开发了面向登革热早期预警的多步预测框架。Nikparvar等\citep{nikparvar2021}对多种机器学习方法在登革热风险预测中的表现进行了系统比较,发现随机森林和梯度提升树在中短期(1--8周)预测中表现最优,但在长期趋势预测上不及机制模型。

\subsubsection{符号回归方法}

符号回归(Symbolic Regression, SR)是近年来兴起的一种自动化数学公式发现方法,代表了机器学习与科学发现的交叉前沿。与传统的回归方法(如线性回归、多项式回归)预先假定函数形式不同,符号回归在由基本数学运算(加减乘除、指数、对数、三角函数等)构成的广阔组合空间中进行搜索,自动发现能够最优拟合数据的解析表达式\citep{cranmer2023}。这一方法的核心价值在于其输出不是一个不可解释的参数向量,而是一个人类可阅读、可验证的数学公式。

Murphy等\citep{murphy2021}将符号回归应用于物理学领域,成功从实验数据中重新发现了若干经典物理定律(如牛顿万有引力定律的函数形式),展示了该方法在科学发现中的革命性潜力。Holm等\citep{holm2019}和Makke等\citep{makke2024}分别从理论和综述的角度探讨了符号回归在复杂系统建模中的数学基础、搜索策略和应用前景。Cranmer\citep{cranmer2023}开发的PySR工具通过多目标遗传编程在Pareto前沿上同时优化公式的精度和复杂度,成为目前应用最广的开源符号回归框架。

在传染病领域,Zhang等\citep{zhang2024plos}在PLOS Computational Biology上发表了一项具有开创意义的研究:他们将符号回归与SIR动力学模型相结合,提出了一种两阶段的传播效率公式发现方法。第一阶段利用逆问题从流行曲线中反演出时变传播率$\beta(t)$序列;第二阶段以PySR对$\beta(t)$与气候变量(温度、绝对湿度)进行符号回归搜索,自动发现了新冠病毒传播效率与温度、湿度的显式函数关系。然而,Zhang等的工作聚焦于COVID-19这一呼吸道传染病,未涉及登革热等蚊媒传染病的特殊传播机制。Kamyshnyi等\citep{kamyshnyi2026}将类似思路应用于流感的传播建模,发现了温度-湿度与流感传播率之间的经验公式。Fajardo等\citep{fajardo2024}和Adeoye等\citep{adeoye2025}分别在不同传染病场景下验证了符号回归的有效性和鲁棒性。Ouedraogo等\citep{ouedraogo2025}将符号回归与元学习结合,通过在多个传染病数据集上的预训练提升了公式发现的泛化能力和收敛速度。White等\citep{white2025}探讨了符号回归在生态学和流行病学模型发现中的方法论地位和应用前景。

\subsection{研究目标与创新}

综合以上文献分析,当前登革热建模领域存在以下四方面不足:

(1)Li等\citep{li2019pnas}在PNAS上的里程碑工作仅考虑了温度单因素对传播效率的调控,完全忽略了降水和湿度这两个被大量流行病学证据证实的重要驱动因子,导致模型对热带季风气候区(如华南地区)的适用性受限。

(2)Zhang等\citep{zhang2024plos}的方法针对呼吸道传染病(COVID-19)设计,未考虑蚊媒传播的特殊性——登革热的传播链中存在"人-蚊-人"的双宿主循环,需要额外建模蚊虫种群动力学和降水等蚊媒生态相关变量。

(3)大多数AI融合模型(如LSTM、PINN等)虽然取得了较高的预测精度,但停留在"黑箱"预测阶段,未能从深度学习模型中提取出可解释的传播机制公式,限制了其科学发现和政策指导价值。

(4)多城市空间迁移验证在登革热领域仍严重不足。大多数模型仅在单一城市或国家尺度进行验证,缺乏对公式跨区域泛化能力的系统评估。

针对上述问题,本文的研究目标明确如下:

\begin{enumerate}[leftmargin=2em]
\item 构建SEIR仓室模型+神经网络的耦合框架,通过逆问题反演和深度学习实现从温度、降水、湿度三个气象变量到传播效率$\bprime$的非线性映射。
\item 利用符号回归对训练好的神经网络进行知识蒸馏,自动发现传播效率$\bprime$关于温度、降水、湿度的显式闭合公式。
\item 将广州发现的公式迁移至广东省16个地级市,通过多种标度方案和评价指标系统验证其空间泛化能力,建立"先排序后量级"的双层评价体系。
\item 与PNAS(Li等, 2019)和Zhang等(2024)的方法进行对比分析,阐明本文三阶段框架在蚊媒传染病建模中的独特优势和方法论贡献。
\end{enumerate}

本文的创新点可概括为四个方面:

\begin{enumerate}[leftmargin=2em]
\item \textbf{方法创新:}提出"SEIR + 神经网络 + 符号回归"的三阶段端到端机制发现流程。与现有方法相比,该框架首次将知识蒸馏思想应用于蚊媒传染病的传播效率公式发现,实现了从黑箱到白箱的自动化转换。
\item \textbf{公式创新:}通过知识蒸馏从353个参数的MLP神经网络中提取出仅含10个参数的二次多项式闭合公式,在几乎不损失精度的条件下($R^{2}=0.999987$)实现了参数压缩比{$>$}35:1,同时保证了每个系数的物理可解释性。
\item \textbf{验证创新:}在广东省16个地级市进行了系统的空间迁移验证,年度排序Spearman $\rho=0.900$($p=2.05\times10^{-6}$)证明了公式的强泛化能力。这是目前已知的首次在省级多城市尺度对气候驱动登革热公式进行空间交叉验证。
\item \textbf{评价创新:}提出"先排序后量级"的分层评价框架,将流行病学上最有决策意义的城市间风险排序能力作为首要评价维度,为资源有限地区的公共卫生资源配置提供了分级决策依据。
\end{enumerate}

\subsection{全文结构}

本文共分为四个部分,逻辑线索为"综述$\to$发现$\to$验证$\to$总结"。

第一部分为前言(本节),系统综述了登革热的全球和中国流行态势、气候因素与蚊媒传播效率的定量关系、现有建模方法(统计模型、机制模型、AI融合模型、符号回归)的进展与不足,在此基础上凝练了本文的研究目标与创新点。

第二部分为单城市机制发现(第\ref{sec:part1}节),以广州市为案例城市,详细介绍了三阶段框架(SEIR仓室模型$\to$神经网络$\to$符号回归)的数据基础、模型构建、训练策略及结果分析,重点展示Phase 1的预测性能和Phase 2发现的传播效率闭合公式。

第三部分为多城市验证(第\ref{sec:part2}节),将广州发现的传播效率公式迁移至广东省16个地级市,介绍三种城市间标度方案和"先排序后量级"的双层评价策略,呈现年度排序、城市月度拟合、新旧数据对比和敏感性分析等验证结果。

第四部分为总结与展望(第\ref{sec:conclusion}节),凝练全文的五条主要结论和三项创新贡献,坦诚讨论五个方面的局限性并展望未来工作方向。
