\documentclass[12pt,a4paper,oneside]{ctexart}
\setCJKmainfont{FandolSong}
\setCJKsansfont{FandolHei}
\setmainfont{TeX Gyre Termes}
\usepackage[top=2.5cm,bottom=2.5cm,left=3cm,right=2.5cm]{geometry}
\usepackage{setspace}
\setstretch{1.35}
\setlength{\parindent}{2em}
\ctexset{
  section={format=\centering\sffamily\zihao{3},beforeskip=24pt,afterskip=18pt},
  subsection={format=\raggedright\sffamily\zihao{4},beforeskip=18pt,afterskip=12pt},
  subsubsection={format=\raggedright\sffamily\zihao{-4},beforeskip=12pt,afterskip=6pt},
}
\usepackage{amsmath,amssymb,amsfonts}
\usepackage{graphicx}
\usepackage{booktabs}
\usepackage{multirow}
\usepackage{tabularx}
\usepackage{longtable}
\usepackage{float}
\usepackage{caption}
\captionsetup{font=footnotesize,labelsep=space}
\usepackage{subcaption}
\usepackage[colorlinks,linkcolor=blue,citecolor=blue,urlcolor=blue]{hyperref}
\usepackage{cleveref}
\usepackage{enumitem}
\usepackage{url}
\usepackage{natbib}
\bibliographystyle{unsrtnat}
\heavyrulewidth=1.5pt
\lightrulewidth=0.75pt
\newcommand{\bprime}{\beta^{\prime}}
\begin{document}

% ===================== 封面 =====================
\begin{titlepage}
\centering
\vspace*{2cm}
{\sffamily\zihao{2}\bfseries 本科毕业论文}\par
\vspace{2cm}
{\sffamily\zihao{-2}\bfseries 基于神经网络耦合动力学模型的\\[6pt]登革热传播效率发现与多城市验证}\par
\vspace{3cm}
{\zihao{3}
\renewcommand{\arraystretch}{1.8}
\begin{tabular}{r l}
学\hspace{2em}院 & ~~~~~~~~~~~~~~~~~~~~~~~~~~~~~~~~~~~~~~~~ \\
专\hspace{2em}业 & ~~~~~~~~~~~~~~~~~~~~~~~~~~~~~~~~~~~~~~~~ \\
学\hspace{2em}号 & ~~~~~~~~~~~~~~~~~~~~~~~~~~~~~~~~~~~~~~~~ \\
姓\hspace{2em}名 & ~~~~~~~~~~~~~~~~~~~~~~~~~~~~~~~~~~~~~~~~ \\
指导教师 & ~~~~~~~~~~~~~~~~~~~~~~~~~~~~~~~~~~~~~~~~ \\
\end{tabular}
}\par
\vfill
{\zihao{3} 二〇二六年六月}
\end{titlepage}
\setcounter{page}{1}
\pagenumbering{Roman}
