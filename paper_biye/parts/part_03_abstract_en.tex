
% ===================== 英文摘要 =====================
\newpage
\begin{center}
{\sffamily\zihao{3} ABSTRACT}
\end{center}
\addcontentsline{toc}{section}{ABSTRACT}
\vspace{12pt}

Dengue fever is one of the most important mosquito-borne infectious diseases globally, with its transmission dynamics non-linearly regulated by climatic factors including temperature, precipitation, and humidity. Existing studies predominantly rely on purely statistical models or mechanistic models with fixed parameters, making it difficult to simultaneously achieve interpretability and high accuracy. This thesis proposes a three-stage ``SEIR + Neural Network + Symbolic Regression'' framework, aiming to automatically extract interpretable transmission efficiency formulas from data-driven black boxes and validate their generalization capability at the multi-city scale.

\textbf{Part I (Single-City Mechanism Discovery):} Using Guangzhou as a case study, biweekly dengue case counts from 2005--2019, NOAA meteorological data (temperature, precipitation, relative humidity), and the Breteau Index (BI) for mosquito surveillance were incorporated. A discrete-time SEIR compartmental model was first constructed, and the time-varying transmission efficiency $\bprime(t)$ series was obtained through inverse problem estimation. A 3-layer multilayer perceptron (MLP, 353 parameters) was then trained to approximate the mapping between $\bprime$ and meteorological variables. Finally, symbolic regression was applied to distill the neural network into an explicit closed-form formula. Training employed a leave-one-out cross-validation scheme with 2014 as the test year. Results showed that Phase~1 (SEIR+NN) predictions achieved a Pearson correlation coefficient $r=0.612$ and Spearman rank correlation $\rho=0.705$ with observed biweekly case counts, with $R^{2}_{\log}=0.450$. The Phase~2 symbolic regression discovered a quadratic polynomial formula (including quadratic terms and interaction terms of temperature, humidity, and precipitation) with $R^{2}=0.999987$, nearly perfectly reproducing the neural network output.

\textbf{Part II (Multi-City Validation):} The formula discovered in Guangzhou was transferred to all 16 prefecture-level cities in Guangdong Province. Three inter-city scaling approaches (Guangzhou scaling, non-Guangzhou linear, non-Guangzhou log-linear) were introduced, with a ``ranking-first, magnitude-second'' two-tier evaluation strategy. Results showed that the Spearman $\rho=0.900$ ($p=2.05\times10^{-6}$) for annual case ranking across 16 cities, with non-Guangzhou MAE$=61.8$ cases and RMSE$=116.8$ cases. The median monthly-level Pearson $r$ was 0.481 and median Spearman $\rho$ was 0.469.

The innovations of this thesis include: (1)~proposing an end-to-end mechanism discovery pipeline coupling neural networks with dynamical models and symbolic regression; (2)~obtaining a physically interpretable closed-form transmission efficiency formula through knowledge distillation; (3)~validating the spatial transferability of the formula across 16 cities; and (4)~adopting a ranking-first evaluation framework suitable for risk stratification in resource-limited settings.

\vspace{12pt}
\noindent\textbf{Keywords:} Dengue fever; SEIR model; Neural network; Symbolic regression; Climate-driven; Transmission efficiency; Multi-city validation; Guangdong Province

\newpage
\tableofcontents
\newpage
\pagenumbering{arabic}
\setcounter{page}{1}
