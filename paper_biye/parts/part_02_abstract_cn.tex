

\newpage
\begin{center}
{\sffamily\zihao{3} 摘\quad 要}
\end{center}
\addcontentsline{toc}{section}{摘要}
\vspace{12pt}

登革热是全球最重要的蚊媒传染病之一,每年造成约3.9亿例感染和2.1万例死亡。其传播动力学受温度、降水和湿度等气候因素的非线性调控,但现有研究多采用纯统计模型或固定参数的机制模型,难以同时实现物理可解释性与高预测精度。本文提出一种"SEIR仓室模型{+}神经网络{+}符号回归"的三阶段端到端框架,旨在从数据驱动的黑箱神经网络中自动提取可解释的传播效率闭合公式,并在省级多城市尺度验证其空间泛化能力。

\textbf{第一部分(单城市机制发现)}:以广州市为案例城市,纳入2005--2019年双周分辨率的登革热确诊病例数据、美国国家海洋和大气管理局(NOAA)气象观测数据(温度、降水、相对湿度)以及广州市疾控中心蚊媒布雷图指数(BI)监测数据。研究框架分三步实施。第一步,构建离散时间SEIR仓室模型,通过逆问题优化反演获得时变传播效率序列,其中人群潜伏期速率为5.9天的倒数、恢复速率为14天的倒数。第二步,训练一个含353个参数的轻量级多层感知机(MLP),逼近归一化气象变量到传播效率的非线性映射关系。第三步,利用PySR符号回归工具对神经网络在均匀网格(8000个采样点)上的输出进行知识蒸馏,搜索在Pareto前沿上精度与复杂度最优的解析公式。训练采用留一法交叉验证方案,以2014年极端暴发年作为测试集。

结果表明,第一阶段预测的双周病例数与观测值的Pearson相关系数为0.612、Spearman等级相关为0.705,对数尺度决定系数为0.450,平均绝对误差为51.23例,均方根误差为139.10例。第二阶段符号回归从两类候选公式(物理模板族和二次多项式族)中选出最优公式:包含温度、湿度、降水的线性项、二次项和全部两两交互项的10参数二次多项式,对神经网络输出的拟合决定系数达到0.999987,参数压缩比超过35倍,几乎完美复现了353参数神经网络的输出。公式系数具有明确的物理含义:温度线性正效应与二次负效应构成倒U型响应,最优温度估计约29摄氏度,与Mordecai等(2019)的独立实验估计一致;温度--降水交互项系数显著为正,揭示了高温高雨对传播效率的协同增强效应;降水二次项系数为负,体现了暴雨冲刷对蚊媒栖息地的抑制效应。在2014年测试年中,公式对传播效率年均值的预测偏差仅为0.025\%,证明了对极端气候条件的外推能力。

\textbf{第二部分(多城市迁移验证)}:将广州发现的传播效率公式迁移至广东省全部16个地级市(广州、深圳、佛山、东莞、中山、珠海、惠州、江门、肇庆、湛江、汕头、潮州、揭阳、清远、韶关、梅州)。设计了三种城市间标度方案(广州标度、非广州线性标度、非广州对数线性标度),以补偿各城市在人口规模和蚊密度基线上的差异。采用"先排序后量级"的双层评价策略:第一层以Spearman等级相关衡量城市间年度病例排序的一致性(排序评价不受标度方案影响),第二层以MAE和RMSE衡量绝对量级偏差。

结果显示,16城年度病例排序的Spearman等级相关系数达到0.900(显著性水平极高),排序位次预测误差在两位以内的城市占75\%。非广州城市的年均绝对误差MAE为61.8例、RMSE为116.8例。城市月度拟合的Pearson相关系数中位数为0.481、Spearman等级相关中位数为0.469,珠三角城市的拟合效果显著优于粤东和粤北城市。相比旧版数据集的验证结果(排序相关系数0.713,MAE为504.7例),新版标准化数据集在排序和量级两个维度均有大幅提升。敏感性分析表明,将训练窗口从2005--2019年扩展至2004--2023年后,排序指标保持稳定(0.893),量级指标进一步改善(MAE为52.5例),证明了公式结构的稳健性。

本文的主要创新点包括:(1)提出神经网络耦合动力学模型加符号回归知识蒸馏的端到端机制发现流程,首次将该框架应用于蚊媒传染病;(2)通过知识蒸馏获得具有物理可解释性的传播效率闭合公式,每个系数均可映射到明确的流行病学机制;(3)在广东省16城尺度进行了迄今为止最系统的气候驱动登革热公式空间迁移验证;(4)采用"先排序后量级"的分层评价体系,为资源有限地区的公共卫生风险分级和资源配置提供了分级决策依据。

\vspace{12pt}
\noindent\textbf{关键词:}登革热;SEIR仓室模型;神经网络;符号回归;知识蒸馏;气候驱动;传播效率;多城市验证;广东省
