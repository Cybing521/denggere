

\subsection{结果}

\subsubsection{描述性分析}

广州市2005--2019年共报告登革热确诊病例约64,000例,年际变异极大。从年度报告数据来看,2005--2013年间多数年份报告数百至数千例(中位数约为1,200例),2014年单年即报告超过37,000例,为正常年份的20--30倍,构成了研究期内最极端的暴发事件。2015年后病例数迅速回落至正常水平,2016--2019年年均约为2,500例。这种"脉冲式"暴发模式是登革热在非地方性流行区的典型特征——多数年份维持低水平本地传播,偶尔在特定气候条件和输入病例叠加下引发大规模暴发。

双周病例数的时间序列呈现出强烈的季节性模式。在典型年份中,1--6月(第1--13个双周)为流行间歇期,双周病例数通常为个位数或零;7--8月(第14--17个双周)为上升期,病例数开始攀升;9--10月(第18--22个双周)为高峰期,集中了全年病例的70\%以上;11--12月(第23--26个双周)为下降期。这一季节模式与气温的年内变化存在约4--6周的滞后,反映了从气温升高到蚊虫种群扩增再到病毒传播的级联生态学过程。

逆问题反演得到的时变传播效率$\bprime(t)$序列与气象变量的单因素相关分析显示:$\bprime$与温度的Pearson相关系数为$r=0.51$($p<0.001$),与降水的相关系数为$r=0.38$($p<0.001$),与湿度的相关系数为$r=0.29$($p<0.01$)。温度是传播效率最重要的单一驱动因子,但仅能解释其方差的约26\%($r^2 = 0.26$),说明降水和湿度的独立贡献以及因素间的交互效应不可忽略。$\bprime(t)$的时间序列在2014年暴发期间并未出现显著高于其他年份的峰值,这提示2014年暴发的"放大"效应更多来自易感人群的累积规模和感染力方程中$I_h$的正反馈,而非传播效率$\bprime$本身的异常升高。这一发现对理解登革热暴发的驱动机制具有重要意义。

\subsubsection{Phase 1:SEIR+NN模型结果}

表\ref{tab:phase1}给出了SEIR+NN耦合模型在2014年留一法验证下的核心性能指标。这些指标反映了模型在"已知气象条件但未知病例数据"的条件下,对双周病例数时间序列的预测能力。

\begin{table}[H]
\centering
\caption{Phase 1(SEIR+NN)模型性能指标——广州市留一法验证(2014年为测试年)}
\label{tab:phase1}
\begin{tabular}{lcc}
\toprule
指标 & 数值 & 解释 \\
\midrule
Pearson $r$ & 0.612 & 线性相关中等偏强 \\
Spearman $\rho$ & 0.705 & 排序一致性较高 \\
Kendall $\tau$ & 0.531 & 排序一致性中等 \\
$R^{2}_{\log}$ & 0.450 & 对数尺度可解释45\%方差 \\
MAE & 51.23 & 平均每双周偏差约51例 \\
RMSE & 139.10 & 大偏差集中在暴发高峰 \\
WAPE & 0.384 & 总偏差占总量38.4\% \\
RMSLE & 0.892 & 对数空间拟合中等 \\
\bottomrule
\end{tabular}
\end{table}

结果分析表明:(1)排序指标($\rho=0.705$)显著优于线性相关指标($r=0.612$),说明模型在捕捉病例数的相对高低(哪些双周高发、哪些低发)方面的能力强于精确预测绝对数值。这对于公共卫生预警(需要识别高风险时段)具有直接的实用价值。(2)对数尺度$R^{2}_{\log}=0.450$虽未达到"优秀"水平($>0.7$),但考虑到模型仅使用三个气象变量和蚊媒指数作为输入,完全未纳入人口流动模式、人群免疫背景、公共卫生干预强度和输入性病例数据等重要因素,这一结果已证明了气候变量对登革热传播效率的显著解释力。(3)RMSE(139.10)远大于MAE(51.23),表明预测误差存在显著的偏态分布——大误差集中出现在2014年暴发高峰的少数几个双周。

图\ref{fig:phase1}展示了模型预测病例数与观测值的时间序列对比。

\begin{figure}[H]
\centering
\includegraphics[width=0.85\textwidth]{../results/data2_1plus3/phase1_guangzhou_data2.png}
\caption{Phase 1模型预测的双周病例数与观测值对比(广州市2005--2019年)。阴影区域标记2014年测试年。模型较好地捕捉了季节性波动,但在2014年暴发高峰期存在一定的低估。}
\label{fig:phase1}
\end{figure}

从图中可以观察到:模型在非暴发年份(2005--2013, 2015--2019)的季节性上升和下降趋势拟合良好,峰值时间点的定位基本准确(偏差在1--2个双周以内)。在2014年测试年中,模型成功识别了该年为"高风险年份"(预测峰值显著高于其他年份),但对暴发峰值的绝对量级存在低估,反映了极端事件预测中固有的"回归均值"效应。

\subsubsection{Phase 2:符号回归结果}

表\ref{tab:phase2compare}比较了物理模板族和二次多项式族在逼近MLP神经网络输出方面的拟合精度。

\begin{table}[H]
\centering
\caption{两类候选公式的$\bprime$拟合精度对比}
\label{tab:phase2compare}
\begin{tabular}{lccc}
\toprule
公式族 & $R^{2}$ & 参数数量 & MAE$_{\bprime}$ \\
\midrule
物理模板(Gaussian $\times$ 指数饱和) & 0.9973 & 6 & $8.2\times10^{-3}$ \\
二次多项式(含交互项) & 0.999987 & 10 & $5.6\times10^{-4}$ \\
\bottomrule
\end{tabular}
\end{table}

物理模板族虽然以仅6个参数达到了$R^{2}=0.9973$的较高精度,但其残差中存在系统性的空间结构——在高温高雨组合区域和低温低湿区域的拟合偏差最为显著,说明乘积耦合的函数形式无法充分刻画因素间的交互效应。二次多项式族以4个额外参数的代价(10 vs 6)将$R^{2}$从0.9973提升至0.999987(提升$>$两个数量级),且残差在整个$[0,1]^3$定义域上均匀分布,无明显的空间结构。因此选择二次多项式作为最终公式形式。

最终公式为:
\begin{equation}
\bprime = \max\!\big(0,\; a_0 + a_T T + a_H H + a_R R + a_{TT} T^2 + a_{HH} H^2 + a_{RR} R^2 + a_{TH} TH + a_{TR} TR + a_{HR} HR\big)
\label{eq:final_formula}
\end{equation}

表\ref{tab:coeffs}列出了10个最优拟合系数及其物理解释。

\begin{table}[H]
\centering
\caption{二次多项式公式的拟合系数及物理解释}
\label{tab:coeffs}
\begin{tabular}{lrl}
\toprule
参数 & 数值 & 物理解释 \\
\midrule
$a_0$ & $-0.0312$ & 截距(基线水平),负值确保低气候条件下$\bprime\approx 0$ \\
$a_T$ & $+0.1847$ & 温度线性正效应——升温促进传播 \\
$a_H$ & $+0.0523$ & 湿度线性正效应——增湿促进蚊虫存活 \\
$a_R$ & $+0.0691$ & 降水线性正效应——降水提供孳生水体 \\
$a_{TT}$ & $-0.0834$ & 温度二次项(负值$\Rightarrow$倒U型响应) \\
$a_{HH}$ & $-0.0219$ & 湿度二次项(负值$\Rightarrow$饱和效应) \\
$a_{RR}$ & $-0.0478$ & 降水二次项(负值$\Rightarrow$暴雨冲刷效应) \\
$a_{TH}$ & $+0.0156$ & 温度-湿度交互(正值$\Rightarrow$协同增强) \\
$a_{TR}$ & $+0.0389$ & 温度-降水交互(正值$\Rightarrow$强协同增强) \\
$a_{HR}$ & $-0.0103$ & 湿度-降水交互(负值$\Rightarrow$弱拮抗效应) \\
\bottomrule
\end{tabular}
\end{table}

\begin{figure}[H]
\centering
\includegraphics[width=0.85\textwidth]{../results/data2_1plus3/phase2_formula_fit_data2.png}
\caption{符号回归公式输出与MLP神经网络输出的散点对比(8000个网格点,$R^{2}=0.999987$)。虚线为$y=x$参考线。}
\label{fig:phase2}
\end{figure}

\textbf{公式的物理解读。}公式中每个系数都可以映射到明确的流行病学机制:

\begin{itemize}[leftmargin=2em]
\item $a_T > 0$且$a_{TT} < 0$构成了传播效率对温度的\textbf{倒U型响应}。由$a_T$和$a_{TT}$可估算最优温度$T_{\text{opt}} = -a_T/(2a_{TT}) \approx 1.11$(归一化单位),反归一化后约对应28--29$^{\circ}$C,与Mordecai等\citep{mordecai2019}独立估计的29$^{\circ}$C高度一致。

\item $a_R > 0$且$a_{RR} < 0$反映了降水对传播效率的\textbf{先促后抑}效应。适量降水为蚊虫提供孳生水体($a_R>0$),但暴雨冲刷幼虫栖息地($a_{RR}<0$)导致边际效应递减\citep{nosrat2021}。

\item $a_{TR} = +0.0389$是所有交互项中绝对值最大的,揭示了温度与降水的\textbf{强协同增强效应}——高温加速蚊虫发育和病毒复制,充足降水同时提供繁殖水体,两者形成正反馈循环。这与Leung等\citep{leung2023}报告的超加性交互效应一致。

\item $a_{HR} = -0.0103$表现为\textbf{弱拮抗效应}——当湿度和降水同时处于极高水平时,过饱和的水汽环境可能对蚊虫的飞行活性产生轻微抑制。
\end{itemize}

\subsubsection{2014年极端暴发分析}

2014年是研究期内的极端暴发年,也是留一法验证的测试年份。对2014年$\bprime$进行公式预测与逆问题反演值的精细比较,结果如下:

公式预测的2014年$\bprime$年均值为0.183539,逆问题反演的$\bprime$年均值为0.183585。两者的绝对差异为$4.6 \times 10^{-5}$,相对差异仅为$0.025\%$。这一极小的偏差表明:(1)公式在极端气候条件下仍保持了极高的$\bprime$拟合精度;(2)尽管2014年的数据未参与公式训练(留一法),公式仍能准确外推到该年的气候特征空间,证明了其泛化能力。

进一步分析2014年的双周$\bprime$动态,公式预测值与反演值在所有26个双周时间步上的相关系数$r=0.987$,说明公式不仅在年均值上准确,在季节内的时间变化模式上也高度一致。

\begin{figure}[H]
\centering
\includegraphics[width=0.85\textwidth]{../results/data2_1plus3/outbreak_2014_beta_compare_data2.png}
\caption{2014年极端暴发年的双周$\bprime$对比:二次多项式公式预测值(蓝线)与逆问题反演值(橙线)。两条曲线几乎完全重合。}
\label{fig:2014}
\end{figure}

值得注意的是,虽然公式精确预测了2014年的$\bprime$水平,但2014年的$\bprime$年均值(0.184)与2013年(0.181)和2015年(0.179)并无显著差异。这一事实再次印证了上述描述性分析的发现:2014年暴发的"放大"效应并非源于$\bprime$的异常升高,而是源于SEIR模型中感染力方程的非线性放大机制——当初始感染人数$I_h$因输入性病例增多而升高时,$\lambda = \bprime \cdot \hat{M}/N_h \cdot I_h$的正反馈导致指数增长。

\subsection{讨论}

\textbf{方法的可学习性。}本部分的核心发现之一是:气象变量到传播效率的映射关系是可学习的(learnable)。尽管MLP仅有353个参数(远小于典型深度学习模型的数万至数百万参数),在非暴发年份仍能较好地捕捉$\bprime$的季节波动模式。Phase 1的$\rho=0.705$表明,模型在70\%以上的时间步中正确预测了传播效率的相对排序。$R^{2}_{\log}=0.450$虽未达到"优秀"水平,但模型仅以3个气象变量和1个蚊媒代理指标为输入,完全未纳入人口流动、群体免疫状态、公共卫生干预力度、输入性病例数量和社会经济条件等重要影响因素。在这一"极简输入"的约束下,气候变量已展示了对传播效率的显著且可量化的解释力。

\textbf{知识蒸馏的有效性。}Phase 2的符号回归将353参数的黑箱MLP压缩为仅含10参数的显式二次多项式公式,参数压缩比{$>$}35:1,而拟合精度损失可忽略($R^{2}=0.999987$)。这一结果表明:MLP学到的"知识"虽然分布在353个权重参数中,但其本质结构可以用一个低维的二次响应面精确描述。从数学角度看,这暗示了气象变量到传播效率的映射在归一化空间中近似为一个光滑的二次曲面,交互效应可由二阶交叉项充分刻画。

\textbf{公式系数的外部验证。}公式中的关键系数可以与独立的流行病学研究进行交叉验证:(1)最优传播温度的公式估计(28--29$^{\circ}$C)与Mordecai等\citep{mordecai2019}基于蚊虫生物学参数的独立估计(29$^{\circ}$C)高度一致,增强了公式在温度维度上的科学可信度。(2)降水的先促后抑效应($a_R>0$, $a_{RR}<0$)与Nosrat等\citep{nosrat2021}和Zhou等\citep{zhou2025}报告的非线性降水-登革热关系完全吻合。(3)温度-降水的强正交互($a_{TR}>0$)与Leung等\citep{leung2023}发现的超加性效应一致。(4)湿度阈值效应($a_H>0$, $a_{HH}<0$)与Wu等\citep{wu2018}报告的76\%湿度阈值兼容。

\textbf{极端年份辨识能力的局限。}虽然公式对$\bprime$的年均预测偏差仅为$0.025\%$,但公式无法单独预测暴发的绝对规模——2014年与相邻年份的$\bprime$差异不足1\%,而实际病例数差异却超过20倍。这一"$\bprime$-病例数解耦"现象揭示了SEIR模型中的非线性放大机制:微小的$\bprime$差异在正反馈条件下可被指数放大。因此,精确预测暴发规模需要结合$\bprime$公式与完整的SEIR动力学模型,而非仅依赖$\bprime$值本身。
