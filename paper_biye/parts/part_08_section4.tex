

\section{总结与展望}
\label{sec:conclusion}

\subsection{主要结论}

本文围绕"气候因素如何定量驱动登革热传播效率"这一核心科学问题,提出了"SEIR仓室模型+神经网络+符号回归"的三阶段端到端机制发现框架。以广州市为案例城市进行传播效率公式的数据驱动发现,并在广东省16个地级市进行了系统的空间迁移验证。经过严格的留一法时间交叉验证(2014年)和多城市空间交叉验证,本文得出以下五条主要结论:

\begin{enumerate}[leftmargin=2em]
\item \textbf{传播效率可学习。}SEIR仓室模型与353参数MLP神经网络的耦合框架成功学习了温度、降水和湿度三个气象变量到传播效率$\bprime$的非线性映射关系。在2014年极端暴发年的留一法验证中,模型对双周病例数的预测获得了Pearson $r=0.612$(中等相关)和Spearman $\rho=0.705$(较强排序相关)的精度。模型在非暴发年份的季节峰值时间预测误差在1--2个双周以内,具有实用的公共卫生预警价值。这一结果在"极简输入"(仅三个气象变量和一个蚊媒代理指标)的约束下实现,证明了气候因素对登革热传播效率的显著且可量化的解释力。

\item \textbf{公式可蒸馏。}符号回归成功从353参数的黑箱MLP神经网络中蒸馏出一个仅含10参数的二次多项式闭合公式,参数压缩比{$>$}35:1。蒸馏后的公式对MLP在8000个网格点上输出的拟合精度达到$R^{2}=0.999987$,几乎完美复现了神经网络的非线性响应面。这一高精度蒸馏结果表明:(1)气象变量到传播效率的映射在归一化空间中本质上是一个光滑的二次曲面;(2)知识蒸馏作为"黑箱$\to$白箱"的方法论工具在传染病建模中是有效的。

\item \textbf{公式可解释。}公式中的每一个系数都可以映射到明确的流行病学机制:温度线性正效应($a_T>0$)和二次负效应($a_{TT}<0$)构成倒U型响应,最优温度估计28--29$^{\circ}$C与Mordecai等的独立实验估计一致;降水先促后抑效应($a_R>0$, $a_{RR}<0$)与Nosrat等和Zhou等的全球元分析结论吻合;温度-降水强正交互($a_{TR}>0$)与Leung等的超加性效应发现一致。公式的每一个系数都通过了与独立文献的交叉验证,增强了其科学可信度。

\item \textbf{公式可迁移。}广州发现的传播效率公式在省级16城市年度排序验证中获得了$\rho=0.900$($p=2.05\times10^{-6}$),排序位次预测误差在$\pm 2$以内的城市占75\%。非广州城市的年均MAE$=61.8$例、RMSE$=116.8$例。这一排序精度远优于随机基准,有力证明了气候驱动公式的空间泛化能力。更重要的是,公式仅通过各城市各自的气象数据即可预测相对风险,无需城市特异的流行病学数据,具有显著的实际应用前景。

\item \textbf{评价体系合理。}"先排序后量级"的双层评价策略有效揭示了登革热传播的双层驱动结构:气候驱动的"传播潜力"决定了城市间风险排序(可跨城市迁移),而非气候的城市特异性因素调制了实际病例量级(需额外标度)。这一分层评价框架为公共卫生决策提供了清晰的指导:在资源有限时,应优先利用排序结果进行城市间的风险分级和资源配置。
\end{enumerate}

\subsection{创新贡献}

本文的创新贡献可从方法论、科学发现和应用验证三个层面加以总结:

\begin{enumerate}[leftmargin=2em]
\item \textbf{方法论创新:}首次将"神经网络耦合仓室动力学模型+符号回归知识蒸馏"的完整框架应用于蚊媒传染病的传播效率公式发现。与PNAS(Li等, 2019)的单因素样条方法和Zhang等(2024)的呼吸道传染病方法相比,本文的三阶段框架实现了三个突破:(a)多因素(温度+降水+湿度)的同时建模;(b)从非参数到参数的自动化转换(知识蒸馏);(c)首次在蚊媒传染病场景下验证了该框架的可行性。这一方法论框架具有广泛的可移植性,可推广至疟疾、寨卡、基孔肯雅热等其他蚊媒传染病。

\item \textbf{科学发现:}发现了登革热传播效率关于温度、降水和湿度的显式闭合二次多项式公式,首次定量揭示了三个气象因素的二阶非线性效应和两两交互效应。特别是温度-降水强正交互项($a_{TR}=+0.0389$)的发现,为理解"高温叠加充沛降水为何易触发登革热暴发"这一长期流行病学观察提供了定量的数学解释。

\item \textbf{应用验证:}在广东省16个地级市尺度进行了迄今为止最系统的气候驱动登革热公式空间迁移验证。$\rho=0.900$的排序精度不仅证明了公式的泛化能力,更证明了"气候是广东省登革热空间分布的主要驱动力"这一科学假说。提出的"先排序后量级"评价框架为传染病模型的空间验证提供了新的方法论范式。
\end{enumerate}

\subsection{局限性与未来工作}

尽管本文取得了积极的研究成果,仍存在以下五个方面的局限性,每一项都指向了明确的未来工作方向:

\begin{enumerate}[leftmargin=2em]
\item \textbf{蚊媒数据局限与改进路径。}BI蚊媒监测数据仅在广州市可系统获取,其他15个城市不得不依赖共享假设。这一假设在珠三角城市基本成立,但在气候差异较大的粤北城市可能引入系统偏差。未来的改进方向包括:(a)利用Sentinel-2卫星影像和地表温度数据构建蚊虫适宜性空间指数(Mosquito Suitability Index),替代BI作为各城市独立的蚊密度输入;(b)与广东省各市疾控中心合作获取更广泛的BI监测数据。

\item \textbf{人口流动与输入性病例。}本文未显式建模城市间的人口流动和输入性病例对登革热空间传播的贡献。在2014年暴发中,东南亚回国旅客的输入性病例被认为是触发暴发的关键种子事件\citep{cheng2016}。未来可利用手机信令大数据或百度迁徙指数构建城市间人口流动矩阵,建立空间耦合的多城市SEIR网络模型,显式刻画输入性病例的空间传播贡献。

\item \textbf{多血清型免疫动力学。}本文的SEIR模型假设单一血清型且不考虑交叉免疫和抗体依赖增强效应(ADE)。实际上,广东省已检测到DENV 1--4型的共循环,二次异型感染可能因ADE效应导致重症率升高和暴发规模扩大。未来可将模型扩展为多血清型SIR$\times$4框架,引入交叉免疫和ADE参数,以更真实地刻画登革热的免疫流行病学。

\item \textbf{时间分辨率与滞后效应。}当前双周分辨率可能无法充分解析蚊虫世代周期(10--14天)和病毒外潜伏期(7--12天)引入的精细时间滞后效应。未来可将模型的时间步长缩短至周甚至日尺度,并引入分布滞后非线性模型(DLNM)结构显式刻画气候因子的多步滞后效应。

\item \textbf{气候变化预估与风险评估。}本文基于2005--2019年的历史气候数据建模,未考虑未来气候变化情景下传播效率和风险格局的可能演变。未来可将本文发现的$\bprime$公式嵌入CMIP6耦合模式的气候预估输出中,结合SSP1-2.6、SSP2-4.5和SSP5-8.5等社会经济路径情景,评估2030--2100年广东省乃至中国南方地区登革热风险的时空演变趋势\citep{dennington2025},为长期的气候适应性公共卫生规划提供定量依据。
\end{enumerate}
