
% =====================================================================
%  第2章  第一部分:单城市机制发现(广州)
% =====================================================================
\section{第一部分:单城市机制发现(广州)}
\label{sec:part1}

\subsection{引言}

广州市是中国登革热流行最为严重的城市,2005--2019年间累计报告确诊病例约64,000例,占广东省同期总量的50\%以上,占全国总量的35\%以上。广州市地处珠江三角洲核心区域,地理坐标为北纬23.1$^{\circ}$、东经113.3$^{\circ}$,属南亚热带季风海洋性气候,年均温约22$^{\circ}$C、年降水量约1700\,mm、年均相对湿度约77\%。这一气候条件为白纹伊蚊的全年孳生提供了优越环境。2012年广州市常住人口约1426万,城市化率超过85\%,高密度的城市人口和频繁的国际旅行使其成为登革热输入和本地扩增的理想"放大器"。

本部分以广州市2005--2019年的登革热病例数据、NOAA气象数据和蚊媒布雷图指数为基础,构建"SEIR仓室模型+神经网络+符号回归"的三阶段耦合框架。研究流程为:首先通过逆问题从观测的流行曲线中反演出时变传播效率$\bprime(t)$序列(Phase 0);随后训练一个轻量级多层感知机逼近$\bprime$与气象变量之间的非线性映射关系(Phase 1);最后利用符号回归对神经网络进行知识蒸馏,得到传播效率的显式闭合公式(Phase 2)。整个流程采用2014年留一法交叉验证方案,以评估模型对极端暴发事件的泛化能力。

\subsection{数据材料和方法}

\subsubsection{研究区域与数据来源}

\textbf{研究区域。}广州市位于珠江三角洲核心区域,下辖越秀、海珠、荔湾、天河、白云、黄埔、花都、番禺、南沙、从化、增城共11个行政区。本文以2012年(研究时间窗口2005--2019年的中位年份)的常住人口$N_h = 1.426 \times 10^{7}$作为固定基数。采用固定人口的合理性包括以下三点:(1)登革热在广州不构成地方性稳态流行,且病死率极低($<0.1\%$),即便在2014年暴发高峰期对总人口规模的影响也可忽略不计——感染者峰值约占总人口的0.26\%;(2)研究期内广州常住人口从约1200万增至约1530万,增幅约27\%,而登革热年际发病量的变异幅度超过100倍(最低年份约200例,2014年超过37,000例),人口增长的效应远小于疾病本身的年际波动;(3)固定$N_h$可避免引入年际人口估计的不确定性(特别是流动人口的统计偏差),使模型能够聚焦于气候驱动效应的分离和量化。

\textbf{登革热病例数据。}双周汇总的登革热确诊病例数来源于中国疾病预防控制中心传染病报告信息管理系统(National Notifiable Disease Reporting System, NNDRS),经广东省卫生健康委员会按照国家传染病诊断标准(WS 216-2018)进行质量控制和标准化处理后发布\citep{ccm14}。本研究所用数据的时间范围为2005年1月1日至2019年12月31日,共15个完整年份。原始数据为逐日报告数,本文将其按14天为窗口聚合为双周(biweekly)病例数,形成时间序列$\{C_{\text{obs}}(t)\}_{t=1}^{T}$,其中$T \approx 390$个双周时间步。选择双周分辨率的考量是:该粒度既能捕捉登革热季节内的上升和下降动态(相比月度数据),又能有效平滑由报告延迟和就诊日期不确定性导致的周度噪声(相比周数据)。

\textbf{气象数据。}逐日气象观测数据来自美国国家海洋和大气管理局(NOAA)全球历史气候网络日值数据集(Global Historical Climatology Network-Daily, GHCN-D),该数据集经过严格的质量控制(包括空间一致性检验、极值检验和重复记录检验)。本文提取广州站(WMO站号59287,位于天河区五山)的三个核心气象变量:日均温度$T$($^{\circ}$C)、日累积降水量$R$(mm)和日均相对湿度$H$(\%)。原始日值数据按与病例数据相同的双周窗口进行聚合:温度$T$取窗口内日均温的算术平均值,降水$R$取窗口内日降水量的累积值(反映可用于蚊虫孳生的总水量),湿度$H$取窗口内日均值的算术平均值。对于气象数据中的少量缺失值(占比$<2\%$),采用该日前后各7天观测值的线性插值进行填补。

\textbf{蚊媒监测数据。}布雷图指数(Breteau Index, BI)定义为每百户居民住宅中检查发现的伊蚊幼虫阳性容器数量,是世界卫生组织推荐的蚊媒密度核心监测指标。BI$\geq$20被WHO定义为登革热暴发的高风险阈值。广州市的BI数据来源于广州市疾病预防控制中心的常规月度蚊媒监测系统\citep{chengq2023},监测点覆盖全市11个行政区。BI反映的是幼虫期蚊虫密度,而模型中需要的是成蚊丰度。考虑到幼虫-成蚊的发育周期约为7--14天,且BI与成蚊密度之间存在显著正相关\citep{roiz2015},本文将BI经归一化处理后作为成蚊丰度$\hat{M}$的代理指标直接纳入SEIR模型的感染力方程。

\subsubsection{数据预处理}

为消除不同变量量纲差异对神经网络训练收敛性的影响,对所有输入变量进行Min-Max归一化:
\begin{equation}
\tilde{x} = \frac{x - x_{\min}}{x_{\max} - x_{\min}}
\label{eq:minmax}
\end{equation}
其中$x_{\min}$和$x_{\max}$分别为\textbf{训练集}(非测试年数据)上该变量的最小值和最大值。测试年(2014年)的数据使用训练集的归一化参数进行变换,严格确保无数据泄漏(Data Leakage)。归一化后所有变量均映射至$[0, 1]$区间。

对BI数据的预处理额外包括以下步骤:(1)由于BI数据的原始分辨率为月度,本文将其通过线性插值转换为双周分辨率,以与其他变量的时间粒度一致;(2)对极少量缺失月份($<3\%$)的BI值,采用相邻两个月观测值的算术平均进行填补;(3)利用公式(\ref{eq:minmax})将填补后的BI序列归一化为$\tilde{M} \in [0, 1]$,作为模型中的蚊虫丰度代理指标。

\subsubsection{SEIR模型}

本文采用离散时间SEIR仓室模型描述登革热在人群中的传播动力学。该模型将人群划分为易感者(Susceptible, $S$)、暴露者/潜伏期(Exposed, $E$)、感染者(Infectious, $I$)和恢复者(Recovered, $R$)四个仓室,各仓室间的转移由以下连续时间动力学方程刻画:

\begin{align}
\frac{dS}{dt} &= -\lambda(t) \cdot S(t) \label{eq:dSdt} \\
\frac{dE}{dt} &= \lambda(t) \cdot S(t) + \eta - \sigma_h \cdot E(t) \label{eq:dEdt} \\
\frac{dI}{dt} &= \sigma_h \cdot E(t) - \gamma \cdot I(t) \label{eq:dIdt} \\
\frac{dR}{dt} &= \gamma \cdot I(t) \label{eq:dRdt}
\end{align}
其中$\lambda(t)$为感染力(Force of Infection),综合描述了易感者在单位时间内被感染的概率。感染力定义为:
\begin{equation}
\lambda(t) = \bprime(t) \cdot \frac{\hat{M}(t)}{N_h} \cdot I_h(t)
\label{eq:foi}
\end{equation}
该公式的物理含义为:感染力等于传播效率$\bprime(t)$(综合反映蚊虫叮咬率、人-蚊和蚊-人传播概率、病毒EIP等因素)乘以蚊虫丰度$\hat{M}(t)$与人口$N_h$之比(即人均蚊密度)再乘以当前感染者人数$I_h(t)$(决定了感染源的可用性)。在频率依赖传播假设下,$\lambda(t)$与$I_h(t)/N_h$成正比。

\textbf{固定参数设定。}本模型中有两个固定的动力学参数:

人群潜伏期速率$\sigma_h = 1/5.9 \text{ d}^{-1}$(即平均潜伏期为5.9天)。该值取自Chan和Johansson\citep{chan2012}对全球发表的登革热内潜伏期(Intrinsic Incubation Period, IIP)研究的系统综述和元分析。IIP定义为人体感染登革病毒后到出现临床症状之间的时间,文献中的估计范围为3--14天,中位数约5.9天。

恢复速率$\gamma = 1/14 \text{ d}^{-1}$(即平均感染持续期为14天)。该值参考Mordecai等\citep{mordecai2017}的温度依赖参数化结果中的中心估计值。感染持续期包括病毒血症期(约5--7天)和后续的免疫恢复期,总计约10--21天,中心估计14天。

\textbf{可训练参数。}输入项$\eta$为可训练参数,代表外部输入源(如来自东南亚的输入性感染者每个时间步向暴露仓室补充的人数),初始化为一个较小的正值($\eta_0 = 0.1$),以保证流行初期有足够的种子感染启动传播链。

\textbf{时间尺度转换。}由于模型的时间步长为双周($\Delta t = 14$天),而$\sigma_h$和$\gamma$的定义单位为天$^{-1}$,在离散时间数值积分时需通过指数衰减关系将连续速率转换为离散概率:
\begin{equation}
P_{\text{trans}} = 1 - e^{-r \cdot \Delta t}
\label{eq:rate2prob}
\end{equation}
其中$r$为连续日速率。例如,$\sigma_h = 1/5.9 \approx 0.169 \text{ d}^{-1}$对应的双周转移概率为$P_\sigma = 1 - e^{-0.169 \times 14} \approx 0.907$;$\gamma = 1/14 \approx 0.071 \text{ d}^{-1}$对应的双周转移概率为$P_\gamma = 1 - e^{-0.071 \times 14} \approx 0.632$。

\textbf{基本再生数。}在给定传播效率$\bprime$和蚊虫丰度$\hat{M}$的条件下,登革热的瞬时基本再生数(即在完全易感人群中一个典型感染者在其整个感染期内产生的二代感染者数)可近似表示为:
\begin{equation}
R_0(t) \approx \frac{\bprime(t) \cdot \hat{M}(t)}{\gamma}
\label{eq:R0}
\end{equation}
当$R_0 > 1$时传播链可持续增长(流行暴发),当$R_0 < 1$时传播链将自然衰减。

\subsubsection{神经网络}

为学习传播效率$\bprime$与气象变量之间的非线性映射关系$\bprime = f_\theta(\tilde{T}, \tilde{R}, \tilde{H})$,本文设计了一个轻量级多层感知机(Multilayer Perceptron, MLP)。网络架构的选择遵循"奥卡姆剃刀"原则,在保证足够表达能力的前提下尽量减少参数量,以降低过拟合风险并便于后续符号回归的知识蒸馏。

网络结构为$3 \to 16 \to 16 \to 1$的全连接前馈网络:
\begin{itemize}[leftmargin=2em]
\item \textbf{输入层:}3个神经元,分别对应归一化后的温度$\tilde{T}$、降水$\tilde{R}$和湿度$\tilde{H}$。不额外包含蚊媒指标BI(已通过SEIR模型的感染力方程(\ref{eq:foi})独立纳入)。
\item \textbf{隐藏层1:}16个神经元,激活函数为Softplus($f(x) = \ln(1 + e^x)$)。选择Softplus而非ReLU的原因是:Softplus在$x=0$处可微,输出严格为正,且其平滑性有利于后续符号回归搜索输出函数的连续解析近似。
\item \textbf{隐藏层2:}16个神经元,激活函数同样为Softplus,提供第二层非线性变换以增强网络对复杂交互效应的表达能力。
\item \textbf{输出层:}1个神经元,激活函数为Sigmoid($f(x) = 1/(1+e^{-x})$),将输出约束在开区间$(0, 1)$内,确保传播效率$\bprime$满足物理约束(正值且有界)。
\end{itemize}

网络总参数量为$(3 \times 16 + 16) + (16 \times 16 + 16) + (16 \times 1 + 1) = 48 + 272 + 17 + 16 = 353$个。选择如此小型的网络基于三方面考量:(1)训练数据量有限(约330个双周样本用于训练),大型网络极易过拟合时间序列中的噪声;(2)小型网络的输出响应面更为平滑,是后续符号回归能够以低复杂度公式成功逼近的前提条件;(3)参数量适中使得训练过程高效稳定,Adam优化器可在500轮内收敛。

\subsubsection{训练策略}

模型训练分为两个顺序执行的步骤(两阶段训练):

\textbf{Step 1:逆问题反演$\bprime(t)$序列。}给定观测病例数时间序列$\{C_{\text{obs}}(t)\}_{t=1}^{T}$和SEIR模型动力学方程(\ref{eq:dSdt})--(\ref{eq:dRdt}),将每个时间步的传播效率$\bprime(t)$视为独立的待估参数(共$T$个自由参数),利用Adam优化器\citep{kingma2015}最小化如下对数空间的均方误差损失函数:
\begin{equation}
\mathcal{L}_1 = \frac{1}{T} \sum_{t=1}^{T} \Big[\log(C_{\text{pred}}(t)+1) - \log(C_{\text{obs}}(t)+1)\Big]^2
\label{eq:loss1}
\end{equation}
其中$C_{\text{pred}}(t) = \sigma_h \cdot E(t) \cdot \Delta t$为模型预测的双周新发病例数(暴露仓室向感染仓室的转移通量)。对数变换的必要性在于:登革热病例数的动态范围极大(从0到数千),直接使用原始空间的MSE会使高峰期的拟合主导优化方向而忽视低发期,对数变换可有效平衡两个时期的拟合权重。Adam优化器的初始学习率设为$10^{-2}$,训练轮次为2000轮,学习率在后500轮线性衰减至$10^{-4}$。

\textbf{Step 2:训练神经网络。}以Step 1反演得到的$\bprime(t)$序列为监督目标,训练MLP网络$f_\theta(\tilde{T}, \tilde{R}, \tilde{H})$学习气象变量到传播效率的映射关系。损失函数设计为精度项和相关性项的组合:
\begin{equation}
\mathcal{L}_2 = \text{MSE}\big(\bprime_{\text{pred}},\; \bprime_{\text{inv}}\big) - 0.5 \cdot \text{Corr}\big(\bprime_{\text{pred}},\; \bprime_{\text{inv}}\big)
\label{eq:loss2}
\end{equation}
其中第一项(MSE)确保网络输出在数值上接近反演的$\bprime$目标值,第二项(加权的Pearson相关系数,系数$-0.5$表示最大化相关性)鼓励网络捕捉$\bprime$的时间变化趋势。两项的权衡使得即便在绝对值拟合存在偏差的情况下,网络仍能准确反映传播效率的相对高低变化。训练使用Adam优化器,学习率$10^{-3}$,训练轮次500,批量大小为全部训练样本(Full-Batch)。

\textbf{留一法验证方案。}以2014年(研究期内最极端的暴发年)为测试集,其余14个年份(2005--2013年和2015--2019年)为训练集。选择2014年的原因是:(1)该年的发病量($>$37,000例)远超正常年份(数百至数千例),构成了"极端事件"预测的理想测试案例;(2)如果模型在如此极端的条件下仍能保持合理的预测精度,将有力证明其泛化能力。

\subsubsection{符号回归}

符号回归阶段的核心目标是对训练好的神经网络进行\textbf{知识蒸馏}(Knowledge Distillation),即用一个人类可读的数学公式来替代黑箱神经网络,同时保持与神经网络输出高度一致的拟合精度。

\textbf{蒸馏数据生成。}为获得符号回归的训练数据集,在归一化变量$(\tilde{T}, \tilde{R}, \tilde{H})$的完整定义域$[0,1]^3$上进行均匀网格采样,每个维度取20个等距点,共生成$20^3=8000$个网格节点。对每个节点输入训练好的MLP,得到对应的$\bprime$预测值。最终构成蒸馏数据集$\mathcal{D} = \{(\tilde{T}_i, \tilde{R}_i, \tilde{H}_i, \bprime_i)\}_{i=1}^{8000}$。使用网格采样(而非仅使用历史观测点)的优势在于:网格覆盖了变量空间的全部组合(包括历史上未观测到的极端组合),使得符号回归搜索到的公式具有更好的定义域覆盖性和外推能力。

\textbf{候选公式族。}本文考虑两类结构不同的候选公式族,以检验先验物理知识对公式发现的影响:

\begin{enumerate}[leftmargin=2em]
\item \textbf{物理模板族:}根据流行病学先验知识预设函数骨架,仅对其中的参数进行优化拟合。温度效应假设为高斯核函数(反映传播效率在最优温度处达峰的钟形曲线):
\begin{equation}
f_T(T) = \exp\!\left(-\frac{(T - T_{\text{opt}})^2}{2\sigma_T^2}\right)
\label{eq:ft}
\end{equation}
其中$T_{\text{opt}}$为最优传播温度(初始值27$^{\circ}$C,参考Mordecai等\citep{mordecai2019}),$\sigma_T$为温度敏感性宽度。降水效应假设为指数饱和函数(反映"先促进后饱和"的非线性响应):
\begin{equation}
f_R(R) = 1 - \exp(-k_R \cdot R)
\label{eq:fr}
\end{equation}
总传播效率为各因素的乘积耦合形式:$\bprime = c_0 \cdot f_T(T) \cdot f_R(R) \cdot f_H(H)$,共6个待估参数。

\item \textbf{二次多项式族:}不预设物理函数形式,允许温度、湿度、降水的线性项、二次项和所有两两交互项自由组合:
\begin{equation}
\bprime = \max\!\Big(0,\; a_0 + a_T T + a_H H + a_R R + a_{TT} T^2 + a_{HH} H^2 + a_{RR} R^2 + a_{TH} TH + a_{TR} TR + a_{HR} HR\Big)
\label{eq:poly}
\end{equation}
共10个待估参数。$\max(0, \cdot)$截断保证了传播效率的物理非负性。该公式族的表达能力覆盖了任意二阶非线性响应面,包括倒U型单峰效应、饱和效应和因素间的交互效应。
\end{enumerate}

\textbf{搜索工具与配置。}使用PySR(Python Symbolic Regression)工具\citep{cranmer2023}执行自动化公式搜索。PySR基于多岛屿并行遗传编程算法,在公式的Pareto前沿(精度-复杂度权衡面)上同时优化。配置参数为:运算符集合$\{+, -, \times, \div, \exp, \log, \text{pow}\}$,种群规模300个候选公式,进化代数500代,Pareto前沿保留复杂度1--20(运算符节点数)的最优公式。

\textbf{最终公式选择准则。}在Pareto前沿上综合考虑公式的拟合精度($R^{2}$)和复杂度(参数数量),遵循以下优先级:(1)$R^{2} > 0.999$为精度门槛;(2)在满足精度门槛的公式中选择参数最少者。

\subsubsection{评估指标}

为全面衡量模型在不同维度上的预测性能,本文采用以下八项评估指标:

\begin{itemize}[leftmargin=2em]
\item \textbf{Spearman等级相关系数$\rho$:}基于秩次的相关度量,衡量预测值与观测值在排序上的一致性,对异常值和非线性关系具有鲁棒性。
\item \textbf{Kendall等级相关系数$\tau$:}另一种基于concordant和discordant pair的排序一致性度量,相比$\rho$对异常值更为稳健。
\item \textbf{Pearson相关系数$r$:}衡量预测值与观测值之间的线性相关强度。
\item \textbf{对数尺度决定系数$R^{2}_{\log}$:}定义为$R^2_{\log} = 1 - \frac{\sum(\log(y_i+1) - \log(\hat{y}_i+1))^2}{\sum(\log(y_i+1) - \overline{\log(y_i+1)})^2}$。对数变换使得该指标对从个位数到万位数的宽动态范围病例数据同等敏感。
\item \textbf{平均绝对误差MAE:}$= \frac{1}{n}\sum|y_i - \hat{y}_i|$,直观衡量预测偏差的平均量级。
\item \textbf{均方根误差RMSE:}$= \sqrt{\frac{1}{n}\sum(y_i - \hat{y}_i)^2}$,对大偏差更为敏感。
\item \textbf{加权绝对百分比误差WAPE:}$= \frac{\sum|y_i - \hat{y}_i|}{\sum y_i}$,给出了误差相对于总量的百分比。
\item \textbf{均方根对数误差RMSLE:}$= \sqrt{\frac{1}{n}\sum(\log(y_i+1)-\log(\hat{y}_i+1))^2}$,在对数尺度上衡量拟合精度。
\end{itemize}
