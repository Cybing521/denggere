

\section{第二部分:多城市机制迁移与验证}
\label{sec:part2}

\subsection{引言}

第一部分在广州单城市尺度成功发现了传播效率$\bprime$的气候驱动闭合公式。然而,一个模型或公式的科学价值和实际应用潜力在很大程度上取决于其\textbf{空间泛化能力}——即从训练城市的数据中学到的映射关系能否迁移至地理位置、气候条件和社会经济特征不同的其他城市。如果公式仅在广州一市有效,其科学贡献将局限于个案描述层面;相反,如果公式在多个城市展现出一致的预测能力,则意味着它捕捉到了具有普遍性的气候-传播效率物理规律。

广东省共辖21个地级市,横跨北纬20.2$^{\circ}$(湛江)至25.5$^{\circ}$(韶关),经度范围从东经109.7$^{\circ}$(湛江)至116.6$^{\circ}$(汕头),气候条件呈现显著的南北梯度——南部沿海城市(如湛江、珠海)接近热带海洋性气候,中部珠三角城市(如广州、佛山、深圳)为典型亚热带季风气候,北部山区城市(如韶关、梅州)则具有中亚热带特征,年均温较珠三角低2--4$^{\circ}$C,降水时空分布也存在差异。此外,各城市在人口规模(从珠海的约189万到广州的约1426万)、城市化程度、医疗卫生体系完善度和登革热防控投入方面均存在显著的空间异质性。

本部分选取数据可得性满足分析要求的16个地级市作为验证对象,通过多种城市间标度方案和分层评价指标,系统评估广州公式在省级空间尺度上的迁移性能。

\textbf{空间交叉验证的流行病学意义。}传统的模型验证多采用时间交叉验证(如留一年法、滚动窗口法),即利用历史数据训练、用未来数据测试。空间交叉验证则关注模型在\textit{未参与训练的地理区域}上的表现,是检验模型空间泛化能力的金标准\citep{roberts2017}。对于传染病模型而言,空间泛化面临两大独特挑战:第一,不同城市的蚊虫种群组成(如白纹伊蚊与埃及伊蚊的相对比例)、密度基线和季节波动模式可能存在差异,这些差异无法仅通过气象变量来解释;第二,社会经济因素(城市化程度、公共卫生资源配置、居民防护意识和行为模式)的空间异质性可能混淆气候效应,使得在训练城市有效的公式在目标城市产生系统性偏差。因此,如果一个纯气候驱动的传播效率公式能在空间交叉验证中取得显著优于随机基准的预测精度,将有力支持以下核心假说:气候是广东省登革热空间分布的\textbf{主要}驱动力,而非气候因素仅构成二阶修正。

\subsection{数据与方法}

\subsubsection{16城市数据集}

本部分的验证城市包括广东省以下16个地级市:广州、深圳、佛山、东莞、中山、珠海、惠州、江门、肇庆、湛江、汕头、潮州、揭阳、清远、韶关、梅州。这16个城市的选取标准为:2005--2019年间至少有10个年份的完整登革热病例报告数据和NOAA气象站观测数据可用。各城市的双周登革热病例数来源与广州一致,均从中国CDC传染病报告信息管理系统获取\citep{ccm14}。气象数据从NOAA GHCN-Daily数据集中提取各城市地理距离最近的气象站的温度、降水和湿度三变量,预处理流程与广州数据相同。

这16个城市在登革热流行强度上存在显著梯度。以2005--2019年年均病例数排序:广州(约4,300例)$\gg$佛山(约680例)$>$深圳(约290例)$>$中山(约180例)$>$东莞(约160例)$>$其他11个城市(年均10--100例不等)。广州以绝对优势位居第一,这也是选择广州作为训练城市的原因之一——丰富的病例数据为SEIR模型的参数反演和神经网络的训练提供了充足的信号量。

由于非广州城市的BI蚊媒监测数据不可得(仅广州市有系统的月度BI监测数据),本文对蚊虫丰度代理指标$\hat{M}(t)$做出如下\textbf{共享假设}:所有16城市共享广州的归一化BI季节模式,即假设省内各城市的蚊虫丰度年内波动形态(何时上升、何时达峰、何时下降)相似,但基线水平可能不同。这一假设的合理性建立在以下生态学依据之上:(1)广东省全域均位于白纹伊蚊的连续分布区内,该蚊种在全省城市环境中普遍存在;(2)BI的季节波动主要由温度和降水两个气候因子驱动,而省内城市的气候季节性模式具有高度相似性(均呈"夏热冬温、夏雨冬干"格局);(3)不同城市之间的BI基线差异可通过后续的标度方案进行补偿。

\subsubsection{城市间标度方案}

各城市在人口规模、蚊虫密度基线和疾病报告能力等方面的差异意味着:即使所有城市具有相同的传播效率$\bprime(t)$,其预测的绝对病例数也会大相径庭。为处理这一"基线异质性"问题,本文设计了三种递进复杂度的城市间标度方案:

\begin{enumerate}[leftmargin=2em]
\item \textbf{广州标度(GZ Scaling):}最简方案,直接使用广州的全部模型参数($\hat{M}$、$N_h$、$\eta$等)对所有城市进行预测,不做任何城市适应性调整。该方案等价于假设所有城市与广州具有相同的蚊密度基线和人口规模,作为最保守的基准(baseline),预期在非广州城市产生显著的量级偏差。其价值在于检验公式的"零适应"迁移能力——即仅通过各城市各自的气象数据,公式能否正确识别城市间的相对风险排序。

\item \textbf{非广州线性标度(Non-GZ Linear):}引入城市特异的线性缩放因子$\alpha_c$以补偿基线差异:
\begin{equation}
\hat{C}_c(t) = \alpha_c \cdot C_{\text{GZ-pred}}(t), \quad \alpha_c = \frac{\bar{C}_c^{\text{obs}}}{\bar{C}_{\text{GZ-pred}}}
\label{eq:linear_scale}
\end{equation}
其中$\bar{C}$为训练期内的年均病例数。线性标度保持了季节内的时间变化模式不变,仅调整了量级基线。

\item \textbf{非广州对数线性标度(Non-GZ Log-Linear):}考虑到城市间量级差异可能呈非线性(如大城市的暴发放大效应更强),引入对数空间的线性回归:
\begin{equation}
\log(\hat{C}_c + 1) = \beta_{0,c} + \beta_{1,c} \cdot \log(C_{\text{GZ-pred}} + 1)
\label{eq:loglinear_scale}
\end{equation}
该方案允许不同城市具有不同的"弹性系数"$\beta_{1,c}$,可以适应非线性的量级映射关系。
\end{enumerate}

\subsubsection{评价策略}

本文采用"先排序后量级"的双层评价策略,这一设计体现了流行病学决策的优先级逻辑:

\textbf{第一层——排序评价。}评估模型能否正确预测各城市间的年度病例排序,即"哪些城市风险高、哪些城市风险低"。使用Spearman $\rho$和Kendall $\tau$度量16城市年度总病例数排序的一致性。排序评价完全不受标度方案的影响(因为单调变换不改变排序),直接反映传播效率公式对空间风险格局的核心识别能力。对于公共卫生资源配置而言,正确的风险排序意味着能够将有限的防控资源优先投向真正的高风险城市,其决策价值远大于精确预测每个城市的绝对病例数。

\textbf{第二层——量级评价。}在排序可靠的前提下,进一步评估预测值与观测值在绝对量级上的接近程度。使用MAE(平均绝对误差)、RMSE(均方根误差)和WAPE(加权绝对百分比误差)等指标。量级评价受标度方案选择的直接影响,同一公式在不同标度方案下的量级指标可能差异显著。

\subsection{结果}

\subsubsection{年度排序验证}

表\ref{tab:ranking}给出了16城市年度总病例数排序的检验结果。

\begin{table}[H]
\centering
\caption{16城市年度病例排序——排序一致性检验}
\label{tab:ranking}
\begin{tabular}{lcc}
\toprule
指标 & 数值 & 统计意义 \\
\midrule
Spearman $\rho$ (16城市) & 0.900 & 极强正排序相关 \\
$p$值 & $2.05 \times 10^{-6}$ & 极显著 \\
Kendall $\tau$ (16城市) & 0.803 & 强正排序相关 \\
非广州城市 MAE & 61.8 & 年均偏差约62例 \\
非广州城市 RMSE & 116.8 & 大偏差集中于少数高报告城市 \\
\bottomrule
\end{tabular}
\end{table}

$\rho = 0.900$($p < 10^{-5}$)表明,公式预测的城市间年度风险排序与实际观测具有极高的一致性。在16个城市中,排序位次预测误差在$\pm 2$以内的城市占75\%(12/16),无一城市出现超过$\pm 4$的严重错排。特别值得注意的是,公式正确识别了广州为最高风险城市,佛山为第二高风险城市,深圳为第三——这三个城市的排序完全准确。

\begin{figure}[H]
\centering
\includegraphics[width=0.9\textwidth]{../results/data2_1plus3/transfer_2014_bars_data2.png}
\caption{16城市年度登革热病例数的排序对比:模型预测(蓝色柱)与实际观测(橙色柱)。城市按实际观测病例数降序排列。}
\label{fig:bars}
\end{figure}

从图\ref{fig:bars}可以直观看出:模型预测的城市排序与实际排序高度吻合。量级方面,非广州城市的年均MAE为61.8例。考虑到这15个城市的年均实际病例数中位数约为85例,62例的MAE对应的相对误差约为73\%。这一量级偏差主要集中在佛山(预测低估约40\%)和中山(预测低估约55\%),反映了这两个城市可能存在未被气候公式捕捉的城市特异性高风险因素(如产业结构导致的外来人口比例高、工业区积水容器密集等)。对于低报告城市(如韶关、梅州、清远),模型的量级预测反而较为准确(MAE$<$15例),因为这些城市的实际病例数本身就很少,气候驱动的基线预测已足够解释其低风险状态。

\subsubsection{城市月度拟合}

图\ref{fig:citygrid}展示了16城市月度(双周聚合为月)病例曲线的模型拟合情况。

\begin{figure}[H]
\centering
\includegraphics[width=0.95\textwidth]{../results/data2_1plus3/all_cities_fit_grid.png}
\caption{16城市月度登革热病例的模型拟合结果(2005--2019年)。红线为模型预测,灰色柱为实际观测。各子图按城市年均病例数降序排列。}
\label{fig:citygrid}
\end{figure}

表\ref{tab:monthly}汇总了各城市月度拟合指标的分布特征。

\begin{table}[H]
\centering
\caption{16城市月度拟合指标汇总}
\label{tab:monthly}
\begin{tabular}{lccc}
\toprule
指标 & 中位数 & 四分位距 & 最优城市 \\
\midrule
Pearson $r$ & 0.481 & [0.321, 0.612] & 广州 ($r=0.612$) \\
Spearman $\rho$ & 0.469 & [0.298, 0.584] & 佛山 ($\rho=0.601$) \\
$R^{2}_{\log}$ & 0.215 & [0.052, 0.378] & 广州 ($R^2_{\log}=0.450$) \\
\bottomrule
\end{tabular}
\end{table}

月度拟合结果呈现出明显的空间梯度:珠三角核心城市群(广州、佛山、深圳、东莞)的$r$中位数达到0.56,显著优于粤东(汕头、潮州、揭阳,$r$中位数0.34)和粤北(韶关、梅州、清远,$r$中位数0.28)。这一梯度的可能原因包括:(1)珠三角城市的气候条件更接近广州(公式的训练城市),气候-传播效率映射的适用性更强;(2)粤东和粤北城市的登革热病例数较少(年均$<50$例),低信噪比导致相关系数的统计功效降低;(3)粤北城市冬季气温显著低于珠三角,蚊虫越冬模式可能不同,影响了BI共享假设的适用性。

\subsubsection{新旧数据对比}

在研究过程中,本文对原始数据集(data1,来自早期文献汇编)和新版数据集(data2,来自CDC正式发布的标准化数据)分别运行了相同的模型管线。表\ref{tab:oldvsnew}对比了两个数据集上的验证结果。

\begin{table}[H]
\centering
\caption{新旧数据集验证结果对比}
\label{tab:oldvsnew}
\begin{tabular}{lcc}
\toprule
指标 & 旧版数据集(data1) & 新版数据集(data2) \\
\midrule
Spearman $\rho$ (16城市) & 0.713 & 0.879 \\
非GZ MAE & 504.7 & 61.8 \\
非GZ RMSE & 782.3 & 116.8 \\
月度 $r$ 中位数 & 0.355 & 0.481 \\
月度 $\rho$ 中位数 & 0.341 & 0.469 \\
\bottomrule
\end{tabular}
\end{table}

新版数据集在所有指标上均有大幅提升:排序$\rho$从0.713提升至0.879(提升23\%),MAE从504.7降至61.8(降低88\%),月度$r$中位数从0.355提升至0.481(提升35\%)。这一显著改善的主要原因是:旧版数据集中部分城市的病例数来源不统一(混合了月报和年报数据),存在重复计数和行政区划归属不一致的问题。新版数据集采用了CDC统一发布的标准化数据,质量控制更为严格。这一对比也提示:在传染病建模研究中,数据质量对模型验证结果的影响可能远大于模型方法本身的改进。

\subsubsection{敏感性分析}

为检验模型对训练时间窗口的敏感性,表\ref{tab:sensitivity}比较了2005--2019年(基准)和2004--2023年(扩展)两个训练窗口的验证结果。

\begin{table}[H]
\centering
\caption{训练窗口敏感性分析}
\label{tab:sensitivity}
\begin{tabular}{lcc}
\toprule
指标 & 2005--2019(基准) & 2004--2023(扩展) \\
\midrule
Spearman $\rho$ & 0.900 & 0.893 \\
非GZ MAE & 61.8 & 52.5 \\
非GZ RMSE & 116.8 & 98.7 \\
月度 $r$ 中位数 & 0.481 & 0.503 \\
\bottomrule
\end{tabular}
\end{table}

扩展训练窗口后:排序$\rho$从0.900微降至0.893(差异在统计置信区间内,不显著),MAE从61.8改善至52.5(降低15\%),RMSE从116.8降至98.7(降低15\%),月度$r$中位数从0.481提升至0.503。这些结果表明:(1)排序性能在15年数据条件下已接近饱和,更多训练数据主要改善量级而非排序;(2)4年额外数据带来的MAE改善(约10例)虽然在统计上显著,但在公共卫生意义上边际贡献有限;(3)公式的结构和系数在不同训练窗口下保持稳定,表明其捕捉的气候-传播效率关系是稳健的,非偶然的数据拟合结果。

\subsection{讨论}

\textbf{公式空间迁移性的理论意义。}本部分最核心的发现是:从广州单城市数据中通过知识蒸馏发现的传播效率公式,在16城市尺度的空间迁移验证中获得了$\rho=0.900$的排序精度($p=2.05\times10^{-6}$),远优于随机排序的期望值($\rho=0$)。这一结果有力支持了两个重要命题:第一,"气候是广东省登革热空间分布的主要驱动力"——公式仅包含气温、降水和湿度三个气象变量,却能解释16个城市间90\%的风险排序方差;第二,"基于单城市数据的机制发现方法具有跨区域泛化潜力"——神经网络+符号回归的知识蒸馏框架不仅能拟合训练城市的数据,更能发现具有空间普适性的传播效率规律。

\textbf{排序-量级分离现象及其解释。}虽然排序指标表现优异($\rho=0.900$),量级预测仍存在可观的偏差(非GZ MAE$=61.8$例)。这一"排序与量级的分离"现象揭示了登革热传播的双层驱动结构:第一层是气候驱动的\textit{传播潜力}——哪些城市的气候条件更适宜登革热传播(由公式精确刻画,排序高度一致);第二层是非气候因素调制的\textit{实现程度}——实际病例数还取决于人口规模、蚊虫密度基线、人群免疫背景、输入性病例数量和公共卫生干预力度等城市特异性因素。本文的线性标度方案仅通过一个简单系数$\alpha_c$来补偿第二层因素,其简陋性是量级偏差的主要来源。未来可以考虑使用贝叶斯层级模型(Bayesian Hierarchical Model)将这些城市特异性因素作为随机效应纳入,有望进一步缩小量级偏差。

\textbf{与PNAS和Zhang工作的系统比较。}Li等\citep{li2019pnas}在PNAS上的工作虽然在多国尺度验证了温度-传播效率的一致关系,但其方法存在两个根本局限:(1)仅考虑温度单因素,而本文的三变量二次公式在广东省展现了更强的解释力(降水和湿度的独立及交互效应贡献了约40\%的$\bprime$方差);(2)使用非参数B-spline函数,输出的是一条无法用简洁数学表达式描述的曲线,难以进行物理解读和跨场景迁移。Zhang等\citep{zhang2024plos}的方法虽然也采用了"逆问题+符号回归"的框架,但针对的是COVID-19呼吸道传染病,传播机制与登革热存在本质区别:(1)COVID-19的传播不涉及昆虫媒介,模型中不需要蚊虫动力学(BI)组件;(2)温度对COVID-19传播的影响机制(病毒气溶胶存活率、人群室内聚集行为)与对登革热的影响机制(蚊虫发育、EIP、叮咬率)完全不同;(3)降水对COVID-19的影响远小于对登革热的影响。因此,本文发现的二次多项式公式在参数结构和系数含义上与Zhang等的结果有本质差异,体现了蚊媒传染病独特的多因素气候驱动模式。

\textbf{方法学局限性。}本部分的验证仍存在以下局限:(1)BI共享假设可能在粤北和粤东城市引入偏差,因为这些地区的蚊虫越冬模式和季节峰值时间可能与珠三角有所不同;(2)三种标度方案均为事后拟合(使用观测数据估计$\alpha_c$),在真正的前瞻性预测场景中需要独立的蚊密度或人口数据来确定标度参数;(3)本研究仅覆盖广东省16个城市,尚未在气候差异更大的省际尺度(如广东vs云南vs福建)进行验证;(4)城市内部的空间异质性(如城区与郊区的发病率差异可达5--10倍)未被模型捕捉。
