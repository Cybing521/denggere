

\newpage
\addcontentsline{toc}{section}{致谢}
\begin{center}
{\sffamily\zihao{3} 致\quad 谢}
\end{center}
\vspace{12pt}

时光荏苒,四年的大学生活即将画上句号。在毕业论文即将完成之际,回首这段充实而难忘的求学历程,我要向所有在学业和生活中给予我帮助和支持的人表达最诚挚的感谢。

首先,衷心感谢我的指导教师。从大三下学期开始接触课题至今,老师在论文选题方向的确定、研究方案的反复打磨、SEIR模型和神经网络框架的构建调试、符号回归方法的引入和改进、多城市验证策略的设计,以及论文行文结构和学术规范的把关等各个环节,都给予了悉心指导和耐心帮助。老师严谨求实的治学态度、敏锐独到的学术洞察力和开放包容的研究视野,为我树立了终身受益的学术榜样。每当研究遇到瓶颈——无论是Phase 1模型收敛困难、Phase 2符号回归搜索空间爆炸,还是多城市验证中标度方案的选择——老师总能以独到的视角引导我跳出思维定式、找到突破方向。老师不仅传授了科学研究的方法论,更让我领悟了"从数据中发现规律、用模型解释机制"这一定量科学的核心精神。

感谢课题组的各位师兄师姐和同届同学。在数据收集和预处理的繁琐工作中,同学们分工协作、互相核对,保证了数据质量的可靠性;在程序调试的漫长过程中,大家慷慨分享技术经验和代码资源;在每周一次的组会讨论中,各位的提问和建议帮助我不断完善研究思路。特别感谢参与多城市数据整理和敏感性分析的同学们,你们的辛勤付出为论文的验证部分提供了坚实的数据基础。与大家并肩作战的日日夜夜,是我大学生活中最珍贵的记忆之一。

感谢中国疾病预防控制中心和广东省卫生健康委员会发布的传染病监测数据,以及美国国家海洋和大气管理局(NOAA)提供的全球气候日值数据。这些来自不同机构的公开数据资源,构成了本研究的基础素材。没有高质量、长时间序列的公开数据支撑,跨学科的定量研究将无从开展。在此也向所有致力于公共卫生数据开放共享的政策制定者和执行者致以敬意。

感谢开源社区提供的各类优秀工具和框架。PyTorch深度学习框架为神经网络的构建和训练提供了灵活高效的平台,PySR符号回归工具为自动化公式发现提供了强大的搜索引擎,SciPy和NumPy科学计算库为数据处理和统计分析提供了可靠的基础设施。开源精神所代表的知识共享理念,深刻地影响了我的研究方式和学术价值观。

感谢我的父母和家人。是你们二十余年如一日的无条件关爱与默默支持,让我得以心无旁骛地追求学业。每一次深夜的电话、每一顿周末的家常便饭,都是我最温暖的精神港湾。你们的理解与鼓励,是我面对困难时最坚实的后盾。

最后,衷心感谢各位评审专家在百忙之中审阅本文并提出宝贵意见。你们的专业指导将帮助我进一步完善研究内容,提升论文质量。

学无止境,行者无疆。本科阶段的研究经历让我深刻认识到:在传染病建模这一交叉学科领域,还有太多有趣而重要的问题等待探索。我将以本论文为起点,在未来的学术道路上继续追问、不断前行。

\vspace{2cm}
\begin{flushright}
二〇二六年六月
\end{flushright}

\end{document}
