%% ==================================================================
\section{第二部分:多城市机制迁移与验证}
\label{sec:part2}
%% ==================================================================

\subsection{引言}

单城市模型发现的传播效率公式能否推广到其他城市?本章通过广东省16个地级市数据系统验证二次多项式公式的空间泛化能力。空间泛化的核心挑战在于不同城市在人口规模、城市化水平、蚊媒密度基线等方面差异显著,因此重点检验城市间相对风险排名的捕捉能力。

\subsection{数据与方法}

\subsubsection{16城数据概况}

研究涵盖广东省16个地级市:广州、佛山、中山、江门、珠海、深圳、清远、阳江、东莞、肇庆、汕头、湛江、潮州、茂名、揭阳和惠州。数据时间范围2005--2019年,月度分辨率。气象数据来源NOAA GSOD。2014年病例数跨越三个数量级(广州37,382例至惠州37例)。

\subsubsection{三种尺度化方案}

\textbf{方案A(广州尺度化)}:使用广州人口和蚊媒参数,仅替换气候输入,产生"虚拟广州"预测。

\textbf{方案B(非广州线性尺度化)}:引入线性校正因子$\alpha_c = \bar{C}_c / \bar{C}_{\text{GZ}}$,仅用非广州城市拟合。

\textbf{方案C(非广州对数线性尺度化)}:在对数尺度上回归$\log(\hat{C}_c) = \beta_0 + \beta_1 \cdot \log(\text{risk}_c)$。

\subsubsection{评估策略}

"排名优先"策略:首要指标Spearman~$\rho$,辅以MAE、RMSE。2014年为年度验证年,全部15年数据用于月度评估。

\subsection{结果}

\subsubsection{年度排名外推验证}

\begin{table}[H]
\centering
\caption{多城市年度排名验证结果(2014年)}
\label{tab:transfer-annual}
\begin{tabular}{llccccc}
\toprule
子集 & 方案 & $N$ & MAE & RMSE & Spearman $\rho$ & $p$值 \\
\midrule
全部16城 & A(广州尺度化) & 16 & 655.5 & 1499.6 & 0.900 & $2.05\times10^{-6}$ \\
全部16城 & B(非GZ线性) & 16 & 1491.9 & 5737.1 & 0.900 & $2.05\times10^{-6}$ \\
全部16城 & C(非GZ对数线性) & 16 & 1674.3 & 6383.3 & 0.900 & $2.05\times10^{-6}$ \\
\midrule
非广州15城 & A(广州尺度化) & 15 & 699.3 & 1548.8 & 0.879 & $1.63\times10^{-5}$ \\
非广州15城 & B(非GZ线性) & 15 & 61.8 & 116.8 & 0.879 & $1.63\times10^{-5}$ \\
非广州15城 & C(非GZ对数线性) & 15 & 84.0 & 115.2 & 0.879 & $1.63\times10^{-5}$ \\
\bottomrule
\end{tabular}
\end{table}

三种方案排名$\rho=0.900$完全一致(排名仅依赖$\beta'$积分值的相对大小)。方案B非广州15城MAE$=61.8$、RMSE$=116.8$最优。

\begin{figure}[H]
\centering
\includegraphics[width=0.95\textwidth]{../results/data2_1plus3/transfer_2014_bars_data2.png}
\caption{2014年16城年度病例数:观测vs.模型预测}
\label{fig:transfer-bars}
\end{figure}

\subsubsection{城市月度曲线验证}

\begin{table}[H]
\centering
\caption{16城月度预测指标汇总(2005--2019年,每城180个月度观测)}
\label{tab:monthly-summary}
\begin{tabular}{lccccc}
\toprule
 & Pearson $r$ & Spearman $\rho$ & $R^2_{\log}$ & MAE & RMSE \\
\midrule
中位数 & 0.481 & 0.469 & 0.342 & 5.62 & 22.80 \\
均值 & 0.467 & 0.462 & 0.277 & 22.47 & 134.89 \\
最高 & 0.589 (惠州) & 0.716 (广州) & 0.603 (广州) & -- & -- \\
最低 & 0.269 (潮州) & 0.218 (茂名) & $-0.208$ (肇庆) & -- & -- \\
\bottomrule
\end{tabular}
\end{table}

中位$r=0.481$和$\rho=0.469$表明公式对大多数城市能捕捉中等强度季节性趋势。

\begin{figure}[H]
\centering
\includegraphics[width=0.95\textwidth]{../results/data2_1plus3/all_cities_fit_grid.png}
\caption{16城月度预测与观测曲线对比(2005--2019年)}
\label{fig:all-cities-grid}
\end{figure}

\subsubsection{新旧数据集对比}

\begin{table}[H]
\centering
\caption{新旧数据集关键指标比较}
\label{tab:old-vs-new}
\begin{tabular}{lcc}
\toprule
指标 & 旧数据集(13城/2003--2017) & 新数据集(16城/2005--2019) \\
\midrule
非广州排名 Spearman $\rho$ & 0.713 & 0.879 \\
非广州排名 Kendall $\tau$ & 0.545 & 0.771 \\
Phase~1 Pearson $r$ & 0.976 & 0.612 \\
Phase~1 Spearman $\rho$ & 0.634 & 0.705 \\
非广州MAE & 504.7 & 61.8 \\
非广州RMSE & 908.7 & 116.8 \\
\bottomrule
\end{tabular}
\end{table}

排名相关从$\rho=0.713$提升至$\rho=0.879$(+23\%),MAE从504.7降至61.8(-88\%)。新数据集Phase~1 $\rho$从0.634提升至0.705,旧数据集高$r=0.976$来源于2014年极端暴发的主导效应。

\subsubsection{敏感性分析}

2005--2019年与2004--2023年两个时间窗口对比,Phase~1核心指标差异不超过5\%,多城市排名相关变化在可接受范围内。2020年后数据因COVID-19防控可能引入额外混杂。

\subsection{讨论}

\textbf{空间可迁移性}。$\rho=0.900$证实广州发现的气候--$\beta'$关系具有空间泛化能力,理论基础在于登革热病毒生物传播机制在地理上的共性。

\textbf{排名vs.量级}。排名优于绝对误差,因为城市间病例差异不仅来自气候,还受人口密度、蚊媒控制等非气候因素影响。公式最适合用于跨城市风险分层和资源优先分配。

\textbf{与PNAS方法比较}。Li等\cite{li2019pnas}的样条$\beta(T)$不含降水湿度信息且无法以闭合公式迁移。本方法在可迁移性和可解释性方面具有显著优势。

\textbf{局限性}。部分低发病城市月度相关较低(信噪比不足);尺度化回归系数在$n=15$时可能不稳定;蚊媒密度数据仅广州可用。
