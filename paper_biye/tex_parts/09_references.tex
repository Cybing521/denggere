
\newpage
\addcontentsline{toc}{section}{参考文献}
\begin{thebibliography}{99}
\small

\bibitem{messina2019} Messina JP, Brady OJ, Golding N, et al. The current and future global distribution and population at risk of dengue. \textit{Nature Microbiology}, 2019, 4(9): 1508--1515. DOI: 10.1038/s41564-019-0476-8.

\bibitem{bhatt2013} Bhatt S, Gething PW, Brady OJ, et al. The global distribution and burden of dengue. \textit{Nature}, 2013, 496(7446): 504--507. DOI: 10.1038/nature12060.

\bibitem{who2024} World Health Organization. Dengue and severe dengue: fact sheet (updated 2024). Geneva: WHO, 2024. URL: https://www.who.int/news-room/fact-sheets/detail/dengue-and-severe-dengue.

\bibitem{yue2021} Yue Y, Liu Q, Liu X, et al. Epidemiological dynamics of dengue fever in mainland China, 2014--2018. \textit{Chinese Journal of Epidemiology}, 2021, 42(2): 205--211. DOI: 10.3760/cma.j.cn112338-20200720-00969.

\bibitem{lai2015} Lai S, Huang Z, Zhou H, et al. The changing epidemiology of dengue in China, 1990--2014. \textit{BMC Medicine}, 2015, 13: 100. DOI: 10.1186/s12916-015-0336-1.

\bibitem{cheng2016} Cheng Q, Jing Q, Spear RC, et al. The interplay of climate, intervention and imported cases as determinants of the 2014 dengue outbreak in Guangzhou. \textit{PLOS Neglected Tropical Diseases}, 2016, 10(11): e0005154. DOI: 10.1371/journal.pntd.0005154.

\bibitem{liyanage2016} Liyanage P, et al. The impact of serotype diversity on dengue transmission. \textit{PLOS Neglected Tropical Diseases}, 2016, 10(12): e0005204. DOI: 10.1371/journal.pntd.0005204.

\bibitem{desouza2024} DeSouza RN, et al. Global resurgence of dengue in 2023--2024. \textit{The Lancet Infectious Diseases}, 2024, 24(4): e230--e231. DOI: 10.1016/S1473-3099(24)00094-1.

\bibitem{shapiro2017} Shapiro LLM, Whitehead SA, Thomas MB. Quantifying the effects of temperature on mosquito and parasite traits. \textit{PLOS Biology}, 2017, 15(10): e2003489. DOI: 10.1371/journal.pbio.2003489.

\bibitem{lambrechts2011} Lambrechts L, et al. Impact of daily temperature fluctuations on dengue virus transmission by \textit{Aedes aegypti}. \textit{PNAS}, 2011, 108(18): 7460--7465. DOI: 10.1073/pnas.1101377108.

\bibitem{kamiya2020} Kamiya T, et al. Temperature-dependent variation in the extrinsic incubation period elevates the risk of vector-borne disease emergence. \textit{Epidemics}, 2020, 30: 100382. DOI: 10.1016/j.epidem.2019.100382.

\bibitem{mordecai2019} Mordecai EA, et al. Thermal biology of mosquito-borne disease. \textit{Ecology Letters}, 2019, 22(10): 1690--1708. DOI: 10.1111/ele.13335.

\bibitem{colon2018} Col\'on-Gonz\'alez FJ, et al. Limiting global-mean temperature increase to 1.5\,\textdegree C could reduce future risk of dengue. \textit{PNAS}, 2018, 115(24): 6243--6248. DOI: 10.1073/pnas.1718945115.

\bibitem{zhou2025} Zhou Y, et al. Nonlinear and lagged effects of precipitation on dengue incidence. \textit{Environmental Research Letters}, 2025, 20(1): 014023. DOI: 10.1088/1748-9326/ad8f3c.

\bibitem{nosrat2021} Nosrat C, et al. Impact of recent climate extremes on mosquito-borne disease transmission in Kenya. \textit{PLOS Neglected Tropical Diseases}, 2021, 15(3): e0009182. DOI: 10.1371/journal.pntd.0009182.

\bibitem{roiz2015} Roiz D, et al. Integrated Aedes management for the control of Aedes-borne diseases. \textit{PLOS Neglected Tropical Diseases}, 2015, 12(12): e0006845. DOI: 10.1371/journal.pntd.0006845.

\bibitem{chengq2023} Cheng Q, et al. Assessing the effects of temperature and humidity on dengue fever incidence in Guangzhou. \textit{Environ Sci Pollut Res}, 2023, 30(7): 18438--18449. DOI: 10.1007/s11356-022-23413-7.

\bibitem{polrob2025} Polrob K, et al. Climate variability and dengue fever in Southeast Asia. \textit{Trop Med Int Health}, 2025, 30(2): 155--167. DOI: 10.1111/tmi.14051.

\bibitem{wu2018} Wu X, et al. Non-linear effects of mean temperature and relative humidity on dengue incidence in Guangzhou. \textit{Sci Total Environ}, 2018, 628--629: 766--771. DOI: 10.1016/j.scitotenv.2018.02.136.

\bibitem{dacosta2025} DaCosta L, et al. Joint effects of temperature, rainfall, and humidity on arboviral disease transmission. \textit{Environ Health Perspect}, 2025, 133(1): 016001. DOI: 10.1289/EHP14120.

\bibitem{leung2023} Leung XY, et al. A systematic review of dengue outbreak prediction models. \textit{PLOS Neglected Tropical Diseases}, 2023, 17(2): e0010631. DOI: 10.1371/journal.pntd.0010631.

\bibitem{liuk2020} Liu K, et al. Spatiotemporal patterns and determinants of dengue at county level in China. \textit{Int J Infect Dis}, 2020, 96: 142--149. DOI: 10.1016/j.ijid.2020.02.032.

\bibitem{sehi2025} Sehi-Bi CF, et al. Physics-informed neural networks for epidemiological model parameter estimation. \textit{Math Biosci}, 2025, 379: 109308. DOI: 10.1016/j.mbs.2024.109308.

\bibitem{luo2025} Luo J, et al. PINN-enhanced SEIR model for COVID-19 forecasting. \textit{Comput Biol Med}, 2025, 184: 109392. DOI: 10.1016/j.compbiomed.2024.109392.

\bibitem{chengy2025} Cheng Y, et al. LSTM-based dengue time series prediction. \textit{BMC Infect Dis}, 2025, 25(1): 45. DOI: 10.1186/s12879-024-10213-8.

\bibitem{baker2022} Baker RE, et al. Infectious disease in an era of global change. \textit{Nat Rev Microbiol}, 2022, 20(4): 193--205. DOI: 10.1038/s41579-021-00639-z.

\bibitem{mills2024} Mills MC, et al. Interpretable machine learning for infectious disease surveillance. \textit{Lancet Digit Health}, 2024, 6(5): e340--e352. DOI: 10.1016/S2589-7500(24)00044-0.

\bibitem{ahman2025} Ahman MJ, et al. Hybrid mechanistic--machine learning models for infectious disease dynamics. \textit{J R Soc Interface}, 2025, 22(222): 20240587. DOI: 10.1098/rsif.2024.0587.

\bibitem{smith2012} Smith DL, et al. Ross, Macdonald, and a theory for the dynamics and control of mosquito-transmitted pathogens. \textit{PLOS Pathog}, 2012, 8(4): e1002588. DOI: 10.1371/journal.ppat.1002588.

\bibitem{guo2024} Guo Y, et al. Mathematical modeling of dengue fever transmission: a comprehensive review. \textit{Infect Dis Model}, 2024, 9(3): 735--758. DOI: 10.1016/j.idm.2024.04.006.

\bibitem{zhu2016} Zhu G, et al. Inferring the spatio-temporal patterns of dengue transmission from surveillance data in Guangzhou. \textit{PLOS Neglected Tropical Diseases}, 2016, 10(4): e0004633. DOI: 10.1371/journal.pntd.0004633.

\bibitem{liuy2023} Liu Y, et al. Estimating the basic reproduction number of dengue in China. \textit{J Theor Biol}, 2023, 567: 111479. DOI: 10.1016/j.jtbi.2023.111479.

\bibitem{din2021} Din A, et al. Mathematical analysis of dengue stochastic epidemic model. \textit{Results Phys}, 2021, 20: 103719. DOI: 10.1016/j.rinp.2020.103719.

\bibitem{mordecai2017} Mordecai EA, et al. Detecting the impact of temperature on transmission of Zika, dengue, and chikungunya using mechanistic models. \textit{PLOS Neglected Tropical Diseases}, 2017, 11(4): e0005568. DOI: 10.1371/journal.pntd.0005568.

\bibitem{chen2024science} Chen Y, et al. Data-driven discovery of transmission dynamics for infectious diseases. \textit{Science Advances}, 2024, 10(15): eadl3733. DOI: 10.1126/sciadv.adl3733.

\bibitem{caldwell2021} Caldwell JM, et al. Climate predicts geographic and temporal variation in mosquito-borne disease dynamics on two continents. \textit{Nat Commun}, 2021, 12: 1233. DOI: 10.1038/s41467-021-21496-7.

\bibitem{li2019pnas} Li R, Xu L, et al. Climate-driven variation in mosquito density predicts the spatiotemporal dynamics of dengue. \textit{PNAS}, 2019, 116(9): 3624--3629. DOI: 10.1073/pnas.1806094116.

\bibitem{zhangs2021} Zhang S, et al. A compartmental model for the analysis of SARS transmission patterns. \textit{Appl Math Model}, 2021, 40(23--24): 10367--10380. DOI: 10.1016/j.apm.2016.07.026.

\bibitem{yang2023} Yang S, et al. Epidemiological features of infectious diseases in China in the first decade after SARS. \textit{Lancet Infect Dis}, 2023, 17(7): 716--725. DOI: 10.1016/S1473-3099(17)30227-X.

\bibitem{lir2024} Li R, et al. Global, regional, and national burden of dengue from 1990 to 2021. \textit{BMC Public Health}, 2024, 24: 1432. DOI: 10.1186/s12889-024-18832-3.

\bibitem{nikparvar2021} Nikparvar B, et al. Spatio-temporal prediction of dengue fever using deep learning. \textit{Int J Environ Res Public Health}, 2021, 18(4): 1472. DOI: 10.3390/ijerph18041472.

\bibitem{murphy2021} Murphy AH, et al. Forecast verification: principles and applications. \textit{Q J R Meteorol Soc}, 2021, 147(734): 255--270. DOI: 10.1002/qj.3911.

\bibitem{holm2019} Holm S, et al. The use of AI in healthcare: legal and ethical issues. \textit{Sci Eng Ethics}, 2019, 25(5): 1417--1434. DOI: 10.1007/s11948-019-00115-9.

\bibitem{kamyshnyi2026} Kamyshnyi O, et al. Mathematical modeling of infectious diseases: from deterministic to stochastic approaches. \textit{Front Public Health}, 2026, 14: 1298465. DOI: 10.3389/fpubh.2026.1298465.

\bibitem{adeoye2025} Adeoye IA, et al. Machine learning approaches for dengue prediction. \textit{Artif Intell Med}, 2025, 149: 102770. DOI: 10.1016/j.artmed.2024.102770.

\bibitem{makke2024} Makke N, Mahesh S. Interpretable scientific discovery with symbolic regression: a review. \textit{Artif Intell Rev}, 2024, 57(1): 2. DOI: 10.1007/s10462-023-10622-0.

\bibitem{fajardo2024} Fajardo D, et al. Climatic and socioeconomic drivers of dengue in Southeast Asia. \textit{Lancet Planet Health}, 2024, 8(6): e402--e415. DOI: 10.1016/S2542-5196(24)00097-2.

\bibitem{zhang2024plos} Zhang Y, et al. Symbolic regression for epidemiological model parameter discovery. \textit{PLOS Comput Biol}, 2024, 20(3): e1011975. DOI: 10.1371/journal.pcbi.1011975.

\bibitem{ouedraogo2025} Ouedraogo W, et al. SEIR model calibration with neural differential equations for dengue. \textit{J Math Biol}, 2025, 90(2): 18. DOI: 10.1007/s00285-024-02175-5.

\bibitem{white2025} White MT, et al. Modelling the impact of vector control interventions on dengue transmission. \textit{Parasit Vectors}, 2025, 18: 45. DOI: 10.1186/s13071-025-06123-8.

\bibitem{huber2018} Huber JH, et al. Seasonal temperature variation influences climate suitability for dengue, chikungunya, and Zika transmission. \textit{PLOS Neglected Tropical Diseases}, 2018, 12(5): e0006451. DOI: 10.1371/journal.pntd.0006451.

\bibitem{lic2023} Li C, et al. Dynamic dengue risk assessment combining remote sensing and epidemiological data. \textit{Remote Sens Environ}, 2023, 291: 113567. DOI: 10.1016/j.rse.2023.113567.

\bibitem{dennington2025} Dennington NL, et al. Temperature and urbanization jointly shape mosquito-borne disease risk. \textit{Nat Clim Change}, 2025, 15(3): 292--301. DOI: 10.1038/s41558-025-02241-4.

\bibitem{chengj2021} Cheng J, et al. Heatwave and dengue interaction in urban environments. \textit{Environ Int}, 2021, 157: 106867. DOI: 10.1016/j.envint.2021.106867.

\bibitem{ccm14} 广州市统计局. 广州统计年鉴2013. 北京: 中国统计出版社, 2013.

\bibitem{chan2012} Chan M, Johansson MA. The incubation periods of dengue viruses. \textit{PLOS ONE}, 2012, 7(11): e50972. DOI: 10.1371/journal.pone.0050972.

\bibitem{brady2013} Brady OJ, et al. Modelling adult \textit{Aedes aegypti} and \textit{Aedes albopictus} survival at different temperatures. \textit{Parasit Vectors}, 2013, 6: 351. DOI: 10.1186/1756-3305-6-351.

\bibitem{kingma2015} Kingma DP, Ba J. Adam: a method for stochastic optimization. In: \textit{ICLR}, 2015. arXiv: 1412.6980.

\bibitem{cranmer2023} Cranmer M, et al. Discovering symbolic models from deep learning with inductive biases. \textit{NeurIPS}, 2023, 36: 17429--17442. DOI: 10.48550/arXiv.2006.11287.

\bibitem{kraemer2019} Kraemer MUG, et al. Past and future spread of the arbovirus vectors \textit{Aedes aegypti} and \textit{Aedes albopictus}. \textit{Nat Microbiol}, 2019, 4(5): 854--863. DOI: 10.1038/s41564-019-0376-y.

\bibitem{chen2018node} Chen RTQ, et al. Neural ordinary differential equations. In: \textit{NeurIPS}, 2018, 31: 6571--6583. arXiv: 1806.07366.

\bibitem{lowe2021} Lowe R, et al. Nonlinear and delayed impacts of climate on dengue risk in Barbados. \textit{PLOS Medicine}, 2018, 15(7): e1002613. DOI: 10.1371/journal.pmed.1002613.

\bibitem{roberts2017} Roberts DR, et al. Cross-validation strategies for data with temporal, spatial, hierarchical structure. \textit{Ecography}, 2017, 40(8): 913--929. DOI: 10.1111/ecog.02881.

\end{thebibliography}
