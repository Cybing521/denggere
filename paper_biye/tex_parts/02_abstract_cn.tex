
\newpage
\begin{center}
{\sffamily\zihao{3} 摘\quad 要}
\end{center}
\addcontentsline{toc}{section}{摘要}

登革热是全球最严重的蚊媒传染病之一,中国南方尤其广东省是国内最主要的流行区域。理解气候因素如何驱动登革热传播效率,对于建立早期预警系统和制定精准防控策略具有重要意义。然而,传统机制模型往往依赖先验假设固定传播率函数形式,难以从数据中自动发现最优的气候--传播率关系;纯数据驱动的机器学习方法则存在可解释性不足的瓶颈。

本文提出一种"神经网络耦合SEIR动力学+符号回归"三阶段混合建模框架,旨在兼顾机制可解释性与数据适应性。该框架包含三个核心阶段:(1) 基于SEIR仓室模型反演月尺度传播系数$\beta'$时间序列;(2) 以多层感知机(MLP)学习气候变量(温度$T$、降水$R$、相对湿度$H$)到$\beta'$的非线性映射,实现逐月病例数预测;(3) 利用符号回归对神经网络进行知识蒸馏,发现可解释的闭合公式。

研究分为两个部分。第一部分以广州市为单城市案例(2005--2019年双周数据)。Phase~1阶段,耦合模型对广州月度病例的预测达到Pearson相关系数$r=0.612$、Spearman秩相关$\rho=0.705$、对数$R^2=0.450$、MAE$=51.23$。Phase~2阶段,符号回归从物理模板族和多项式族两类候选中发现最优公式,其中二次多项式含交互项的公式以$R^2=0.999987$的精度拟合神经网络输出,揭示了温度--降水正交互($a_{TR}>0$)和降水平方负效应($a_{RR}<0$)等物理规律。对2014年特大暴发的分析表明,该公式能区分极端年份的气候异常信号。

第二部分将广州发现的公式迁移至广东省16个地级市(2005--2019年),采用三种城市尺度化方案进行年度和月度验证。结果表明,16城年度总病例排名的Spearman相关$\rho=0.900$($p=2.05\times10^{-6}$),非广州15城MAE$=61.8$、RMSE$=116.8$,证实了公式的空间泛化能力。与旧数据集(13城/2003--2017年)相比,新数据集使排名相关从$\rho=0.713$提升至$\rho=0.879$。城市月度曲线的中位Pearson~$r=0.481$、中位Spearman~$\rho=0.469$,表明公式可捕捉多数城市的季节性趋势。

本研究的主要创新包括:(1) 提出"NN逆问题+符号蒸馏"范式,克服了传统SEIR模型依赖先验函数形式的局限;(2) 发现的闭合公式具有明确物理含义且可直接迁移至其他城市;(3) 系统验证了单城市机制在多城市空间尺度上的可迁移性;(4) 相比Li等(2019 PNAS)的样条方法和Zhang等(2024)的纯符号回归方法,本框架在可解释性和泛化性之间取得了更好的平衡。

\vspace{1em}
\noindent {\sffamily 关键词:}登革热;SEIR模型;神经网络;符号回归;传播效率;广东省;多城市验证
