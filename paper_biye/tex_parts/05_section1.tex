%% ==================================================================
\section{前言}
\label{sec:intro}
%% ==================================================================

\subsection{登革热全球与中国流行概况}
\label{sec:intro-epi}

登革热(Dengue Fever)是由登革病毒(DENV,分为DENV-1至DENV-4四个血清型)引起的急性蚊媒传染病,主要通过埃及伊蚊(\textit{Aedes aegypti})和白纹伊蚊(\textit{Aedes albopictus})叮咬传播。Bhatt等\cite{bhatt2013}估计全球每年约3.9亿登革病毒感染,其中约9600万表现出临床症状。世界卫生组织\cite{who2024}指出,过去二十年间登革热报告病例数增长了八倍以上,从2000年的50万例升至2023年的超过600万例,已成为热带和亚热带地区最严峻的公共卫生挑战之一。Messina等\cite{messina2019}利用全球尺度的统计模型预测,到2080年气候变化和城市化将使全球约63亿人面临登革热风险,较2015年增加约22亿人。

中国大陆的登革热疫情以境外输入和本地暴发交替为特征,流行区域主要集中在广东、云南、浙江等南方省份\cite{lai2015}。其中广东省是中国大陆登革热负担最重的省份,历史上多次出现大规模暴发:2014年广东省报告超过45,000例,广州市单城报告逾37,000例,创下历史纪录\cite{cheng2016}。Yue等\cite{yue2021}的系统综述表明,自2004年以来广东省贡献了全国超过70\%的登革热报告病例,年度病例数呈波动性上升趋势,暴发间隔呈缩短趋势。这一流行模式与该地区亚热带季风气候、高度城市化、人口密集以及频繁的国际人员流动密切相关。

从血清型分布来看,广东省历年暴发中DENV-1最为常见,但也检测到DENV-2、DENV-3和DENV-4的输入性和本地传播病例\cite{cheng2016}。值得注意的是,由于不同血清型之间仅存在短暂的交叉免疫保护,二次感染可能导致更严重的登革出血热(DHF)和登革休克综合征(DSS),给公共卫生系统带来额外压力\cite{liyanage2016}。

近年来,全球气候变暖、极端天气事件增加以及"一带一路"倡议下国际贸易与旅游的扩大,进一步加剧了登革热北扩和暴发频次增加的风险。DeSouza等\cite{desouza2024}指出,2023--2024年全球登革热病例再创历史新高,部分与厄尔尼诺现象引发的异常高温和强降水有关。在中国语境下,气候变化、城市扩张和蚊媒分布北移的叠加效应使得登革热防控面临前所未有的挑战,亟需建立基于机制理解的精准预测与早期预警体系。

\subsection{气候因素与蚊媒传播}
\label{sec:intro-climate}

登革热的传播强度受到多种气候因素的共同影响,其中温度、降水和湿度是最关键的三个变量。这些气候因素通过影响蚊媒的生存、繁殖、发育速率以及病毒在蚊体内的外潜伏期(Extrinsic Incubation Period, EIP),从而间接调控病毒在人群中的传播效率。

\textbf{温度}是影响登革热传播的核心气候因素。Mordecai等\cite{mordecai2019}的全面实验研究表明,蚊媒传播能力(以基本再生数$R_0$的组分度量)对温度呈单峰响应,最优传播温度约为29$^\circ$C。在此温度下,蚊虫叮咬率最高、病毒外潜伏期最短、蚊虫存活率最大,三者的乘积效应使传播效率达到峰值。当温度低于约18$^\circ$C或高于约34$^\circ$C时,传播能力显著下降\cite{mordecai2017}。Shapiro等\cite{shapiro2017}和Lambrechts等\cite{lambrechts2011}进一步指出,温度日较差(Diurnal Temperature Range, DTR)对传播效率也有重要影响:在接近最适温度时,较大的DTR会降低传播效率;而在偏低温度条件下,适度的DTR反而可能增强传播。Kamiya等\cite{kamiya2020}的荟萃分析确认了温度对蚊媒传染病传播的非线性调控作用在全球不同地理区域具有一致性。

\textbf{降水}对登革热的影响具有双重性。一方面,降水为蚊虫提供了必要的繁殖栖息地——积水容器、洼地和废弃物中的积水是埃及伊蚊和白纹伊蚊的主要产卵场所\cite{roiz2015};适量降水显著增加蚊虫密度,从而提高传播风险\cite{nosrat2021}。另一方面,极端强降水可能冲刷幼虫栖息地、降低蚊虫存活率,产生抑制效应\cite{colon2018}。Zhou等\cite{zhou2025}的纵向研究发现,降水与登革热发病率之间存在显著的非线性关系和时间滞后效应,累积降水量超过一定阈值后传播风险不再持续增加,呈现饱和或下降趋势。这种"先增后平"的模式在本文模型发现中也得到了印证。

\textbf{相对湿度}影响蚊虫的存活和活动能力。Wu等\cite{wu2018}对中国南方登革热暴发的时间序列分析发现,相对湿度存在一个约76\%的阈值效应——当湿度超过此值时,蚊虫存活率和叮咬活跃度显著提高,登革热传播风险明显增大。Cheng等\cite{chengq2023}对广州的研究进一步证实了湿度与登革热发病率之间的正相关关系,尤其在高温环境下湿度的促进作用更为显著。Polrob等\cite{polrob2025}在东南亚的研究中发现,湿度与蚊虫叮咬率之间存在协同关系,进一步支持了湿度作为重要传播调节因子的地位。

从生态机制的角度,上述三个气候变量并非独立作用,而是通过复杂的交互效应共同决定传播强度。例如,高温高湿条件下蚊虫的吸血频率和存活率同时增加,产生协同促进效应;而高温干燥条件则可能因蚊虫脱水死亡而抑制传播。DaCosta等\cite{dacosta2025}和Leung等\cite{leung2023}的研究均强调,单独考虑任一气候因子都不足以准确描述传播动态,需要同时纳入温度、降水和湿度的联合效应。这一认识构成了本文将三个气候变量同时纳入神经网络模型的理论基础。

\subsection{模型研究现状}
\label{sec:intro-models}

登革热建模研究经历了从纯统计模型到机制模型、再到人工智能融合模型的发展历程,不同方法在解释能力、预测精度和可推广性方面各有优劣。

\subsubsection{统计模型}

广义加性模型(Generalized Additive Models, GAM)和分布式滞后非线性模型(Distributed Lag Non-linear Models, DLNM)是登革热气候--疫情关系研究中应用最广泛的统计工具\cite{lowe2021}。GAM能够灵活地刻画气候变量与发病率之间的非线性关系,同时控制季节性和长期趋势等混杂因素。DLNM进一步考虑了气候影响的时间滞后结构,能够同时估计暴露--反应关系和滞后效应\cite{roberts2017}。Liu等\cite{liuk2020}利用DLNM分析了中国南方多个城市的气候--登革热关系,发现温度和降水的影响在滞后1--3个月最为显著。

然而,统计模型本质上是"关联性"而非"因果性"工具,其参数不具有直接的流行病学机制含义。此外,统计模型通常以拟合历史数据为主要目标,在外推到未见过的气候条件或新的地理区域时,预测能力往往大幅下降。

\subsubsection{机制模型}

基于仓室结构的传染病动力学模型是理解传播机制的经典工具。Ross-Macdonald模型及其扩展形式将人--蚊传播过程分解为若干关键参数(叮咬率、外潜伏期、蚊虫死亡率等),每个参数都具有明确的生物学含义\cite{smith2012}。SEI-SEIR(蚊-人)耦合模型是登革热研究中常用的仓室结构,将蚊群的"易感--暴露--感染"(SEI)与人群的"易感--暴露--感染--恢复"(SEIR)动态耦合\cite{guo2024}。

Li等\cite{li2019pnas}在2019年《PNAS》发表的研究中提出了一种创新的方法:在SEI-SEIR框架中使用时变三次样条函数拟合传播系数$\beta(t)$与温度的关系,并通过广州2005--2015年的病例和气候数据进行参数估计。该研究发现$\beta(t)$对温度呈单峰响应,最优温度约为27--29$^\circ$C,与实验室研究结果一致。然而,该方法存在以下局限:(1)~三次样条的形式需要预先指定节点数和位置,灵活性受限;(2)~仅考虑温度单一气候变量,忽略了降水和湿度的影响;(3)~最终结果为分段平滑曲线而非可移植的闭合公式,难以直接迁移至其他城市。

\subsubsection{人工智能融合方法}

近年来,将深度学习与微分方程模型结合的"物理信息神经网络"(Physics-Informed Neural Networks, PINN)和"神经常微分方程"(Neural ODE)方法受到越来越多的关注\cite{chen2018node}。在传染病建模领域,Sehi等\cite{sehi2025}和Luo等\cite{luo2025}将PINN应用于SIR/SEIR模型的参数估计和短期预测,取得了优于传统拟合方法的精度。Cheng等\cite{chengy2025}提出了基于LSTM的登革热时间序列预测模型,在短期预测任务中表现出色。

然而,纯神经网络方法的"黑箱"本质使其难以提供机制层面的洞见。即使模型预测准确,研究者仍然无法回答"气候如何影响传播率"这一核心科学问题。Baker等\cite{baker2022}和Mills等\cite{mills2024}均指出,在传染病动力学领域,可解释性和可迁移性通常比单纯的预测精度更有实际价值。Ahman等\cite{ahman2025}的综述进一步强调了"混合机制--数据驱动"框架在传染病建模中的前景。

\subsubsection{符号回归方法}

符号回归(Symbolic Regression, SR)是一种从数据中直接搜索数学表达式的方法,能够在不预设函数形式的前提下发现数据中的数学规律\cite{cranmer2023}。与传统回归方法不同,符号回归的搜索空间包含所有可能的数学表达式(由指定的运算符和变量组合而成),其目标是在精度和复杂度之间取得帕累托最优。

Zhang等\cite{zhang2024plos}在2024年《PLOS Computational Biology》发表了将符号回归应用于传染病模型参数发现的开创性工作。该研究直接在SIR/SEIR模型框架内使用符号回归搜索传播率$\beta$的函数表达式,从模拟数据中成功恢复了已知的传播率函数。然而,该方法面临以下挑战:(1)~直接在高维表达式空间中搜索计算成本极高;(2)~缺乏利用先验物理知识引导搜索的机制;(3)~尚未在真实流行病数据上得到充分验证。

Makke和Mahesh\cite{makke2024}在符号回归综述中指出,结合神经网络预训练和符号蒸馏的两阶段策略是一种有前景的方向:先用神经网络捕获复杂映射关系,再用符号回归提取简洁公式。这种"知识蒸馏"思路正是本文方法论的核心灵感来源。

\subsection{研究目标与创新点}
\label{sec:intro-innovation}

基于上述文献回顾,本文提出以下研究目标:

\begin{enumerate}[leftmargin=2em]
\item 构建"SEIR动力学+神经网络+符号回归"三阶段混合建模框架,从时间序列数据中自动发现气候变量到登革热传播系数$\beta'$的最优函数关系。
\item 以广州市为核心案例,利用2005--2019年双周尺度病例和气候数据,训练耦合模型并提取可解释闭合公式。
\item 将发现的公式迁移至广东省16个地级市,在空间维度上验证其泛化能力和可迁移性。
\item 与现有方法(尤其是Li等2019年PNAS的样条方法和Zhang等2024年的纯符号回归方法)进行比较,论证本框架在可解释性--泛化性平衡方面的优势。
\end{enumerate}

与现有工作相比,本文的创新点包括:

\begin{enumerate}[leftmargin=2em]
\item \textbf{方法论创新——"NN逆问题+符号蒸馏"范式}:不同于Li等\cite{li2019pnas}预设样条函数形式,本文通过神经网络自由学习$\beta'$的气候映射关系,再用符号回归提取公式,实现了"数据驱动的函数发现"。
\item \textbf{多变量联合建模}:不同于仅考虑温度单一变量的传统做法,本文同时纳入温度、降水和相对湿度三个气候变量及其交互效应,更全面地刻画了气候对传播效率的调控机制。
\item \textbf{可迁移的闭合公式}:符号回归发现的二次多项式公式具有明确的系数含义,可直接通过城市尺度参数进行迁移,无需在每个城市重新训练模型。
\item \textbf{系统性的空间验证}:通过16城年度排名和月度曲线的双重验证,首次在中国南方多城市尺度上系统评估了单城市发现的传播效率公式的空间泛化性能。
\end{enumerate}

\subsection{全文结构}
\label{sec:intro-structure}

本文其余部分组织如下:

\textbf{第二章(第一部分:单城市机制发现)}以广州市为案例,详细阐述数据来源、SEIR模型构建、神经网络耦合训练策略、符号回归方法以及评估指标体系,并呈现Phase~1(耦合模型预测)和Phase~2(公式发现)的结果与讨论。

\textbf{第三章(第二部分:多城市机制迁移与验证)}将广州发现的公式迁移至广东省16个地级市,介绍三种城市尺度化方案,呈现年度和月度验证结果,并与旧数据集进行对比分析。

\textbf{第四章(总结与展望)}总结主要发现和创新点,讨论研究局限性,提出未来改进方向。
