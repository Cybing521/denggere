%% ==================================================================
\section{前言}
\label{sec:intro}
%% ==================================================================

\subsection{登革热全球与中国流行概况}
\label{sec:intro-epi}

登革热(Dengue Fever)是由登革病毒(DENV,分为DENV-1至DENV-4四个血清型)引起的急性蚊媒传染病,主要通过埃及伊蚊(\textit{Aedes aegypti})和白纹伊蚊(\textit{Aedes albopictus})叮咬传播。Bhatt等\cite{bhatt2013}估计全球每年约3.9亿登革病毒感染,其中约9600万表现出临床症状。世界卫生组织\cite{who2024}指出,过去二十年间登革热报告病例数增长了八倍以上,从2000年的50万例升至2023年的超过600万例,已成为热带和亚热带地区最严峻的公共卫生挑战之一。Messina等\cite{messina2019}利用全球尺度的统计模型预测,到2080年气候变化和城市化将使全球约63亿人面临登革热风险,较2015年增加约22亿人。

中国大陆的登革热疫情以境外输入和本地暴发交替为特征,流行区域主要集中在广东、云南、浙江等南方省份\cite{lai2015}。其中广东省是中国大陆登革热负担最重的省份,历史上多次出现大规模暴发:2014年广东省报告超过45,000例,广州市单城报告逾37,000例,创下历史纪录\cite{cheng2016}。Yue等\cite{yue2021}的系统综述表明,自2004年以来广东省贡献了全国超过70\%的登革热报告病例,年度病例数呈波动性上升趋势,暴发间隔呈缩短趋势。这一流行模式与该地区亚热带季风气候、高度城市化、人口密集以及频繁的国际人员流动密切相关。

从血清型分布来看,广东省历年暴发中DENV-1最为常见,但也检测到DENV-2、DENV-3和DENV-4的输入性和本地传播病例\cite{cheng2016}。值得注意的是,由于不同血清型之间仅存在短暂的交叉免疫保护,二次感染可能导致更严重的登革出血热(DHF)和登革休克综合征(DSS),给公共卫生系统带来额外压力\cite{liyanage2016}。

近年来,全球气候变暖、极端天气事件增加以及"一带一路"倡议下国际贸易与旅游的扩大,进一步加剧了登革热北扩和暴发频次增加的风险。DeSouza等\cite{desouza2024}指出,2023--2024年全球登革热病例再创历史新高,部分与厄尔尼诺现象引发的异常高温和强降水有关。在中国语境下,气候变化、城市扩张和蚊媒分布北移的叠加效应使得登革热防控面临前所未有的挑战,亟需建立基于机制理解的精准预测与早期预警体系。

\subsection{气候因素与蚊媒传播}
\label{sec:intro-climate}

登革热的传播强度受到多种气候因素的共同影响,其中温度、降水和湿度是最关键的三个变量。这些气候因素通过影响蚊媒的生存、繁殖、发育速率以及病毒在蚊体内的外潜伏期(Extrinsic Incubation Period, EIP),从而间接调控病毒在人群中的传播效率。

\textbf{温度}是影响登革热传播的核心气候因素。Mordecai等\cite{mordecai2019}的全面实验研究表明,蚊媒传播能力(以基本再生数$R_0$的组分度量)对温度呈单峰响应,最优传播温度约为29$^\circ$C。在此温度下,蚊虫叮咬率最高、病毒外潜伏期最短、蚊虫存活率最大,三者的乘积效应使传播效率达到峰值。当温度低于约18$^\circ$C或高于约34$^\circ$C时,传播能力显著下降\cite{mordecai2017}。Shapiro等\cite{shapiro2017}和Lambrechts等\cite{lambrechts2011}进一步指出,温度日较差(Diurnal Temperature Range, DTR)对传播效率也有重要影响:在接近最适温度时,较大的DTR会降低传播效率;而在偏低温度条件下,适度的DTR反而可能增强传播。Kamiya等\cite{kamiya2020}的荟萃分析确认了温度对蚊媒传染病传播的非线性调控作用在全球不同地理区域具有一致性。

\textbf{降水}对登革热的影响具有双重性。一方面,适量降水为伊蚊提供了繁殖所需的积水容器(如废弃轮胎、花盆托盘、建筑工地积水等),增加幼虫孳生地数量,从而提高成蚊密度\cite{li2019pnas}。另一方面,极端强降水可能冲刷幼虫孳生地,导致蚊虫密度短暂下降\cite{desouza2024}。Xu等\cite{xu2020}对广州的研究发现,降水对登革热的影响存在约1--2个月的滞后效应,反映了从降水到积水形成、幼虫发育、成蚊羽化再到病毒传播的完整生态链条。此外,降水的时间分布模式(如连续小雨vs.间歇性暴雨)对蚊虫孳生的影响也存在差异,但目前的月度分辨率数据难以捕捉这种细粒度效应。

\textbf{相对湿度}主要通过影响成蚊存活率和飞行活动能力来调控传播效率。高湿度环境有利于蚊虫存活和活动,而低湿度条件下蚊虫脱水死亡率增加\cite{xu2020}。Xu等\cite{xu2020}的研究表明,广州地区相对湿度与登革热发病率呈正相关,但其独立效应弱于温度。值得注意的是,温度、降水和湿度之间存在较强的共线性(如高温季节通常伴随高湿度和多降水),这给单独量化各因素的独立贡献带来了统计学挑战。

除上述三个核心气候变量外,风速、日照时数、蒸发量等因素也可能影响蚊媒活动和病毒传播,但其效应相对较弱且研究证据不够充分。本研究聚焦于温度、降水和相对湿度三个变量,这与大多数登革热气候建模研究的变量选择一致。

值得强调的是,气候因素对登革热传播的影响并非简单的线性关系,而是通过复杂的非线性机制发挥作用。首先,温度对蚊虫生命周期各阶段的影响存在不同的最适温度和阈值温度,这些效应的叠加产生了传播效率对温度的非对称单峰响应。其次,气候因素之间存在交互效应——例如,高温高湿条件下蚊虫活动最为活跃,而高温低湿条件下蚊虫存活率显著下降。第三,气候因素对传播效率的影响存在时间滞后,从气候条件变化到蚊虫种群响应再到人群感染,通常需要数周到数月的时间。这些非线性特征使得传统的线性统计模型难以准确捕捉气候--传播率关系,需要引入具有非线性建模能力的方法。

此外,城市热岛效应对登革热传播也有重要影响。城市中心区域的温度通常比郊区高2--5$^\circ$C,这种温度差异可能导致城市中心的传播季节更长、传播效率更高。广州作为超大城市,其热岛效应尤为显著,这也是广州成为广东省登革热负担最重城市的原因之一。然而,由于本研究使用的是气象站点数据而非空间分辨率更高的遥感温度数据,城市内部的温度空间异质性未能被充分捕捉,这是未来研究可以改进的方向。

\subsection{蚊媒密度监测与布雷图指数}
\label{sec:intro-bi}

蚊媒密度是连接气候因素与登革热传播的关键中间变量。在实际监测中,布雷图指数(Breteau Index, BI)是最广泛使用的伊蚊幼虫密度指标,定义为每百户调查中发现的阳性容器数。世界卫生组织将BI$\geq$20作为登革热暴发风险的警戒阈值。

然而,BI数据在实际应用中面临诸多挑战。首先,BI监测需要大量人力进行入户调查,覆盖范围和时间连续性受限。在中国,仅部分城市建立了系统性的BI监测网络,且监测频率和方法标准在不同城市间存在差异。其次,BI反映的是幼虫密度而非成蚊密度,两者之间的转化受到温度、容器类型、天敌等多种因素的影响。第三,不同城市的BI基线水平差异显著——本研究中8个有BI数据的城市,BI均值从深圳的2.2到揭阳的13.1不等,最大/最小比达5.9倍,这种差异反映了城市化水平、居住环境、防控力度等非气候因素的影响。

鉴于BI数据的稀缺性和异质性,本研究采用两种互补策略:(1)~利用有BI数据的8个城市训练蚊媒密度的气候驱动公式,通过城市内归一化消除基线差异;(2)~在传播率建模中采用具有昆虫学物理先验的Bri\`{e}re函数,减少对BI数据的直接依赖。

\subsection{传染病动力学模型}
\label{sec:intro-seir}

仓室模型(Compartmental Models)是传染病动力学建模的经典框架。其核心思想是将人群划分为若干互斥的流行病学状态(仓室),并用微分方程描述个体在仓室间的转移速率。对于登革热,最常用的是SEIR模型,将人群分为易感者(Susceptible, S)、潜伏者(Exposed, E)、感染者(Infectious, I)和恢复者(Recovered, R)四个仓室。

经典SEIR模型的连续时间形式为:
\begin{equation}
\label{eq:seir-ode}
\frac{dS}{dt} = -\beta SI/N, \quad
\frac{dE}{dt} = \beta SI/N - \sigma E, \quad
\frac{dI}{dt} = \sigma E - \gamma I, \quad
\frac{dR}{dt} = \gamma I
\end{equation}
其中$\beta$为传播率,$\sigma$为潜伏期倒数($1/\sigma$为平均潜伏期),$\gamma$为恢复率($1/\gamma$为平均感染期),$N$为总人口。

在登革热建模中,传播率$\beta$不是常数,而是受气候因素调控的时变参数。如何建模$\beta(t)$与气候变量的关系,是本研究的核心科学问题。传统方法通常假设$\beta$为温度的某种参数化函数(如高斯函数、多项式等),但这种先验假设可能与真实的生物学机制不符。本研究采用Bri\`{e}re函数作为$\beta(T)$的参数化形式,该函数源于昆虫发育速率的温度依赖性理论,具有明确的生物学基础。

对于月度分辨率的数据,本研究采用离散时间SEIR框架,避免了连续时间ODE在大时间步长下的数值不稳定性。具体而言,每月的新增病例预测为:
\begin{equation}
\hat{C}_t = \beta'(T_t) \cdot \hat{M}_t \cdot S_t + \eta
\end{equation}
其中$\beta'(T_t)$为Bri\`{e}re温度响应函数,$\hat{M}_t$为归一化蚊媒密度,$S_t$为易感人群池,$\eta$为背景输入率。

在参数估计方面,传统方法通常采用最大似然估计(MLE)或马尔可夫链蒙特卡洛(MCMC)方法。然而,SEIR模型的似然函数通常不具有解析形式,且参数空间可能存在多个局部最优解。本研究采用差分进化(Differential Evolution, DE)算法进行全局优化,该算法是一种基于种群的随机搜索方法,通过变异、交叉和选择操作在参数空间中进行高效搜索,能够有效避免陷入局部最优。DE算法的优势在于:(1)~不需要目标函数的梯度信息,适用于非光滑、非凸的优化问题;(2)~通过种群多样性维持全局搜索能力;(3)~参数设置简单,仅需指定种群大小、变异因子和交叉概率。

目标函数的设计对参数估计的质量至关重要。本研究采用Huber损失与Spearman相关性惩罚的加权组合作为目标函数。Huber损失在残差较小时等价于均方误差,在残差较大时等价于绝对误差,对异常值具有鲁棒性——这对于登革热数据尤为重要,因为2014年特大暴发的病例数远超其他年份,纯均方误差会导致优化过度偏向拟合2014年。Spearman相关性惩罚则确保模型不仅在量级上接近观测值,还能正确捕捉病例的时序排名模式。

\subsection{传播率的温度依赖性:Bri\`{e}re函数}
\label{sec:intro-briere}

昆虫的发育速率、叮咬频率和存活率均对温度呈非线性响应。Bri\`{e}re等\cite{briere1999}提出了一种描述昆虫发育速率温度依赖性的经验函数:
\begin{equation}
\label{eq:briere-intro}
r(T) = cT(T - T_{\min})\sqrt{T_{\max} - T}
\end{equation}
其中$c$为尺度参数,$T_{\min}$为发育温度下限,$T_{\max}$为发育温度上限。该函数在$T < T_{\min}$或$T > T_{\max}$时取零值,在两者之间呈不对称的单峰形态,最优温度$T_{\text{opt}}$偏向$T_{\max}$一侧。

Bri\`{e}re函数在蚊媒传染病建模中得到了广泛应用。Mordecai等\cite{mordecai2017}利用Bri\`{e}re函数拟合了白纹伊蚊和埃及伊蚊的多个生命史参数(叮咬率、产卵率、幼虫发育速率、成蚊存活率等)对温度的响应曲线,并将这些组分整合为温度依赖的基本再生数$R_0(T)$。Johnson等\cite{johnson2015}系统比较了Bri\`{e}re函数与其他温度响应函数(如Lactin函数、Sharpe-DeMichele模型等),发现Bri\`{e}re函数在拟合精度和参数可解释性之间取得了良好的平衡。

在本研究中,Bri\`{e}re函数被用于建模传播率$\beta'(T)$的温度依赖性。与纯数据驱动的神经网络方法相比,Bri\`{e}re函数具有以下优势:(1)~仅需3个参数($c, T_{\min}, T_{\max}$),在小样本条件下不易过拟合;(2)~参数具有明确的生物学含义,可与实验室数据交叉验证;(3)~函数形态由昆虫学理论约束,避免了数据驱动方法可能出现的非物理行为(如模式坍缩)。

\subsection{符号回归与知识蒸馏}
\label{sec:intro-sr}

符号回归(Symbolic Regression, SR)是一种从数据中自动发现数学表达式的机器学习方法。与传统回归方法(预设函数形式、仅优化参数)不同,符号回归同时搜索函数的结构和参数,输出人类可读的闭合公式。Cranmer等\cite{cranmer2023}开发的PySR是目前最先进的符号回归工具之一,基于多种群遗传编程算法,支持自定义运算符和约束条件。

在传染病建模领域,符号回归的应用尚处于起步阶段。Zhang等\cite{zhang2024plos}在2024年\textit{PLOS Computational Biology}发表了将符号回归应用于传染病模型参数发现的开创性工作——通过将蚊媒种群动力学模型耦合神经网络,有效揭示了伊蚊产卵率和温度、降水之间的关系,并使用符号回归确定最优函数表达式。然而,该方法面临以下挑战:(1)~直接在高维表达式空间中搜索计算成本极高;(2)~缺乏利用先验物理知识引导搜索的机制。

本研究采用"神经网络预训练+符号回归蒸馏"的两阶段策略:首先用神经网络学习气候变量到蚊媒密度的非线性映射,然后在神经网络生成的蒸馏数据上运行符号回归。这种策略的优势在于:(1)~神经网络可以在有噪声的真实数据上学习平滑的映射关系;(2)~符号回归在干净的蒸馏数据上搜索,降低了搜索难度和计算成本;(3)~最终输出的闭合公式兼具数据适应性和可解释性。

\subsection{物理先验与数据驱动方法的对比}
\label{sec:intro-physics-vs-data}

在传染病建模中,物理先验方法(Physics-Informed)和纯数据驱动方法(Data-Driven)代表了两种不同的建模哲学。物理先验方法利用已知的生物学、流行病学知识约束模型结构,减少需要从数据中学习的自由度;纯数据驱动方法则让模型自由学习数据中的模式,不施加先验约束。

Raissi等\cite{raissi2019}提出的物理信息神经网络(Physics-Informed Neural Networks, PINNs)是将物理先验融入深度学习的代表性工作,通过在损失函数中加入物理方程的残差项,使神经网络的预测满足已知的物理定律。在传染病建模中,类似的思路已被用于将SEIR动力学约束融入神经网络训练\cite{zhang2021plos}。

然而,纯数据驱动方法在小样本条件下面临严峻挑战。当训练数据量有限且分布高度偏斜时(如登革热月度数据中大量零病例月),神经网络容易出现"模式坍缩"(Mode Collapse)——即模型退化为预测常数值,因为这在偏斜分布下可以最小化平均损失。Holm\cite{holm2019}在\textit{Science}上指出,黑箱模型在科学应用中的可解释性问题不容忽视,尤其是在需要机制理解的公共卫生决策场景中。

本研究通过系统对比Bri\`{e}re物理先验方法和NN纯数据驱动方法,为传染病建模中的方法选择提供实证依据。具体而言,在相同的数据条件下(广州180个月度样本,64\%为低/零病例月),比较两种方法在拟合精度、交叉验证、空间迁移和时间泛化等维度上的表现差异。

\subsection{研究目标与创新点}
\label{sec:intro-objective}

基于上述研究背景,本文的核心研究目标为:

\begin{enumerate}[leftmargin=2em]
\item 建立基于Bri\`{e}re温度响应函数的登革热传播率模型,利用物理先验约束传播率的函数形态,避免小样本条件下的模式坍缩问题。
\item 通过神经网络+符号回归的知识蒸馏策略,从气候数据中发现可解释的蚊媒密度公式,为传播率模型提供蚊媒密度输入。
\item 设计共享物理参数+逐城市校准的两阶段迁移策略,将单城市模型推广至广东省16个地级市,验证模型的空间泛化能力。
\item 利用2020--2026年独立时间外数据验证模型的时间泛化能力,评估模型在训练期外的预测稳定性。
\item 系统对比物理先验方法与纯数据驱动方法,揭示小样本传染病建模中物理先验的优势。
\end{enumerate}

本研究的主要创新点包括:

\begin{enumerate}[leftmargin=2em]
\item \textbf{物理先验混合建模框架}:提出"Bri\`{e}re物理先验+PySR蚊媒公式+SEIR动力学"的可解释混合框架,克服了传统SEIR模型依赖先验函数形式的局限,同时避免了纯数据驱动方法的模式坍缩问题。
\item \textbf{城市内归一化蚊媒公式}:通过BI城市内归一化策略消除城市间基线差异(5.9倍),使LOCO CV均值$R^2$从$-1.39$提升至$+0.14$,显著改善了蚊媒公式的跨城泛化能力。
\item \textbf{两阶段空间迁移策略}:共享Bri\`{e}re物理参数(蚊虫温度生理特性)+逐城市$\eta$校准(非气候因素差异),在保持物理一致性的同时适应城市间异质性。
\item \textbf{独立时间外验证}:首次利用2020--2026年新BI监测数据对登革热传播率模型进行独立时间外验证,季节性相关$r=0.920$。
\item \textbf{物理先验vs.数据驱动的实证对比}:系统揭示NN在180个月度样本条件下的模式坍缩现象(变异系数1.17\%),为传染病建模方法选择提供实证依据。
\end{enumerate}

\subsection{全文结构}
\label{sec:intro-structure}

本文其余部分组织如下:

\textbf{第二章(第一部分:单城市机制发现)}以广州市为核心案例,详细阐述数据来源与预处理方法、蚊媒密度公式发现流程(NN+PySR符号回归)、Bri\`{e}re传播率模型构建与参数优化、SEIR离散时间预测框架,以及NN纯数据驱动方法的对比实验。呈现蚊媒公式发现、Bri\`{e}re参数估计、广州拟合与LOYO交叉验证、NN模式坍缩分析等结果,并进行深入讨论。

\textbf{第三章(第二部分:多城市迁移与外部验证)}将广州发现的Bri\`{e}re模型迁移至广东省16个地级市,介绍共享物理参数+逐城市$\eta$校准的两阶段迁移策略,呈现16城月度预测、年度排名验证和改进前后对比结果。同时利用2020--2026年新BI数据进行独立时间外验证,包括逐年相关性分析和MOI补充分析。

\textbf{第四章(总结与展望)}总结主要发现和创新点,讨论研究局限性(蚊媒数据稀缺、低发病城市预测受限、空间耦合缺失等),提出未来改进方向(遥感蚊媒估计、零膨胀模型、气候变化情景预测等)。
