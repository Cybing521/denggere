%% ==================================================================
\section{总结与展望}
\label{sec:conclusion}
%% ==================================================================

\subsection{主要结论}

本文围绕"如何从数据中自动发现气候驱动登革热传播效率的数学规律"这一核心问题,提出并验证了"SEIR动力学+神经网络+符号回归"三阶段混合建模框架。主要结论如下:

\begin{enumerate}[leftmargin=2em]
\item \textbf{耦合模型可学习传播信号}:广州SEIR+MLP耦合模型Spearman~$\rho=0.705$,Pearson~$r=0.612$,对数$R^2=0.450$,MAE$=51.23$,证实气候变量对$\beta'$的可观测调控作用。

\item \textbf{符号回归发现高精度闭合公式}:多项式族二次+交互项公式$R^2=0.999987$,10个系数具有明确物理含义。关键发现:温度--降水正交互($a_{TR}>0$)和降水平方负效应($a_{RR}<0$)。

\item \textbf{公式具有多城市泛化能力}:16城年度排名Spearman~$\rho=0.900$($p=2.05\times10^{-6}$),非广州15城MAE$=61.8$、RMSE$=116.8$。月度中位$r=0.481$、中位$\rho=0.469$。

\item \textbf{新数据集显著优于旧数据集}:排名$\rho$从0.713提升至0.879,MAE从504.7降至61.8。

\item \textbf{极端暴发由非气候因素主导}:2014年$\beta'$均值0.183539 vs 正常年0.183585,差异微小。
\end{enumerate}

\subsection{创新点总结}

\begin{enumerate}[leftmargin=2em]
\item \textbf{方法论创新}:"NN逆问题+符号蒸馏"范式,克服了Li等(2019 PNAS)样条方法和Zhang等(2024)纯符号搜索方法的局限。

\item \textbf{多变量联合发现}:首次在SEIR框架下同时纳入温度、降水和湿度三个气候变量及其二次交互效应,发现物理可解释的多变量闭合公式。

\item \textbf{空间可迁移验证}:首次在中国南方16城市尺度上系统验证单城市传播效率公式的空间泛化性能,建立"排名优先"评估框架。
\end{enumerate}

\subsection{局限性与未来展望}

\begin{enumerate}[leftmargin=2em]
\item \textbf{蚊媒数据局限}:BI指数仅广州可用。未来可利用遥感数据(NDVI、地表水面积指数)构建空间连续蚊媒密度估计。

\item \textbf{人口动态未建模}:使用固定中点人口。未来可引入逐年人口数据和空间分布信息。

\item \textbf{时间分辨率限制}:月度/双周为最小单元。提高至周/日尺度可能改善峰值预测。

\item \textbf{空间耦合缺失}:各城市视为独立系统。未来可引入元群落结构或引力模型描述城市间传播网络。

\item \textbf{气候变化情景预测}:$\beta'$公式可与气候模型耦合预测未来传播效率变化趋势,但需注意外推可靠性。
\end{enumerate}

综合而言,本文提出的"NN耦合动力学+符号蒸馏"框架为传染病机制发现提供了兼顾数据适应性和物理可解释性的新路径,原则上可应用于任何具有气候--传播耦合关系的蚊媒传染病以及其他需要从数据中发现机制性规律的传染病建模问题。
