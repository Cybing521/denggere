\documentclass[12pt,a4paper]{article}
\usepackage[UTF8]{ctex}
\usepackage{amsmath,amssymb,amsfonts}
\usepackage{graphicx}
\usepackage{booktabs}
\usepackage{geometry}
\usepackage{xcolor}
\usepackage{tcolorbox}
\usepackage{float}
\usepackage{hyperref}
\usepackage{longtable}

\geometry{left=2.5cm,right=2.5cm,top=2.5cm,bottom=2.5cm}

\title{\textbf{登革热蚊媒-人群动力学模型} \\ \large SEI-SEIR 模型技术文档}
\author{蚊媒传染病建模项目}
\date{\today}

\begin{document}

\maketitle

\tableofcontents
\newpage

%===========================================
\section{模型概述}
%===========================================

本模型采用 \textbf{SEI-SEIR} 结构,描述登革热在蚊媒和人群之间的传播动态:

\begin{itemize}
    \item \textbf{蚊媒水生期}: 幼虫 (L) $\to$ 蛹 (P)
    \item \textbf{蚊媒成虫}: 易感 (S$_m$) $\to$ 暴露 (E$_m$) $\to$ 感染 (I$_m$)
    \item \textbf{人群}: 易感 (S$_h$) $\to$ 暴露 (E$_h$) $\to$ 感染 (I$_h$) $\to$ 恢复 (R$_h$)
\end{itemize}

\begin{tcolorbox}[colback=blue!5!white,colframe=blue!75!black,title=模型特点]
\begin{enumerate}
    \item 完整的蚊虫生命周期 (幼虫→蛹→成蚊)
    \item 温度依赖的生物学参数
    \item 布雷图指数 (BI) 校正幼虫密度
    \item 输入病例机制
\end{enumerate}
\end{tcolorbox}

%===========================================
\section{完整微分方程组}
%===========================================

\subsection{蚊虫水生阶段}

\begin{tcolorbox}[colback=green!5!white,colframe=green!60!black,title=幼虫 (Larvae, L)]
\begin{equation}
\frac{dL}{dt} = \underbrace{\phi(T) \cdot \sigma_E \cdot M_{total} \cdot \text{BI}_{ratio}}_{\text{产卵输入}} - \underbrace{f_L(T) \cdot L}_{\text{发育为蛹}} - \underbrace{\mu_L(T) \cdot L \cdot \left(1 + \frac{L}{K_L}\right)}_{\text{密度依赖死亡}}
\end{equation}
\end{tcolorbox}

其中:
\begin{itemize}
    \item $\phi(T)$: 温度依赖的产卵率
    \item $\sigma_E = 0.6$: 卵存活率
    \item $M_{total} = S_m + E_m + I_m$: 成蚊总数
    \item $\text{BI}_{ratio} = \text{BI}(t) / \overline{\text{BI}}$: 归一化布雷图指数
    \item $f_L(T)$: 幼虫发育率
    \item $\mu_L(T)$: 幼虫死亡率
    \item $K_L = K_{base} \times 10^7 \times \text{BI}_{ratio}$: 承载力
\end{itemize}

\begin{tcolorbox}[colback=green!5!white,colframe=green!60!black,title=蛹 (Pupae, P)]
\begin{equation}
\frac{dP}{dt} = \underbrace{f_L(T) \cdot L}_{\text{幼虫发育}} - \underbrace{f_P(T) \cdot P}_{\text{羽化为成蚊}} - \underbrace{0.05 \cdot P}_{\text{死亡}}
\end{equation}
\end{tcolorbox}

\subsection{蚊虫成虫阶段 (SEI)}

\begin{tcolorbox}[colback=orange!5!white,colframe=orange!80!black,title=易感成蚊 (Susceptible, S$_m$)]
\begin{equation}
\frac{dS_m}{dt} = \underbrace{\sigma_P \cdot f_P(T) \cdot P}_{\text{羽化}} - \underbrace{\lambda_m \cdot S_m}_{\text{被感染}} - \underbrace{\mu_m(T) \cdot S_m}_{\text{死亡}}
\end{equation}
\end{tcolorbox}

\begin{tcolorbox}[colback=orange!5!white,colframe=orange!80!black,title=暴露成蚊 (Exposed, E$_m$)]
\begin{equation}
\frac{dE_m}{dt} = \underbrace{\lambda_m \cdot S_m}_{\text{新感染}} - \underbrace{\epsilon(T) \cdot E_m}_{\text{转为感染}} - \underbrace{\mu_m(T) \cdot E_m}_{\text{死亡}}
\end{equation}
\end{tcolorbox}

\begin{tcolorbox}[colback=orange!5!white,colframe=orange!80!black,title=感染成蚊 (Infectious, I$_m$)]
\begin{equation}
\frac{dI_m}{dt} = \underbrace{\epsilon(T) \cdot E_m}_{\text{暴露期结束}} - \underbrace{\mu_m(T) \cdot I_m}_{\text{死亡}}
\end{equation}
\end{tcolorbox}

蚊虫感染力:
\begin{equation}
\lambda_m = a(T) \cdot b_{hv}(T) \cdot \frac{I_h + \text{imp}}{N_H}
\end{equation}

\subsection{人群动态 (SEIR)}

\begin{tcolorbox}[colback=red!5!white,colframe=red!75!black,title=易感人群 (Susceptible, S$_h$)]
\begin{equation}
\frac{dS_h}{dt} = -\lambda_h \cdot S_h
\end{equation}
\end{tcolorbox}

\begin{tcolorbox}[colback=red!5!white,colframe=red!75!black,title=暴露人群 (Exposed, E$_h$)]
\begin{equation}
\frac{dE_h}{dt} = \underbrace{\lambda_h \cdot S_h}_{\text{新感染}} - \underbrace{\delta \cdot E_h}_{\text{转为发病}}
\end{equation}
\end{tcolorbox}

\begin{tcolorbox}[colback=red!5!white,colframe=red!75!black,title=感染人群 (Infectious, I$_h$)]
\begin{equation}
\frac{dI_h}{dt} = \underbrace{\delta \cdot E_h}_{\text{发病}} - \underbrace{\gamma \cdot I_h}_{\text{恢复}}
\end{equation}
\end{tcolorbox}

\begin{tcolorbox}[colback=red!5!white,colframe=red!75!black,title=恢复人群 (Recovered, R$_h$)]
\begin{equation}
\frac{dR_h}{dt} = \gamma \cdot I_h
\end{equation}
\end{tcolorbox}

人群感染力:
\begin{equation}
\lambda_h = a(T) \cdot b_{vh}(T) \cdot \frac{I_m}{N_H}
\end{equation}

\subsection{每周新发病例}

模型输出的周新发病例数:
\begin{equation}
\text{Cases}_{week} = \gamma \cdot I_h \times 7
\end{equation}

%===========================================
\section{温度依赖参数}
%===========================================

所有生物学参数均为温度 $T$ (°C) 的函数:

\subsection{产卵率}
\begin{equation}
\phi(T) = \max\left(0.1, \phi_{max} \cdot \exp\left(-\left(\frac{T-28}{7}\right)^2\right)\right)
\end{equation}
其中 $\phi_{max} = 4.0$,最适温度 28°C。

\subsection{发育率}

\textbf{幼虫发育率}:
\begin{equation}
f_L(T) = \max\left(0.01, 0.10 \cdot \exp\left(-\left(\frac{T-27}{9}\right)^2\right)\right)
\end{equation}

\textbf{蛹发育率}:
\begin{equation}
f_P(T) = \max\left(0.01, 0.15 \cdot \exp\left(-\left(\frac{T-27}{9}\right)^2\right)\right)
\end{equation}

\subsection{死亡率}

\textbf{幼虫死亡率}:
\begin{equation}
\mu_L(T) = \begin{cases}
0.30 & T < 15 \\
0.25 & T > 35 \\
\max(0.05, 0.15 - 0.006(T-15)) & \text{otherwise}
\end{cases}
\end{equation}

\textbf{成蚊死亡率} (U形曲线):
\begin{equation}
\mu_m(T) = \mu_{min} + 0.002 \cdot (T - T_{opt})^2
\end{equation}
其中 $\mu_{min} = 0.04$,$T_{opt} = 26$°C。

\subsection{叮咬与传播}

\textbf{叮咬率}:
\begin{equation}
a(T) = \max\left(0.1, 0.5 \cdot \exp\left(-\left(\frac{T-28}{8}\right)^2\right)\right)
\end{equation}

\textbf{人$\to$蚊传播概率}:
\begin{equation}
b_{hv}(T) = 0.4 \cdot \exp\left(-\left(\frac{T-27}{6}\right)^2\right)
\end{equation}

\textbf{蚊$\to$人传播概率}:
\begin{equation}
b_{vh}(T) = 0.45 \cdot \exp\left(-\left(\frac{T-26}{6}\right)^2\right)
\end{equation}

\subsection{外潜伏期 (EIP)}

蚊虫外潜伏期转化率:
\begin{equation}
\epsilon(T) = \begin{cases}
0.05 & T < 18 \\
0.20 & T > 32 \\
0.08 + 0.01(T-20) & \text{otherwise}
\end{cases}
\end{equation}

%===========================================
\section{固定参数}
%===========================================

\begin{table}[H]
\centering
\caption{固定生物学参数}
\begin{tabular}{lccc}
\toprule
\textbf{参数} & \textbf{符号} & \textbf{值} & \textbf{说明} \\
\midrule
卵存活率 & $\sigma_E$ & 0.6 & 卵→幼虫存活 \\
蛹存活率 & $\sigma_P$ & 0.7 & 蛹→成蚊存活 \\
人潜伏期转化率 & $\delta$ & 1/5 天$^{-1}$ & 内潜伏期约5天 \\
人恢复率 & $\gamma$ & 1/7 天$^{-1}$ & 感染期约7天 \\
人口总数 & $N_H$ & 14,000,000 & 广州市人口 \\
\bottomrule
\end{tabular}
\end{table}

%===========================================
\section{估计参数(通过优化)}
%===========================================

以下参数通过差分进化算法 + Nelder-Mead局部优化估计:

\begin{table}[H]
\centering
\caption{优化估计的参数}
\begin{tabular}{lccl}
\toprule
\textbf{参数} & \textbf{符号} & \textbf{优化范围} & \textbf{含义} \\
\midrule
繁殖缩放因子 & $k_{scale}$ & [0.1, 20] & 调整产卵率和承载力 \\
传播缩放因子 & $b_{scale}$ & [0.01, 5] & 调整传播效率 \\
输入病例率 & imp & [0, 50] & 每周外部输入病例数 \\
基础承载力因子 & $K_{base}$ & [0.1, 10] & 幼虫承载力基数 \\
\bottomrule
\end{tabular}
\end{table}

\textbf{注}: 这些参数目前被估计为\textbf{常数}。如需进行符号回归,需将 $b_{scale}$ 改为时变参数 $\beta(t)$。

%===========================================
\section{BI 校正机制}
%===========================================

布雷图指数 (BI) 是蚊媒幼虫密度的\textbf{相对指标},模型中通过以下方式校正:

\subsection{BI归一化}
\begin{equation}
\text{BI}_{ratio}(t) = \frac{\text{BI}(t)}{\overline{\text{BI}}}
\end{equation}

\subsection{校正方式}

\begin{enumerate}
    \item \textbf{产卵输入}: 
    \begin{equation}
    \text{产卵} = \phi(T) \cdot \sigma_E \cdot M_{total} \cdot \textcolor{red}{\text{BI}_{ratio}}
    \end{equation}
    BI高 → 环境适宜 → 产卵增加
    
    \item \textbf{环境承载力}: 
    \begin{equation}
    K_L = K_{base} \times 10^7 \times \textcolor{red}{\text{BI}_{ratio}}
    \end{equation}
    BI高 → 栖息地多 → 承载力大
\end{enumerate}

\subsection{验证指标}
通过计算模型幼虫 $L(t)$ 与观测 BI 的相关系数来验证校正效果:
\begin{equation}
r_{BI-L} = \text{Corr}\left(\text{BI}_{normalized}, \frac{L(t)}{\overline{L}}\right)
\end{equation}

%===========================================
\section{基本再生数 R$_0$}
%===========================================

根据 Ross-MacDonald 模型推导:

\begin{equation}
R_0 = \sqrt{\frac{a^2 \cdot b_{vh} \cdot b_{hv} \cdot M \cdot \epsilon}{\mu_m \cdot (\epsilon + \mu_m) \cdot \gamma \cdot N_H}}
\end{equation}

其中:
\begin{itemize}
    \item $a$: 叮咬率
    \item $b_{vh}, b_{hv}$: 传播概率
    \item $M$: 成蚊总数
    \item $\epsilon$: EIP转化率
    \item $\mu_m$: 成蚊死亡率
    \item $\gamma$: 人恢复率
    \item $N_H$: 人口总数
\end{itemize}

\textbf{解释}: $R_0 > 1$ 表示疫情可能扩散,$R_0 < 1$ 表示疫情将消退。

%===========================================
\section{时间尺度说明}
%===========================================

\begin{tcolorbox}[colback=yellow!5!white,colframe=yellow!50!black,title=重要说明]
\begin{itemize}
    \item \textbf{ODE积分步长}: 1天 ($dt = 1$)
    \item \textbf{数据尺度}: 周 (月度数据插值为周)
    \item \textbf{参数单位}: 天$^{-1}$
    \item \textbf{周数据索引}: $\text{idx} = \lfloor t/7 \rfloor$
\end{itemize}
周新发病例 = 日发病率 × 7
\end{tcolorbox}

%===========================================
\section{模型结构图}
%===========================================

\begin{figure}[H]
\centering
\begin{verbatim}
                    温度 T(t)
                       |
                       v
    +------------------+------------------+
    |                  |                  |
    v                  v                  v
  phi(T)            mu_m(T)           b(T)
    |                  |                  |
    v                  v                  v
+-------+  f_L   +-------+  f_P   +----------------+
|   L   | -----> |   P   | -----> |  S_m → E_m → I_m  |
| 幼虫   |        |  蛹   |        |     成蚊 SEI      |
+-------+        +-------+        +----------------+
    ^                                      |
    |                                      | b_vh
    |  BI校正                              v
    |                              +----------------+
+-------+                          | S_h → E_h → I_h → R_h |
|  BI   |                          |     人群 SEIR        |
+-------+                          +----------------+
                                           ^
                                           |
                                   imp (输入病例)
\end{verbatim}
\caption{SEI-SEIR 模型结构}
\end{figure}

%===========================================
\section{模型性能(2015-2019广东数据)}
%===========================================

\begin{table}[H]
\centering
\caption{模型评估指标}
\begin{tabular}{lc}
\toprule
\textbf{指标} & \textbf{值} \\
\midrule
Pearson 相关系数 $r$ & 0.70 - 0.76 \\
$R^2$ (对数尺度) & 0.50 - 0.55 \\
BI-幼虫相关 & 0.60 - 0.75 \\
$R_0$ 范围 & 0.5 - 2.5 \\
\bottomrule
\end{tabular}
\end{table}

%===========================================
\section{符号回归扩展(待实现)}
%===========================================

当前模型的 $b_{scale}$ 为常数。如需进行符号回归,需:

\begin{enumerate}
    \item 将 $b_{scale}$ 改为时变参数 $\beta(t)$
    \item 对每周独立估计 $\beta(t)$ 序列
    \item 使用符号回归寻找 $\beta(t) = f(T, \text{humidity}, \text{BI})$
\end{enumerate}

理论公式(基于MLP预测结果):
\begin{equation}
P(T) \approx A \cdot \exp\left(-\left(\frac{T - T_{opt}}{\sigma}\right)^2\right)
\end{equation}
其中 $A \approx 0.35$,$T_{opt} \approx 26.5$°C,$\sigma \approx 5.8$。

\end{document}
