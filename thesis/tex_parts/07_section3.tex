%% ==================================================================
\section{第二部分:Bri\`{e}re-SEIR动力学建模与多城市验证}
\label{sec:part2}
%% ==================================================================

\subsection{引言}

第一部分通过符号回归发现了融合温度、湿度与降水的蚊媒密度公式$\hat{M}$。本章在此基础上,构建基于Bri\`{e}re温度响应函数的SEIR动力学模型,首先在广州市完成参数估计和拟合验证,随后通过两阶段迁移策略推广至广东省16个地级市,并利用2020--2026年独立数据进行时间外验证。

\subsection{数据与方法}

\subsubsection{广州市数据}
\label{sec:p2-gz-data}

广州市(23.13°N, 113.26°E)位于广东省中南部,属南亚热带季风气候,年均气温21.4--22.6°C,年均降水量1600--1900\,mm。全市常住人口约1,426万人($N_h = 1.426 \times 10^7$)。本研究使用广州市2005--2019年月度登革热报告病例数据(180个月度观测值)和对应的月度气象数据(温度$T$、湿度$H$、降水$R$)。病例数据呈极端右偏分布:64.4\%的月份报告病例数$\leq$5例,而2014年暴发年全年累计报告37,382例。图\ref{fig:gz-overview}展示了广州市病例与气象变量的时间序列。

\begin{figure}[H]
\centering
\includegraphics[width=0.95\textwidth]{../figures/fig0a_guangzhou_data_overview.png}
\caption{广州市2005--2019年月度登革热病例数(对数坐标)及气象变量(温度、湿度、降水)时间序列。}
\label{fig:gz-overview}
\end{figure}图\ref{fig:gz-climate}展示了广州市气候与病例的时间序列。

\begin{figure}[H]
\centering
\includegraphics[width=0.95\textwidth]{../figures/fig0a_guangzhou_climate_cases.png}
\caption{广州市2005--2019年月度气候变量与登革热病例时间序列。从上到下:病例数(对数坐标)、温度、湿度、降水。}
\label{fig:gz-climate}
\end{figure}

\subsubsection{Bri\`{e}re传播率模型}
\label{sec:p2-briere}

传播率$\beta$的温度依赖性采用Bri\`{e}re函数\cite{briere1999}建模:
\begin{equation}
\label{eq:briere}
\beta'(T) = c \cdot T \cdot (T - T_{\min}) \cdot \sqrt{T_{\max} - T}, \quad T_{\min} \leq T \leq T_{\max}
\end{equation}
其中$c$为形状系数,$T_{\min}$为发育零点温度(低于此温度蚊虫停止发育),$T_{\max}$为热致死温度(高于此温度蚊虫死亡)。最适温度$T_{\text{opt}}$由$\beta'(T)$的极值点确定。选择Bri\`{e}re函数的理由在于:(1)其不对称单峰形态与蚊媒传播的温度响应实验数据高度吻合\cite{mordecai2017};(2)三个参数均具有明确的昆虫学含义;(3)仅需3个自由参数,在180个月度样本的小样本条件下不易过拟合。

\subsubsection{SEIR离散时间预测框架}
\label{sec:p2-seir}

采用离散时间SEIR仓室模型,以月为时间步长:
\begin{equation}
\label{eq:seir-discrete}
\begin{aligned}
\Delta E_t &= \beta'(T_t) \cdot \hat{M}_t \cdot S_t / N_h + \eta - \sigma E_t \\
\Delta I_t &= \sigma E_t - \gamma I_t \\
C_t &= \sigma E_t
\end{aligned}
\end{equation}
其中$\beta'(T_t)$为Bri\`{e}re温度响应函数,$\hat{M}_t$为第一部分发现的蚊媒密度公式,$S_t$为易感人群池,$\eta$为背景输入率(捕捉非气候因素的恒定输入),$\sigma = 1/7$天$^{-1}$为潜伏期倒数,$\gamma = 1/7$天$^{-1}$为恢复率。$C_t$为模型预测的月度新增病例数。

参数优化采用差分进化(Differential Evolution, DE)全局优化算法,目标函数为Spearman~$\rho$(最大化预测序列与观测序列的排名相关性)。优化参数包括Bri\`{e}re三参数$(c, T_{\min}, T_{\max})$和广州$\eta$。

\subsubsection{评估指标体系}
\label{sec:p2-metrics}

模型评估采用多维度指标体系:(1)Spearman~$\rho$衡量排名一致性;(2)对数$R^2$($R^2_{\log}$)衡量对数空间拟合优度;(3)加权绝对百分比误差(WAPE)衡量量级准确性;(4)留一年交叉验证(LOYO CV)评估时间泛化能力。

\subsubsection{16城数据概况}
\label{sec:p2-16city-data}

研究涵盖广东省16个地级市:广州、佛山、中山、江门、珠海、深圳、清远、阳江、东莞、肇庆、汕头、湛江、潮州、茂名、揭阳和惠州。数据时间范围为2005--2019年,月度分辨率,共计180个月$\times$16城$=$2880个城市-月观测值。气象数据(日均温度、降水量、相对湿度)来源于NOAA全球地面观测日报(GSOD),按月聚合为月均温度、月累计降水和月均相对湿度。登革热病例数据来源于中国疾病预防控制中心法定传染病报告系统。

广东省16城的登革热发病水平差异极为悬殊。以2014年大流行年为例,广州全年报告37,382例,而惠州仅37例,跨越三个数量级。在整个研究期间(2005--2019),广州累计病例数占16城总量的60\%以上,佛山、中山、深圳等珠三角核心城市次之,而清远、肇庆、揭阳等粤北和粤东城市的年均病例数不足10例。这种高度偏斜的分布特征对模型迁移提出了严峻挑战:模型需要同时适应高发病城市的复杂动态和低发病城市的稀疏信号。

在16个城市中,有8个城市(广州、深圳、东莞、汕头、江门、揭阳、茂名、惠州)在研究期间拥有布雷图指数(BI)监测数据,其余8个城市(佛山、中山、珠海、清远、阳江、肇庆、湛江、潮州)缺乏BI数据。这一自然分组为后续分析提供了一个有价值的对照:可以比较有无BI数据对模型迁移效果的影响,从而间接评估蚊媒监测数据在模型校准中的作用。

\subsubsection{两阶段迁移策略}

多城市迁移采用两阶段策略,其核心思想是将模型参数分为"物种共享"和"城市特异"两类,分别在不同阶段确定。

\textbf{阶段1:共享Bri\`{e}re物理参数。}Bri\`{e}re函数的三个参数$(c, T_{\min}, T_{\max})$在广州数据上通过差分进化(Differential Evolution, DE)全局优化获得,随后作为所有16个城市共享的物理先验固定不变。这一设计的理论依据来自昆虫生理学:Bri\`{e}re函数描述的是白纹伊蚊(\textit{Aedes albopictus})对温度的生理响应曲线,包括发育零点温度$T_{\min}$、热致死温度$T_{\max}$和形状系数$c$。这些参数反映的是物种水平的温度生理特性,由蚊虫的遗传背景决定,不因城市环境的不同而改变\cite{mordecai2017,mordecai2019}。Mordecai等\cite{mordecai2017}的实验室研究表明,同一蚊种在不同地理种群间的温度响应曲线高度一致,支持了跨城市共享物理参数的合理性。

具体而言,Bri\`{e}re传播率函数的形式为:
\begin{equation}
\beta'(T) = c \cdot T \cdot (T - T_{\min}) \cdot \sqrt{T_{\max} - T}, \quad T_{\min} \leq T \leq T_{\max}
\label{eq:briere-transfer}
\end{equation}
其中$T$为月均温度($^\circ$C)。当$T < T_{\min}$或$T > T_{\max}$时,$\beta'(T) = 0$。该函数的三个参数在广州DE优化中已确定,在迁移阶段保持固定。

\textbf{阶段2:逐城市$\eta$网格搜索。}固定Bri\`{e}re参数后,对每个城市独立校准输入率参数$\eta$。$\eta$在SEIR模型中控制外部感染输入的强度,其物理含义是单位时间内从外部引入的感染个体数占易感人群的比例。在多城市迁移框架中,$\eta$承担了吸收各城市非气候因素差异的关键角色:不同城市的蚊媒基线密度、人口流动性、城市化水平、公共卫生防控力度等因素的综合效应,均通过$\eta$的差异来体现。

$\eta$的校准采用80点对数均匀网格搜索(log-uniform grid search),搜索范围覆盖$[10^{-2}, 10^{2}]$。对数网格的设计考虑了$\eta$可能跨越多个数量级的特点。对于每个候选$\eta$值,运行完整的SEIR模拟并计算目标函数:
\begin{equation}
\mathcal{L}(\eta) = \text{MSE}_{\log} - \lambda \cdot \rho_s
\label{eq:eta-objective}
\end{equation}
其中$\text{MSE}_{\log}$为对数空间均方误差,$\rho_s$为Spearman相关系数,$\lambda$为平衡系数。该目标函数同时优化量级拟合和时序排名能力。选择使$\mathcal{L}(\eta)$最小的$\eta$作为该城市的最优值。

\subsubsection{量级校准方法}

在完成$\eta$校准后,对每个城市进一步实施量级校准(ratio scaling),以消除系统性的量级偏差。具体方法为:在非2014年数据(即排除验证年后的训练期数据)上,计算观测病例均值与模型预测均值的比值作为缩放因子:
\begin{equation}
\text{scale}_i = \frac{\overline{y}_i^{\text{obs}}}{\overline{y}_i^{\text{pred}}}, \quad i = 1, 2, \ldots, 16
\label{eq:ratio-scaling}
\end{equation}
其中$\overline{y}_i^{\text{obs}}$和$\overline{y}_i^{\text{pred}}$分别为城市$i$在非2014年月份的观测和预测病例均值。校准后的预测值为$\hat{y}_i^{\text{cal}} = \text{scale}_i \cdot \hat{y}_i^{\text{pred}}$。

量级校准的必要性在于:SEIR模型的绝对输出量级受人口基数$N$、初始条件和$\eta$等多个参数的联合影响,即使$\eta$已经过优化,不同城市的预测量级仍可能存在系统性偏差。ratio scaling通过一个简单的线性缩放即可消除这种偏差,且不改变预测的时序形态(因此不影响Spearman~$\rho$)。排除2014年数据是为了保证年度排名验证的独立性。

\subsubsection{评估策略}

多城市迁移评估采用"排名优先"(ranking-first)策略。这一策略的核心理念是:在跨城市比较中,准确预测城市间的相对风险等级比精确预测绝对病例数更具公共卫生意义。原因在于:(1)不同城市的病例报告率、诊断标准和监测覆盖率可能存在系统性差异,绝对数值的可比性有限;(2)公共卫生资源分配决策通常基于风险排名而非绝对数值;(3)排名指标对量级缩放不敏感,更能反映模型捕捉传播动态的本质能力。

具体评估指标包括三个层次:

(1)\textbf{Spearman秩相关系数$\rho$}(首要指标):衡量模型预测与观测值在时序排名上的一致性。$\rho = 1$表示完美排名一致,$\rho = 0$表示无排名相关性。该指标对异常值和量级缩放均具有鲁棒性。

(2)\textbf{对数空间决定系数$R^2_{\log}$}(辅助指标):定义为$R^2_{\log} = 1 - \text{SS}_{\text{res}} / \text{SS}_{\text{tot}}$,其中残差和总变异均在$\log(y+1)$空间计算。对数变换压缩了高发病月份的量级差异,使低发病月份的拟合质量也能得到合理评估。$R^2_{\log} > 0$表示模型优于对数空间均值基线。

(3)\textbf{加权绝对百分比误差WAPE}(量级指标):定义为$\text{WAPE} = \sum|y_i - \hat{y}_i| / \sum y_i$,衡量预测的绝对量级准确性。WAPE$<1$表示总误差小于总观测值,WAPE$=0$表示完美预测。

对于年度验证,以2014年为验证年,计算16城年度总病例的Spearman排名相关系数。2014年是广东省登革热历史上最严重的大流行年,各城市病例数差异显著,为排名验证提供了理想的测试场景。全部15年(2005--2019)数据用于月度评估。

\subsubsection{外部时间验证数据}

为检验模型的时间泛化能力,本研究利用2020--2026年广州市新采集的蚊媒监测数据作为完全独立的外部验证集。该数据集包含60个月度样本,涵盖两个核心蚊媒指标:布雷图指数(Breteau Index, BI)和蚊虫叮咬指数(Mosquito Oviposition Index, MOI)。BI定义为每百户阳性容器数,是我国登革热防控中最常用的蚊媒密度指标\cite{lai2015};MOI基于诱蚊诱卵器监测,反映成蚊产卵活跃度。

该验证数据集与模型训练数据(2005--2019年)在时间上完全不重叠,确保了验证的独立性。需要特别指出的是,2020--2026年期间经历了COVID-19大流行及其后续影响,社区防控措施(如大规模消杀、居民出行限制等)可能对蚊媒密度产生了显著干扰。新数据的BI均值为2.67,远低于训练期BI均值6.18,这一差异可能部分反映了COVID-19防控措施对蚊媒环境的间接影响。

原始BI数据的时间粒度不完全统一:部分年份为月度数据,部分年份为旬度(每10天)或周度数据。为保持与模型输出的一致性,对非月度数据按月聚合取均值。聚合后的月度BI序列用于与Bri\`{e}re~$\beta'(T)$和PySR~$\hat{M}$进行相关性分析。

\subsection{结果}

\subsubsection{Bri\`{e}re传播率参数}
\label{sec:p2-res-briere}

差分进化算法在广州2005--2019年数据上优化得到的Bri\`{e}re函数参数为:尺度系数$c = 7.15 \times 10^{-4}$,发育温度下限$T_{\min} = 15.1$°C,发育温度上限$T_{\max} = 42.0$°C。由此计算的最优传播温度为$T_{\text{opt}} = 35.5$°C,背景输入率$\eta = 0.709$。

$T_{\min} = 15.1$°C落在文献报道的白纹伊蚊发育温度下限范围(14--16°C)之内\cite{mordecai2017,brady2013}。$T_{\max} = 42.0$°C略高于部分实验室报道的上限(约38--40°C),可能因为广州月均温极少超过35°C,高温端数据约束较弱。$T_{\text{opt}} = 35.5$°C高于Mordecai等\cite{mordecai2019}报道的最优传播温度(约29°C),但两者不具有直接可比性——Mordecai等的估计基于$R_0$的多个组分,而本研究的$\beta'(T)$仅反映传播率的温度依赖性,蚊虫存活率等因素已部分被$\hat{M}$所捕捉。

\begin{figure}[H]
\centering
\includegraphics[width=0.7\textwidth]{../figures/fig5_briere_curve.png}
\caption{Bri\`{e}re传播率函数$\beta'(T)$曲线。$T_{\min}=15.1^\circ$C, $T_{\max}=42.0^\circ$C, $T_{\text{opt}}=35.5^\circ$C。}
\label{fig:briere-curve}
\end{figure}

\subsubsection{广州拟合与交叉验证}
\label{sec:p2-res-gz}

表\ref{tab:gz-metrics}汇总了Bri\`{e}re模型在广州市不同评估场景下的预测性能。

\begin{table}[H]
\centering
\caption{Bri\`{e}re模型在广州市的预测性能}
\label{tab:gz-metrics}
\begin{tabular}{lcccc}
\toprule
评估场景 & Spearman $\rho$ & $R^2_{\log}$ & Pearson $r$ & WAPE \\
\midrule
全期拟合(2005--2019) & 0.814 & 0.851 & 0.714 & 0.679 \\
LOYO CV 全部年份 & $0.740 \pm 0.185$ & $0.442 \pm 0.534$ & --- & --- \\
LOYO CV 暴发年(病例$>$50) & $0.806 \pm 0.098$ & $0.660 \pm 0.218$ & --- & --- \\
LOYO CV 低发年 & $0.561 \pm 0.218$ & --- & --- & --- \\
\bottomrule
\end{tabular}
\end{table}

全期拟合$\rho = 0.814$、$R^2_{\log} = 0.851$,表明模型能够较好地捕捉广州登革热的时序变化模式。LOYO CV中,暴发年份平均$\rho = 0.806$,性能接近全期拟合水平;低发年份$\rho = 0.561$,性能下降,因为低发年份信噪比极低,传播主要由随机输入事件驱动。

\begin{figure}[H]
\centering
\includegraphics[width=0.85\textwidth]{../figures/fig6_guangzhou_fit.png}
\caption{广州市2005--2019年月度病例观测值与Bri\`{e}re模型预测值对比。}
\label{fig:gz-fit}
\end{figure}

\begin{figure}[H]
\centering
\includegraphics[width=0.75\textwidth]{../figures/fig7_loyo_cv.png}
\caption{广州市留一年交叉验证(LOYO CV)各年Spearman $\rho$。}
\label{fig:loyo-cv}
\end{figure}

\subsubsection{逐城市$\eta$校准结果}

16个城市的最优$\eta$值范围为$[0.148, 15.273]$,跨越两个数量级,充分反映了城市间非气候因素的显著差异。表~\ref{tab:transfer-monthly}列出了各城市的$\eta$值及对应的预测指标。

从$\eta$值的分布来看,可以识别出几种典型模式:

(1)\textbf{低$\eta$城市}:深圳($\eta = 0.148$)和广州($\eta = 0.746$)。这两个城市是广东省经济最发达、公共卫生基础设施最完善的城市。低$\eta$意味着模型只需较小的外部输入即可重现观测到的发病模式,这与这些城市较高的蚊媒防控效率和较低的环境蚊媒基线密度相一致。深圳的$\eta$最低(0.148),可能反映了其作为年轻移民城市、城市化程度极高、蚊媒孳生环境相对有限的特点。

(2)\textbf{高$\eta$城市}:东莞($\eta = 15.273$)、佛山($\eta = 15.273$)、中山($\eta = 15.273$)和潮州($\eta = 11.052$)。高$\eta$值表明这些城市需要较大的外部输入才能匹配观测发病水平。对于东莞、佛山和中山等珠三角制造业城市,高$\eta$可能反映了大量流动人口带来的输入性病例风险;对于潮州等粤东城市,高$\eta$可能与较低的蚊媒监测覆盖率和不同的城市环境结构有关。

(3)\textbf{中等$\eta$城市}:茂名($\eta = 3.760$)、惠州($\eta = 3.031$)、江门($\eta = 5.788$)等。这些城市的$\eta$值处于中间水平,反映了介于高度城市化和欠发达地区之间的中等蚊媒环境条件。

值得注意的是,$\eta$值与城市发病水平之间并非简单的正相关关系。例如,广州发病量最高但$\eta$较低,而东莞发病量中等但$\eta$最高。这是因为$\eta$反映的是"在给定温度驱动下,需要多大的外部输入才能重现观测模式",它综合了人口基数、蚊媒密度、防控效率等多种因素的净效应。

\subsubsection{16城月度预测}

表~\ref{tab:transfer-monthly}汇总了16个城市的月度预测指标,按Spearman~$\rho$降序排列。

\begin{table}[H]
\centering
\caption{16城月度预测指标汇总(Bri\`{e}re迁移模型,2005--2019年)}
\label{tab:transfer-monthly}
\small
\begin{tabular}{llcccc}
\toprule
城市 & 有BI数据 & Spearman $\rho$ & $R^2_{\log}$ & WAPE & $\eta$ \\
\midrule
广州   & 是 & 0.814 & 0.848 & 0.804 & 0.746 \\
佛山   & 否 & 0.731 & 0.781 & 0.971 & 15.273 \\
深圳   & 是 & 0.687 & 0.699 & 0.738 & 0.148 \\
中山   & 否 & 0.685 & 0.778 & 0.731 & 15.273 \\
潮州   & 否 & 0.604 & 0.697 & 1.223 & 11.052 \\
东莞   & 是 & 0.555 & 0.656 & 0.570 & 15.273 \\
阳江   & 否 & 0.501 & 0.559 & 0.825 & 2.193 \\
汕头   & 是 & 0.493 & 0.528 & 1.115 & 11.052 \\
湛江   & 否 & 0.482 & 0.500 & 0.894 & 8.909 \\
江门   & 是 & 0.470 & 0.776 & 0.784 & 5.788 \\
揭阳   & 是 & 0.468 & 0.693 & 0.827 & 5.788 \\
茂名   & 是 & 0.443 & 0.677 & 0.716 & 3.760 \\
惠州   & 是 & 0.423 & 0.708 & 0.611 & 3.031 \\
肇庆   & 否 & 0.392 & 0.434 & 0.726 & 4.166 \\
珠海   & 否 & 0.381 & 0.456 & 0.951 & 15.273 \\
清远   & 否 & 0.364 & 0.688 & 0.722 & 5.196 \\
\midrule
\textbf{16城均值} & --- & \textbf{0.531} & \textbf{0.656} & \textbf{0.827} & --- \\
\bottomrule
\end{tabular}
\end{table}

16城月度Spearman~$\rho$均值为0.531,$R^2_{\log}$均值为0.656,WAPE均值为0.827。从整体来看,模型在大多数城市实现了正向的排名相关性和对数空间拟合优度,WAPE均值低于1.0表明总体量级误差可控。

按预测质量可将16城大致分为三个梯队:

\textbf{第一梯队($\rho > 0.65$)}:广州($\rho = 0.814$)、佛山($\rho = 0.731$)、深圳($\rho = 0.687$)和中山($\rho = 0.685$)。这四个城市均位于珠三角核心区域,登革热发病水平较高,数据信噪比好,模型能够准确捕捉其季节性波动和年际变化。广州作为训练城市表现最优符合预期,但佛山、深圳和中山作为迁移城市也达到了$\rho > 0.68$的水平,证明了Bri\`{e}re参数的跨城市适用性。

\textbf{第二梯队($0.45 \leq \rho \leq 0.65$)}:潮州($\rho = 0.604$)、东莞($\rho = 0.555$)、阳江($\rho = 0.501$)、汕头($\rho = 0.493$)、湛江($\rho = 0.482$)、江门($\rho = 0.470$)和揭阳($\rho = 0.468$)。这些城市的预测质量中等,模型能够捕捉大致的季节性趋势但对年际波动的预测精度有限。

\textbf{第三梯队($\rho < 0.45$)}:茂名($\rho = 0.443$)、惠州($\rho = 0.423$)、肇庆($\rho = 0.392$)、珠海($\rho = 0.381$)和清远($\rho = 0.364$)。这些城市的$\rho$较低,将在后续低发病城市分析中详细讨论。

\textbf{BI数据分组比较。}将16城按是否拥有BI监测数据分为两组进行比较:有BI数据的8个城市(广州、深圳、东莞、汕头、江门、揭阳、茂名、惠州)的均值$\rho = 0.544$,$R^2_{\log} = 0.673$;无BI数据的8个城市(佛山、中山、珠海、清远、阳江、肇庆、湛江、潮州)的均值$\rho = 0.517$,$R^2_{\log} = 0.639$。两组之间的差异较小($\Delta\rho = 0.027$,$\Delta R^2_{\log} = 0.034$),表明本迁移方法对BI数据的依赖性不强——即使在缺乏蚊媒监测数据的城市,仅凭气象数据和共享Bri\`{e}re参数也能实现与有BI数据城市相当的预测效果。这一发现具有重要的实际意义:在蚊媒监测网络尚未覆盖的地区,本方法仍可提供有价值的登革热风险评估。

\begin{figure}[H]
\centering
\includegraphics[width=0.95\textwidth]{../figures/fig8_16cities_grid.png}
\caption{16城月度病例观测值与预测值对比(Bri\`{e}re + per-city $\eta$)。}
\label{fig:16cities-grid}
\end{figure}

\subsubsection{年度排名验证}

以2014年为验证年,计算16城年度总病例数的Spearman排名相关系数。结果显示$\rho = 0.947$($p < 10^{-5}$),表明模型能够以极高的准确度预测城市间的相对风险等级。

2014年是广东省登革热有记录以来最严重的大流行年,全省报告病例超过45,000例,其中广州占80\%以上。在这一极端年份中,各城市的发病量差异跨越三个数量级(广州37,382例至惠州37例),为排名验证提供了充分的区分度。$\rho = 0.947$意味着模型预测的城市排名与实际排名几乎完全一致,仅有个别相邻排名的城市发生了微小的位次交换。

这一结果的公共卫生意义在于:即使模型在某些低发病城市的月度$\rho$较低,它仍然能够准确识别哪些城市面临更高的登革热风险。对于省级疾控部门而言,这种跨城市风险分层能力比单城市的精确预测更为实用——它可以指导防控资源的优先分配和预警阈值的差异化设定。

\begin{figure}[H]
\centering
\includegraphics[width=0.6\textwidth]{../figures/fig9_2014_ranking.png}
\caption{2014年16城年度总病例排名:观测值 vs 预测值(Spearman $\rho=0.947$)。}
\label{fig:2014-scatter}
\end{figure}

\subsubsection{改进前后对比}

为量化per-city $\eta$校准和量级缩放带来的改进,将改进后的结果与改进前(全局log-linear校准)的基线进行对比,如表~\ref{tab:improvement}所示。

\begin{table}[H]
\centering
\caption{改进前后16城迁移指标对比}
\label{tab:improvement}
\begin{tabular}{lccc}
\toprule
指标 & 改进前(全局log-linear) & 改进后(per-city $\eta$+scaling) & 变化 \\
\midrule
16城均值 Spearman $\rho$ & 0.531 & 0.531 & 持平 \\
16城均值 $R^2_{\log}$ & 0.641 & \textbf{0.656} & $+0.015$ \\
16城均值 WAPE & 0.909 & \textbf{0.827} & $-0.082$($-9.0\%$) \\
WAPE $> 1.0$ 城市数 & 4 & \textbf{2} & $-2$ \\
\bottomrule
\end{tabular}
\end{table}

改进效果体现在以下几个方面:

(1)\textbf{WAPE显著下降}:16城均值WAPE从0.909降至0.827,相对改善9.0\%。这意味着模型的绝对量级预测误差得到了实质性的压缩。更重要的是,WAPE$>1.0$的城市数从4个减少到2个(仅剩潮州和汕头),表明量级校准有效消除了大部分城市的系统性量级偏差。

(2)\textbf{$R^2_{\log}$小幅提升}:从0.641提升至0.656($+0.015$),表明对数空间的拟合优度也有所改善。虽然提升幅度不大,但方向一致且在所有城市中普遍存在。

(3)\textbf{$\rho$保持不变}:Spearman~$\rho$在改进前后完全一致(0.531),这一结果符合理论预期。per-city $\eta$校准和ratio scaling本质上是对预测序列进行单调变换(乘以正常数),不改变序列的秩次顺序,因此不影响Spearman相关系数。这也从侧面验证了改进方法的内在一致性:量级校准不会以牺牲排名能力为代价。

综合来看,per-city $\eta$+ratio scaling策略在保持排名能力不变的前提下,显著改善了量级预测的准确性,是一种"无损"的改进方案。

\subsubsection{低发病城市分析}

第三梯队城市(清远$\rho = 0.364$、珠海$\rho = 0.381$、肇庆$\rho = 0.392$)的低$\rho$值需要从统计学角度加以理解,而非简单归因于模型失效。

这些城市的共同特征是极低的发病水平:在180个月的观测期内,约84\%的月份报告零病例。这种严重的零膨胀(zero-inflation)数据结构对任何基于连续值的预测模型都构成了根本性的统计挑战。具体而言:

(1)\textbf{秩次退化问题}。当大量观测值为零时,这些零值在排名中被赋予相同的平均秩次(tied ranks)。Spearman~$\rho$的计算依赖于秩次的变异性,而大量并列秩次严重压缩了秩次的有效变异范围,导致$\rho$的理论上限远低于1.0。即使模型完美预测了所有非零月份的排名,仅因零值月份的秩次退化就会大幅拉低$\rho$。

(2)\textbf{信噪比极低}。低发病城市的非零月份病例数通常仅为个位数(1--5例),这些小数值中包含大量随机噪声(如个别输入性病例、局部小规模暴发等),气候驱动的信号被噪声淹没。

(3)\textbf{$R^2_{\log}$仍为正值}。值得注意的是,即使$\rho$较低,这些城市的$R^2_{\log}$仍为正值(清远0.688、珠海0.456、肇庆0.434),表明在对数空间中模型仍优于均值基线。这说明模型在有病例发生的月份能够提供合理的量级估计,低$\rho$主要是零膨胀数据的统计伪影而非模型的系统性失败。

(4)\textbf{WAPE表现合理}。清远WAPE$=0.722$、肇庆WAPE$=0.726$,均低于1.0,进一步支持了模型在量级预测上的合理性。

从方法论角度看,低发病城市的预测改进需要引入零膨胀模型(如hurdle model或zero-inflated Poisson)来显式处理零值过多的问题,这超出了本研究SEIR框架的范畴,可作为未来研究方向。

\subsubsection{外部时间验证}

利用2020--2026年广州新BI数据对Bri\`{e}re~$\beta'(T)$进行独立时间验证,结果汇总于表~\ref{tab:external}和表~\ref{tab:external-yearly}。

\begin{table}[H]
\centering
\caption{Bri\`{e}re模型外部时间验证——整体指标(2020--2026年广州BI数据)}
\label{tab:external}
\begin{tabular}{lcc}
\toprule
验证项 & Pearson $r$ & Spearman $\rho$ \\
\midrule
Bri\`{e}re $\beta'(T)$ vs 观测BI(整体) & \textbf{0.782} & \textbf{0.765} \\
Bri\`{e}re 季节性轮廓 & \textbf{0.920} & 0.867 \\
PySR $\hat{M}$ vs 观测BI & 0.764 & 0.766 \\
MOI vs 观测BI & 0.875 & 0.873 \\
MOI vs $\beta'(T)$ & 0.860 & 0.839 \\
\bottomrule
\end{tabular}
\end{table}

\textbf{整体相关性。}Bri\`{e}re~$\beta'(T)$与观测BI的整体Pearson相关系数$r = 0.782$,Spearman~$\rho = 0.765$,均达到了较高水平。考虑到验证数据与训练数据在时间上完全不重叠(间隔至少1年),且验证期间经历了COVID-19大流行的严重干扰,这一相关性水平充分证明了Bri\`{e}re函数捕捉的温度--传播率关系具有时间稳定性,不是对历史数据的过拟合产物。

\textbf{季节性轮廓。}将$\beta'(T)$和BI按月份聚合为季节性轮廓后,相关系数进一步提升至$r = 0.920$,表明Bri\`{e}re函数对登革热传播季节性节律的刻画极为准确。季节性轮廓消除了年际波动的噪声,更纯粹地反映了温度驱动的周期性模式。

\textbf{PySR公式验证。}PySR符号回归发现的蚊媒适宜度公式$\hat{M}$与观测BI的相关性($r = 0.764$,$\rho = 0.766$)与$\beta'(T)$相当,表明$\hat{M}$中融合的降水和湿度信息在月度尺度上并未显著提升与BI的相关性。这可能是因为BI主要反映幼虫孳生容器中的积水状况,而月度降水的影响已部分被温度的季节性变化所代理。

\begin{table}[H]
\centering
\caption{Bri\`{e}re $\beta'(T)$ vs 观测BI——逐年Pearson相关系数}
\label{tab:external-yearly}
\begin{tabular}{lccl}
\toprule
年份 & Pearson $r$ & 样本月数 & 备注 \\
\midrule
2020 & 0.965 & 10 & COVID初期,BI仍有季节性 \\
2021 & 0.620 & 10 & COVID严格管控,BI受干扰 \\
2022 & 0.967 & 10 & 管控放松,季节性恢复 \\
2023 & 0.844 & 10 & 完全放开后首年 \\
2024 & 0.926 & 10 & 正常年份 \\
2025 & 0.655 & 10 & 整体BI偏低 \\
\bottomrule
\end{tabular}
\end{table}

\textbf{逐年分析。}6个验证年中,4年的$r > 0.8$(2020: $r = 0.965$,2022: $r = 0.967$,2023: $r = 0.844$,2024: $r = 0.926$),表现优异。2021年($r = 0.620$)和2025年($r = 0.655$)相关性偏低,原因各异:2021年正值COVID-19严格管控期,大规模社区消杀和人员流动限制显著压低了BI水平,破坏了温度与蚊媒密度之间的正常关联;2025年整体BI水平偏低,数据变异范围收窄,相关系数对噪声更为敏感。

\textbf{MOI交叉验证。}MOI与BI的相关性极高($r = 0.875$),证实了两种蚊媒监测指标的一致性。MOI与$\beta'(T)$的相关性($r = 0.860$)甚至略高于BI与$\beta'(T)$的相关性($r = 0.782$),这可能是因为MOI基于成蚊产卵活跃度,比BI(基于幼虫容器阳性率)更直接地反映成蚊的活动水平,而Bri\`{e}re函数描述的正是成蚊传播率与温度的关系。

\textbf{COVID效应的量化。}新数据BI均值(2.67)仅为训练期BI均值(6.18)的43\%,这一显著下降可能反映了COVID-19防控措施对蚊媒环境的间接影响(如社区清洁运动、减少户外积水容器等)。尽管BI绝对水平大幅下降,$\beta'(T)$与BI的季节性相关仍高达$r = 0.920$,说明温度驱动的季节性模式是稳健的,不受BI绝对水平变化的影响。

\begin{figure}[H]
\centering
\includegraphics[width=0.85\textwidth]{../figures/fig12_external_timeseries.png}
\caption{2020--2026年Bri\`{e}re $\beta'(T)$与观测BI的时间序列对比。}
\label{fig:ext-timeseries}
\end{figure}

\begin{figure}[H]
\centering
\includegraphics[width=0.7\textwidth]{../figures/fig14_external_peryear.png}
\caption{外部验证逐年Pearson相关系数。4/6年$r>0.8$。}
\label{fig:ext-peryear}
\end{figure}

\subsection{讨论}

\textbf{空间可迁移性的理论基础。}本章的核心发现是:共享Bri\`{e}re物理参数+逐城市$\eta$校准的两阶段策略能够在16个差异显著的城市中实现有意义的预测能力(16城均值$\rho = 0.531$,$R^2_{\log} = 0.656$,2014年度排名$\rho = 0.947$)。这一结果的理论基础在于Bri\`{e}re函数参数的物种特异性:$T_{\min}$、$T_{\max}$和$c$描述的是白纹伊蚊对温度的生理响应,由物种遗传背景决定,在广东省范围内可合理假设为空间不变量\cite{mordecai2017}。而城市间的差异——人口规模、城市化水平、蚊媒基线密度、防控力度——则通过$\eta$参数来吸收。这种"物理共享+环境特异"的分离策略,既保留了模型的机理可解释性,又提供了足够的灵活性来适应不同城市的条件。

\textbf{Per-city $\eta$的流行病学含义。}$\eta$值的两个数量级变异($[0.148, 15.273]$)揭示了广东省城市间非气候因素差异的巨大幅度。从流行病学角度,$\eta$可以被解读为城市的"登革热易感性指数":低$\eta$城市(如深圳)具有高效的蚊媒防控和较低的环境暴露风险,而高$\eta$城市(如东莞、潮州)则面临更大的输入性传播压力或更高的蚊媒基线密度。值得注意的是,$\eta$并非一个可直接测量的物理量,而是模型框架内对多种非气候因素综合效应的参数化表达。未来研究可以尝试将$\eta$与可观测的城市特征(如人口密度、绿化覆盖率、BI均值等)建立回归关系,从而实现$\eta$的预测性估计,进一步扩展模型的适用范围。

\textbf{时间泛化能力与COVID影响。}外部时间验证(2020--2026)的结果为模型的时间稳定性提供了强有力的证据。$\beta'(T)$与BI的季节性相关$r = 0.920$表明,Bri\`{e}re函数捕捉的温度--传播率关系在训练期之外至少7年内仍然成立。这一时间稳定性的物理基础是蚊虫温度生理特性的保守性——在没有重大物种演化或入侵事件的情况下,温度响应曲线不会发生显著变化\cite{mordecai2019}。

COVID-19大流行对验证结果的影响值得深入讨论。2021年$r = 0.620$是6年中最低的,恰好对应COVID严格管控期。大规模社区消杀、居民出行限制和环境整治等措施可能通过减少蚊媒孳生环境而压低了BI水平,破坏了温度与蚊媒密度之间的正常关联。然而,即使在这一极端干扰下,$r$仍为正值且达到中等水平,说明温度驱动的基本季节性模式并未被完全消除。2022年管控放松后$r$立即恢复至0.967,进一步证实了COVID效应的暂时性和温度驱动的根本性。

\textbf{排名vs.量级的方法论意义。}本研究采用的"排名优先"评估策略具有深层的方法论考量。在多城市比较中,排名指标(Spearman~$\rho$)相比绝对误差指标(如RMSE、WAPE)具有多重优势:(1)对量级缩放不敏感,避免了因城市间报告率差异导致的评估偏差;(2)对异常值鲁棒,不会因个别极端月份的预测偏差而大幅波动;(3)更贴近公共卫生决策需求——资源分配通常基于风险排名而非绝对数值。2014年度排名$\rho = 0.947$的结果表明,即使月度$\rho$的均值仅为0.531,模型在年度尺度上的排名能力仍然极为出色。这种"月度中等、年度优秀"的表现模式说明,月度噪声在年度聚合后被有效平滑,温度驱动的长期趋势得以凸显。

\textbf{与Li等\cite{li2019pnas}方法的比较。}Li等在2019年PNAS发表的研究采用样条函数拟合$\beta(T)$,在中国多城市登革热预测中取得了重要成果。与之相比,本方法具有以下特点:(1)Bri\`{e}re函数仅含3个参数,远少于样条函数的节点数,模型更为简约且不易过拟合;(2)Bri\`{e}re参数具有明确的昆虫学含义(发育零点、热致死温度),可解释性更强;(3)Bri\`{e}re函数为闭合解析式,可直接应用于任何有温度数据的城市,无需重新拟合样条节点;(4)本方法通过PySR进一步发现了融合多气象要素的公式$\hat{M}$,提供了超越纯温度模型的可能性。当然,样条方法在灵活性上具有优势,能够捕捉Bri\`{e}re函数无法表达的非标准温度响应形态。两种方法各有所长,可视为互补而非替代关系。

\textbf{低发病城市的统计学局限。}清远、肇庆、珠海等城市$\rho < 0.4$的结果需要审慎解读。如前所述,这些城市84\%的月份报告零病例,零膨胀数据结构从根本上限制了Spearman~$\rho$的理论上限。这并非本模型特有的问题,而是所有基于连续值的传染病预测模型在低发病地区面临的共同挑战\cite{cheng2016}。从实际应用角度看,低发病城市的预测需求与高发病城市不同:前者更关注"是否会发生暴发"(二分类问题),后者更关注"暴发规模有多大"(回归问题)。未来可以考虑在SEIR框架外层嵌套一个暴发概率模型(如logistic回归),将预测问题分解为"是否暴发"和"暴发规模"两个子问题,分别优化。

\textbf{方法的局限性与展望。}本章的迁移方法存在以下局限:(1)$\eta$的校准需要历史病例数据,无法应用于完全无历史数据的新城市;(2)ratio scaling假设量级偏差在时间上恒定,可能无法适应长期趋势变化;(3)SEIR模型未显式考虑人口流动和空间耦合效应\cite{xu2020},可能低估了城市间传播的贡献。未来研究可以从以下方向改进:建立$\eta$与城市特征的回归模型以实现零样本迁移;引入时变ratio scaling以适应长期趋势;在SEIR框架中加入空间耦合项以捕捉城市间传播动态。
