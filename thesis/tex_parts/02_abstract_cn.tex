
\newpage
\begin{center}
{\sffamily\zihao{3} 摘\quad 要}
\end{center}
\addcontentsline{toc}{section}{摘要}

登革热是全球最严重的蚊媒传染病之一,中国南方尤其广东省是国内最主要的流行区域。理解气候因素如何驱动登革热传播效率,对于建立早期预警系统和制定精准防控策略具有重要意义。然而,传统机制模型往往依赖先验假设固定传播率函数形式,难以从数据中自动发现最优的气候--传播率关系;纯数据驱动的机器学习方法则存在可解释性不足和小样本坍缩的瓶颈。

本文提出一种"物理先验耦合SEIR动力学+符号回归蚊媒公式+多城市联合校准"的混合建模框架,旨在兼顾机制可解释性与空间泛化能力。该框架包含三个核心阶段:(1)~利用神经网络学习气候变量(温度$T$、降水$R$、相对湿度$H$)到蚊媒密度的映射,并通过符号回归蒸馏为可解释的蚊媒密度公式$\hat{M}$;(2)~采用Bri\`{e}re温度响应函数$\beta'(T)$作为传播率的物理先验,耦合SEIR仓室模型进行病例预测;(3)~通过共享物理参数+逐城市输入率校准的策略,实现16城市的空间迁移。

研究分为两个部分。第一部分以广州市为核心开展单城市机制发现。首先以广东省8个有布雷图指数(BI)监测数据的城市为训练集(306个样本),通过神经网络+PySR符号回归发现蚊媒密度公式$\hat{M} = \frac{7.36\sqrt{T}}{T_m} - \frac{12.80}{H} - \frac{64.08}{H_m + \sqrt{R_m + R}}$,包含温度、湿度和降水三个气候因素,留一城市交叉验证(LOCO CV)均值$R^2=0.144$。在此基础上,采用Bri\`{e}re函数$\beta'(T) = cT(T-T_{\min})\sqrt{T_{\max}-T}$建模传播率的温度依赖性,通过差分进化优化获得最优参数($T_{\min}=15.1^\circ$C, $T_{\max}=42.0^\circ$C, $T_{\text{opt}}=35.5^\circ$C)。广州拟合Spearman~$\rho=0.814$、对数$R^2=0.851$;留一年交叉验证(LOYO CV)爆发年均值$\rho=0.806$。同时,以纯数据驱动的神经网络$\beta$方法作为对比实验,发现其在180个月度样本(64\%为低/零病例月)条件下发生模式坍缩($\beta_{\text{NN}}$变异系数仅1.17\%),反衬了物理先验方法在小样本传染病建模中的必要性。

第二部分开展多城市空间迁移与独立时间外验证。通过共享Bri\`{e}re参数+逐城市$\eta$校准迁移至广东省16城,月度均值$\rho=0.531$、$R^2_{\log}=0.656$、WAPE$=0.827$;2014年度排名$\rho=0.947$。进一步利用2020--2026年广州新BI监测数据进行独立时间外验证,Bri\`{e}re~$\beta'(T)$与观测BI的Pearson~$r=0.782$、季节性相关$r=0.920$,证实模型的时间泛化能力。

本研究的主要创新包括:(1)~提出"Bri\`{e}re物理先验+PySR蚊媒公式+SEIR动力学"的可解释混合建模框架;(2)~通过共享物理参数+逐城市校准的两阶段策略实现16城空间迁移;(3)~利用独立时间外数据(2020--2026)验证模型泛化能力;(4)~系统对比物理先验与纯数据驱动方法,揭示小样本条件下物理先验的优势。

\vspace{1em}
\noindent {\sffamily 关键词:}登革热;SEIR模型;Bri\`{e}re函数;符号回归;蚊媒密度;广东省;多城市验证;时间外验证
