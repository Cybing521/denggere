
\newpage
\begin{center}
{\sffamily\zihao{3} 摘\quad 要}
\end{center}
\addcontentsline{toc}{section}{摘要}

登革热是全球最严重的蚊媒传染病之一,中国南方尤其广东省是国内最主要的流行区域。理解气候因素如何驱动蚊媒种群动态和登革热传播,对于建立早期预警系统和制定精准防控策略具有重要意义。然而,传统机制模型往往依赖先验假设固定传播率函数形式,难以从数据中自动发现最优的气候--传播率关系;纯数据驱动的机器学习方法则存在可解释性不足和小样本坍缩的瓶颈。

本文提出一种"符号回归蚊媒公式+Bri\`{e}re物理先验SEIR动力学+多城市联合校准"的混合建模框架,旨在兼顾机制可解释性与空间泛化能力。

研究分为三个部分。第一部分以广东省6个有布雷图指数(BI)监测数据的城市为训练集(218个样本),通过神经网络+PySR符号回归发现蚊媒密度公式$\hat{M} = \frac{7.36\sqrt{T}}{T_m} - \frac{12.80}{H} - \frac{64.08}{H_m + \sqrt{R_m + R}}$,包含温度、湿度和降水三个气候因素,留一城市交叉验证(LOCO CV)均值Pearson~$r=0.493$,6个城市中5个$r>0.5$。

第二部分以广州市为核心(2005--2019年月度数据),采用Bri\`{e}re函数$\beta'(T) = cT(T-T_{\min})\sqrt{T_{\max}-T}$建模传播率的温度依赖性,耦合SEIR仓室模型进行病例预测。通过差分进化优化获得最优参数($T_{\min}=14.9^\circ$C, $T_{\max}=42.0^\circ$C, $T_{\text{opt}}=35.5^\circ$C),广州拟合Spearman~$\rho=0.813$、对数$R^2=0.855$;留一年交叉验证(LOYO CV)爆发年均值$\rho=0.798$。随后通过共享Bri\`{e}re参数+逐城市$\eta$校准迁移至16城,月度均值$\rho=0.531$、$R^2_{\log}=0.656$;2014年度排名$\rho=0.938$。利用2020--2026年广州新BI监测数据进行独立时间外验证,Bri\`{e}re~$\beta'(T)$与观测BI的Pearson~$r=0.782$、季节性相关$r=0.919$,证实模型的时间泛化能力。

第三部分以纯数据驱动的神经网络$\beta$方法作为对比实验,发现其在180个月度样本(64\%为低/零病例月)条件下发生模式坍缩($\beta_{\text{NN}}$变异系数仅1.17\%),反衬了物理先验方法在小样本传染病建模中的必要性。

本研究的主要创新包括:(1)~提出"PySR蚊媒公式+Bri\`{e}re物理先验+SEIR动力学"的可解释混合建模框架;(2)~通过共享物理参数+逐城市校准的两阶段策略实现16城空间迁移;(3)~利用独立时间外数据(2020--2026)验证模型泛化能力;(4)~系统对比物理先验与纯数据驱动方法,揭示小样本条件下物理先验的优势。

\vspace{1em}
\noindent {\sffamily 关键词:}登革热;SEIR模型;Bri\`{e}re函数;符号回归;蚊媒密度;广东省;多城市验证;时间外验证
