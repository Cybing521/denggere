%% ==================================================================
\section{第三部分:物理先验与数据驱动方法对比}
\label{sec:part3}
%% ==================================================================

\subsection{引言}

前两部分分别建立了气候驱动的蚊媒密度公式和基于Bri\`{e}re物理先验的SEIR动力学模型。一个自然的问题是:物理先验是否真的必要?如果用纯数据驱动的神经网络直接学习传播率$\beta$与气象变量的映射关系,能否达到同等甚至更好的效果?

本章设计了一组严格的对比实验,将Bri\`{e}re物理先验方法与纯数据驱动的NN方法置于完全相同的数据条件和评估框架下进行比较,以回答上述问题。

\subsection{方法}

\subsubsection{$\beta$反演}

利用SEIR模型,通过二分法(bisection method)从观测病例数反演每月的有效传播率$\beta_t^{\text{obs}}$。具体而言,对于每个月$t$,在给定$S_t$、$I_t$和$N_h$的条件下,搜索使得模型预测的新增病例数等于观测值$C_t$的$\beta$值。这一反演过程将动力学信息压缩为时间序列$\{\beta_t^{\text{obs}}\}_{t=1}^{180}$,为后续的神经网络学习提供监督信号。

反演得到的$\beta^{\text{obs}}$分布呈极端右偏:180个月中约50\%的$\beta$反演值为零或接近零(对应低/零病例月),而少数暴发月的$\beta$值极高。这种高度偏斜的目标分布对数据驱动方法构成了严峻挑战。

\subsubsection{NN $\beta$预测模型}

构建三层MLP神经网络(32-16-1,ReLU激活),输入为7维向量——6个气象特征($T, H, R, \bar{T}, \bar{H}, \bar{R}$)加上月份编码($\sin(2\pi m/12)$),预测目标为反演得到的$\beta_t^{\text{obs}}$。训练配置:Adam优化器,学习率$5\times10^{-3}$,5000个epoch,损失函数为Huber损失与Pearson相关性损失的加权组合($\mathcal{L} = \mathcal{L}_{\text{Huber}} + 0.3 \times (1 - r)$)。

\subsubsection{评估框架}

两种方法在完全相同的条件下进行对比:(1)相同的训练数据(广州2005--2019年180个月度样本);(2)相同的评估指标(Spearman~$\rho$、对数$R^2$、WAPE);(3)相同的SEIR预测框架——NN预测的$\beta_{\text{NN}}(t)$替代Bri\`{e}re的$\beta'(T_t)$,其余模型结构不变。

\subsection{结果}

\subsubsection{模式坍缩现象}

表\ref{tab:briere-vs-nn}对比了两种方法的预测性能。Bri\`{e}re方法在所有指标上均显著优于NN方法。

\begin{table}[H]
\centering
\caption{Bri\`{e}re物理先验 vs NN纯数据驱动方法对比(广州2005--2019)}
\label{tab:briere-vs-nn}
\begin{tabular}{lcc}
\toprule
指标 & Bri\`{e}re & NN \\
\midrule
Spearman $\rho$ & \textbf{0.814} & 0.706 \\
对数 $R^2$ & \textbf{0.851} & 0.230 \\
WAPE & \textbf{0.827} & 1.452 \\
参数数量 & 4 & $\sim$600 \\
\bottomrule
\end{tabular}
\end{table}

更关键的是,NN方法出现了严重的\textbf{模式坍缩}(mode collapse)现象:$\beta_{\text{NN}}$的变异系数(CV)仅为1.17\%,即NN对所有月份预测了几乎相同的$\beta$值,实质上退化为常数预测。相比之下,Bri\`{e}re函数$\beta'(T)$的变异系数为68.3\%,能够充分捕捉传播率的季节性波动。

\subsubsection{坍缩机制分析}

模式坍缩的根本原因在于目标分布的极端偏斜性与损失函数的交互作用:

\begin{enumerate}[leftmargin=2em]
\item \textbf{偏斜目标分布}:180个月中64.4\%为低/零病例月,对应的$\beta^{\text{obs}}$接近零。在Huber损失下,预测所有样本的均值是最小化损失的"安全策略"。
\item \textbf{相关性损失失效}:一旦NN的预测方差趋近于零,Pearson相关性损失项中的标准差分母接近零,导致梯度爆炸或数值不稳定,相关性惩罚实质上失效。
\item \textbf{自我强化陷阱}:预测方差减小$\to$相关性损失失效$\to$仅Huber损失主导$\to$预测进一步趋向均值$\to$方差进一步减小,形成正反馈循环。
\end{enumerate}

这一坍缩现象并非NN架构或超参数选择的偶然结果,而是小样本、高度偏斜分布条件下数据驱动方法的系统性缺陷。Bri\`{e}re函数之所以能避免坍缩,是因为其函数形态被物理先验(不对称单峰温度响应)强约束,仅需估计3个参数,不存在退化为常数的自由度。

\subsection{讨论}

本章的对比实验揭示了一个在传染病建模中具有普遍意义的现象:当训练数据呈极端偏斜分布(大量零/低值样本+少量高值样本)且样本量有限时,纯数据驱动方法容易发生模式坍缩,而物理先验通过约束模型的函数形态,能够有效避免这一问题。

这一发现的方法论意义在于:在传染病建模中,物理先验的价值不仅在于提供可解释性,更在于提供正则化——它将模型的搜索空间从高维参数空间压缩到低维物理参数空间,从而在小样本条件下获得更稳健的估计。这一结论对于其他面临类似数据特征(小样本、零膨胀、高偏斜)的传染病建模问题具有参考价值。

将Bri\`{e}re方法和NN方法置于完全相同的数据条件和评估框架下进行对比,为传染病建模中的方法选择提供了可复制的实证范式。未来的研究可以在不同传染病、不同地理区域和不同数据条件下重复这一对比实验,系统评估物理先验的适用边界。
