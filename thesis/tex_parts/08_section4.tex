%% ==================================================================
\section{总结与展望}
\label{sec:conclusion}
%% ==================================================================

\subsection{主要结论}

本文围绕"如何从数据中发现气候驱动登革热传播效率的数学规律"这一核心问题,提出并验证了"Bri\`{e}re物理先验+PySR蚊媒公式+SEIR动力学"混合建模框架。通过广州单城市机制发现、广东省16城市空间迁移验证以及2020--2026年独立时间外验证三个层次的系统研究,得到以下主要结论:

\begin{enumerate}[leftmargin=2em]
\item \textbf{Bri\`{e}re物理先验显著优于纯数据驱动方法}。Bri\`{e}re温度响应函数$\beta'(T) = cT(T-T_{\min})\sqrt{T_{\max}-T}$仅用3个物理参数即达到广州Spearman~$\rho=0.814$、对数$R^2=0.851$,显著优于7维输入的神经网络方法($\rho=0.706$、$R^2_{\log}=0.230$)。纯数据驱动的NN方法在180个月度样本(其中64.4\%为低/零病例月)条件下发生严重的模式坍缩,$\beta_{\text{NN}}$的变异系数仅为1.17\%,实质上退化为常数预测。这一结果有力地证实了在小样本传染病建模中,利用昆虫学物理先验约束模型函数形态比增加数据驱动模型的容量更为有效。坍缩的根本原因在于:高度偏斜的目标分布(50\%的$\beta$反演值为零)使得预测均值成为Huber损失下的最优策略,而一旦预测方差趋近于零,相关性损失项因标准差接近零而失效,形成自我强化的坍缩陷阱。

\item \textbf{符号回归发现简洁且可解释的蚊媒密度公式}。PySR从8城306个样本中蒸馏出蚊媒密度公式$\hat{M} = \frac{7.36\sqrt{T}}{T_m} - \frac{12.80}{H} - \frac{64.08}{H_m + \sqrt{R_m+R}}$,包含温度、湿度和降水三个气候因素,各项具有明确的生态学含义。该公式以城市内归一化BI($\text{BI}_{\text{rel}} = \text{BI}/\overline{\text{BI}}_c$)为建模目标,使模型聚焦于气候驱动的蚊媒密度季节性相对变化,而非受非气候因素影响的绝对水平,LOCO CV均值$R^2=0.144$。

\item \textbf{两阶段迁移策略实现16城空间泛化}。共享Bri\`{e}re物理参数($c, T_{\min}, T_{\max}$)+逐城市输入率$\eta$校准的策略,在16城月度预测中达到均值$\rho=0.531$、$R^2_{\log}=0.656$、WAPE$=0.827$。2014年度城市排名Spearman~$\rho=0.947$,表明模型能准确捕捉城市间的相对风险等级。逐城市$\eta$校准和量级缩放使WAPE从0.909降至0.827(改善9.0\%),WAPE$>1.0$的城市从4个减少到2个。该策略的理论基础在于:Bri\`{e}re参数反映蚊虫物种的温度生理特性(跨城市共享),而$\eta$吸收了城市间人口规模、城市化水平、蚊媒基线密度和防控力度等非气候因素的差异。

\item \textbf{独立时间外验证证实模型的时间泛化能力}。模型在2005--2019年训练后,在2020--2026年独立BI监测数据上,Bri\`{e}re~$\beta'(T)$与观测BI的整体Pearson~$r=0.782$,季节性轮廓相关$r=0.920$。逐年验证中4/6年$r>0.8$(2020: $r=0.965$, 2022: $r=0.967$, 2024: $r=0.926$, 2023: $r=0.844$),证明Bri\`{e}re函数捕捉的温度--传播率关系具有时间稳定性,不是对历史数据的过拟合。2021年($r=0.620$)和2025年($r=0.655$)相关性偏低,可能与COVID-19防控措施对蚊媒监测和人群活动的影响有关。

\item \textbf{Bri\`{e}re参数具有明确的生物学意义}。优化得到的$T_{\min}=15.1^\circ$C与文献报道的白纹伊蚊活动温度下限(14--16$^\circ$C)高度一致\cite{brady2013},$T_{\text{opt}}=35.5^\circ$C反映了高温条件下病毒外潜伏期缩短与蚊虫叮咬频率增加的综合效应。这些参数可与实验室数据交叉验证,增强了模型的可信度和可解释性。
\end{enumerate}

\subsection{创新点总结}

本研究在方法论、建模策略和验证体系三个层面做出了创新贡献:

\begin{enumerate}[leftmargin=2em]
\item \textbf{物理先验混合建模框架}。提出"Bri\`{e}re物理先验+PySR符号蒸馏+SEIR动力学"的可解释混合框架。与Li等\cite{li2019pnas}的样条$\beta(T)$方法相比,本框架输出的Bri\`{e}re函数是闭合公式,参数具有明确的昆虫学含义,可直接迁移至任何有温度数据的城市。与Zhang等\cite{zhang2024plos}的纯符号回归方法相比,本框架通过NN预训练降低了符号搜索难度,同时利用Bri\`{e}re物理先验避免了在传播率建模中的过拟合风险。

\item \textbf{两阶段空间迁移策略}。共享物理参数(蚊虫温度生理特性)+逐城市校准(非气候因素差异)的两阶段策略,在保持物理一致性的同时适应城市间异质性。该策略避免了全局统一参数对低发病城市的过度预测,也避免了完全独立拟合因数据不足导致的过拟合。

\item \textbf{独立时间外验证}。首次利用2020--2026年新BI监测数据对登革热传播率模型进行独立时间外验证。与常见的交叉验证(时间上有重叠)不同,本验证使用的数据与训练数据完全不重叠,且跨越了COVID-19疫情这一重大外部冲击,是对模型时间泛化能力的严格检验。

\item \textbf{物理先验vs.数据驱动的实证对比}。系统揭示了NN在小样本、高度偏斜分布条件下的模式坍缩现象及其机制,为传染病建模中的方法选择提供了实证依据。这一发现对于其他面临类似数据特征(小样本、零膨胀、高偏斜)的传染病建模问题具有普遍参考价值。
\end{enumerate}

\subsection{局限性分析}

尽管本研究取得了上述成果,仍存在以下局限性:

\begin{enumerate}[leftmargin=2em]
\item \textbf{蚊媒密度数据的稀缺性}。布雷图指数(BI)数据仅8个城市可用,且各城市的监测时间范围和频率不一致(广州87个月、惠州仅9个月)。新BI数据(2020--2026)仅覆盖广州一个城市。蚊媒密度数据的稀缺性限制了蚊媒公式的精度和泛化能力。PySR公式在真实数据上$R^2=0.205$,LOCO CV均值$R^2=0.144$,仍有较大提升空间。未来可考虑利用遥感数据(如归一化植被指数NDVI、地表水面积指数等)构建空间连续的蚊媒密度代理指标,或利用公民科学(Citizen Science)平台收集更广泛的蚊媒监测数据。

\item \textbf{低发病城市预测受限}。清远、肇庆、珠海等城市84\%的月份为零病例,导致连续值预测模型的Spearman~$\rho$天然受限($\rho<0.4$)。这不是模型的缺陷,而是零膨胀数据的统计特性——当大量观测值为零时,任何连续值预测的排名相关性都会被大量并列零值稀释。未来可引入零膨胀模型(如零膨胀泊松回归、Hurdle模型)或两阶段预测框架(先预测是否爆发,再预测爆发规模),以更好地处理低发病城市的预测问题。值得注意的是,这些城市的$R^2_{\log}$仍为正值(0.43--0.69),说明在有病例的月份,模型的量级预测是合理的。

\item \textbf{人口动态未建模}。本研究使用固定的中点人口数据(如广州$N_h = 1.426\times10^7$),未考虑人口的年际变化和季节性流动。广东省作为中国最大的流动人口目的地,春节前后的大规模人口流动可能显著影响登革热的传播动态。未来可引入逐年人口数据和人口流动矩阵(如基于手机信令数据的人口流动估计),以更准确地建模易感人群池的动态变化。

\item \textbf{空间耦合缺失}。本研究将16个城市视为独立系统,未考虑城市间的病例输入和蚊媒扩散。实际上,广东省珠三角地区城市间交通便利、人员流动频繁,登革热的跨城市传播是常见现象\cite{cheng2016}。未来可引入元群落(Metapopulation)结构或引力模型描述城市间的传播网络,将空间耦合效应纳入模型框架。

\item \textbf{气候变量的局限}。本研究仅使用月度平均温度、总降水量和平均相对湿度三个气候变量,未考虑温度日较差(DTR)、极端温度事件、降水时间分布等可能影响蚊媒活动的细粒度气候特征。此外,月度分辨率可能无法捕捉短期气候波动对蚊媒密度的即时影响。未来可尝试引入更高时间分辨率(如周度或旬度)的气候数据,以及更丰富的气候特征(如热浪天数、连续无雨天数等)。

\item \textbf{Bri\`{e}re函数的简化假设}。Bri\`{e}re函数假设传播率仅依赖于温度,忽略了湿度和降水对传播率的直接影响。虽然本研究通过蚊媒密度公式$\hat{M}$间接引入了湿度和降水的效应,但$\hat{M}$的精度有限($R^2=0.206$),可能未能充分捕捉这些因素的贡献。未来可考虑扩展Bri\`{e}re函数为多变量形式,或在SEIR框架中引入额外的气候调制项。

\item \textbf{模型不确定性量化不足}。本研究采用点估计方法(差分进化优化),未提供参数的置信区间和预测的不确定性范围。在公共卫生决策中,了解预测的不确定性与预测值本身同样重要——例如,决策者需要知道"预测下月病例数为500例,95\%置信区间为200--1200例"而非仅仅"预测500例"。未来可引入贝叶斯推断框架(如MCMC采样或变分推断),对Bri\`{e}re参数和逐城市$\eta$进行后验分布估计,从而提供预测的概率区间。此外,集成多个不同初始条件的优化结果也可以提供一种简单的不确定性估计方法。

\item \textbf{时间分辨率的限制}。本研究采用月度分辨率,这在一定程度上平滑了登革热传播的短期动态。登革热的潜伏期约为4--14天,病毒在蚊体内的外潜伏期约为8--12天,这意味着从感染到发病的完整周期约为2--4周。月度分辨率可能无法充分捕捉这种周级别的传播动态,尤其是在暴发初期的快速增长阶段。然而,更高的时间分辨率(如周度或双周)对数据质量的要求也更高,且可能引入更多的报告延迟噪声。在数据质量允许的条件下,未来可尝试双周或周度分辨率的建模,以更精细地刻画传播动态。
\end{enumerate}

\subsection{未来展望}

基于本研究的成果和局限性,提出以下未来研究方向:

\begin{enumerate}[leftmargin=2em]
\item \textbf{气候变化情景预测}。Bri\`{e}re函数的参数$T_{\min}$和$T_{\max}$可直接用于评估气候变暖对登革热传播窗口的影响。例如,若未来广州冬季最低温度从当前的13$^\circ$C升至16$^\circ$C以上(超过$T_{\min}=15.1^\circ$C),则全年均可能存在传播风险,传播季节将从目前的4--11月扩展至全年。将Bri\`{e}re函数与CMIP6气候模型的温度预测耦合,可以定量评估不同排放情景下广东省登革热传播风险的时空变化趋势。

\item \textbf{实时预警系统}。本研究建立的Bri\`{e}re+SEIR框架可以作为实时预警系统的核心算法。输入实时气象观测数据,即可输出未来1--3个月的传播风险预测。结合逐城市$\eta$参数和量级校准因子,可以为广东省16个城市提供差异化的风险预警等级,支持公共卫生资源的精准分配。

\item \textbf{跨区域推广}。Bri\`{e}re函数的物理参数具有物种特异性而非地域特异性,理论上可以推广至中国其他登革热流行区域(如云南、浙江、福建等省份)以及东南亚国家。推广时仅需重新校准逐城市$\eta$参数和量级缩放因子,Bri\`{e}re物理参数可直接复用。值得注意的是,不同地区的主要传播媒介可能不同——广东省以白纹伊蚊为主,而东南亚热带地区以埃及伊蚊为主。两种蚊虫的温度响应特性存在差异(埃及伊蚊的最适温度略高于白纹伊蚊),因此跨物种推广时可能需要重新估计Bri\`{e}re参数。此外,不同地区的报告系统、诊断标准和就医行为差异也会影响$\eta$参数的校准,需要在推广过程中加以考虑。

\item \textbf{多病原体扩展}。本框架的"物理先验+符号蒸馏+动力学"范式原则上可应用于其他具有气候--传播耦合关系的蚊媒传染病(如寨卡病毒、基孔肯雅热、疟疾等),以及其他需要从数据中发现机制性规律的传染病建模问题。不同病原体可能需要不同的温度响应函数参数,但Bri\`{e}re函数的基本形式仍然适用。例如,疟疾的传播媒介按蚊(\textit{Anopheles})的温度响应特性与伊蚊存在显著差异,其最适传播温度约为25$^\circ$C,低于登革热的29$^\circ$C。通过重新估计Bri\`{e}re参数,本框架可以适应不同蚊媒传染病的建模需求。
\end{enumerate}

综合而言,本文提出的"物理先验+符号蒸馏+动力学"框架为传染病机制发现提供了兼顾数据适应性和物理可解释性的新路径。Bri\`{e}re函数的成功应用表明,在小样本条件下,利用昆虫学先验知识约束模型比增加数据驱动模型的容量更为有效。该框架的核心优势在于:物理先验提供了稳健的函数骨架,符号回归提供了可解释的辅助公式,SEIR动力学提供了流行病学的因果结构,三者的有机结合使得模型在有限数据条件下仍能实现可靠的预测和有意义的机制解释。我们期望本研究的方法论贡献能够为蚊媒传染病的精准防控和早期预警提供有价值的理论支撑和技术参考。
