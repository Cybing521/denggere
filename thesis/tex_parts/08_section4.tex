%% ==================================================================
\section{总结与展望}
\label{sec:conclusion}
%% ==================================================================

\subsection{主要结论}

本文围绕"如何从数据中发现气候驱动登革热传播的数学规律"这一核心问题,提出并验证了"PySR蚊媒公式+Bri\`{e}re物理先验+SEIR动力学"混合建模框架。通过气候驱动蚊媒密度公式发现(第一部分)、Bri\`{e}re-SEIR动力学建模与多城市验证(第二部分)、以及物理先验与数据驱动方法的系统对比(第三部分)三个层次的研究,得到以下主要结论:

\begin{enumerate}[leftmargin=2em]
\item \textbf{符号回归发现简洁且可解释的蚊媒密度公式}。PySR从6城218个样本中蒸馏出蚊媒密度公式$\hat{M} = \frac{7.36\sqrt{T}}{T_m} - \frac{12.80}{H} - \frac{64.08}{H_m + \sqrt{R_m+R}}$,包含温度、湿度和降水三个气候因素,各项具有明确的生态学含义。该公式以城市内归一化BI为建模目标,使模型聚焦于气候驱动的蚊媒密度季节性相对变化,LOCO CV均值Pearson~$r=0.597$,4/6城市$r>0.5$。

\item \textbf{Bri\`{e}re-SEIR模型在广州实现高精度拟合}。Bri\`{e}re温度响应函数仅用3个物理参数即达到广州Spearman~$\rho=0.813$、对数$R^2=0.855$。LOYO CV爆发年均值$\rho=0.798$。参数$T_{\min}=14.9^\circ$C与文献报道的白纹伊蚊活动温度下限(14--16$^\circ$C)高度一致\cite{brady2013},增强了模型的可信度。

\item \textbf{两阶段迁移策略实现16城空间泛化}。共享Bri\`{e}re物理参数+逐城市$\eta$校准的策略,在16城月度预测中达到均值$\rho=0.531$、$R^2_{\log}=0.656$。2014年度城市排名$\rho=0.938$,表明模型能准确捕捉城市间的相对风险等级。

\item \textbf{独立时间外验证证实模型的时间泛化能力}。在2020--2026年独立BI监测数据上,Bri\`{e}re~$\beta'(T)$与观测BI的Pearson~$r=0.782$,季节性轮廓相关$r=0.920$。逐年验证中4/6年$r>0.8$,证明模型不是对历史数据的过拟合。

\item \textbf{物理先验显著优于纯数据驱动方法}。纯数据驱动的NN方法在180个月度样本(64.4\%为低/零病例月)条件下发生严重的模式坍缩,$\beta_{\text{NN}}$变异系数仅1.17\%,实质上退化为常数预测。Bri\`{e}re方法在所有指标上均大幅领先($\rho$: 0.813 vs.\ 0.706,$R^2_{\log}$: 0.855 vs.\ 0.230),证实了在小样本传染病建模中物理先验的必要性。
\end{enumerate}

\subsection{创新点总结}

\begin{enumerate}[leftmargin=2em]
\item \textbf{符号回归发现可解释蚊媒密度公式}。PySR从6城BI数据中蒸馏出包含温度、湿度和降水三因素的蚊媒密度公式,各项具有明确的生态学含义,LOCO CV均值$r=0.597$。

\item \textbf{物理先验混合建模框架}。提出"Bri\`{e}re物理先验+PySR符号蒸馏+SEIR动力学"的可解释混合框架。与Li等\cite{li2019pnas}的样条$\beta(T)$方法相比,本框架输出闭合公式,参数具有明确的昆虫学含义。与Zhang等\cite{zhang2024plos}的纯符号回归方法相比,本框架通过NN预训练降低了符号搜索难度。

\item \textbf{两阶段空间迁移策略}。共享物理参数+逐城市校准的策略,在保持物理一致性的同时适应城市间异质性,避免了全局统一参数的过度预测和完全独立拟合的过拟合。

\item \textbf{独立时间外验证}。首次利用2020--2026年新BI监测数据对登革热传播率模型进行独立时间外验证,跨越了COVID-19疫情这一重大外部冲击,是对模型时间泛化能力的严格检验。

\item \textbf{物理先验vs.数据驱动的实证对比}。系统揭示了NN在小样本、高度偏斜分布条件下的模式坍缩现象及其机制,为传染病建模中的方法选择提供了实证依据。
\end{enumerate}

\subsection{局限性分析}

\begin{enumerate}[leftmargin=2em]
\item \textbf{蚊媒数据稀缺}。仅6个城市有BI监测数据(共218个样本),限制了蚊媒公式的训练数据量和跨城泛化评估的可靠性。PySR公式在真实数据上Pearson~$r=0.462$,仍有较大提升空间。

\item \textbf{低发病城市预测受限}。在年均病例数$<$50的城市中,模型预测性能显著下降($\rho$中位数0.35 vs.\ 高发城市0.65),因为低发病城市的传播主要由随机输入事件驱动,确定性气候模型难以捕捉。

\item \textbf{空间耦合缺失}。当前模型将各城市视为独立系统,未考虑城市间的人口流动和病例输入效应。2014年广州大暴发期间,周边城市的病例激增很可能部分源于广州的病例输出,而非仅由本地气候条件驱动。

\item \textbf{月度时间分辨率}。月度分辨率可能掩盖了周度或日度尺度上的传播动态细节。登革热的潜伏期约为4--10天,月度数据无法精确捕捉这一时间尺度的动态变化。

\item \textbf{气候数据空间分辨率}。使用气象站点数据而非空间分辨率更高的遥感数据,未能捕捉城市内部的温度空间异质性(如城市热岛效应)。
\end{enumerate}

\subsection{未来展望}

\begin{enumerate}[leftmargin=2em]
\item \textbf{遥感蚊媒密度估计}。利用卫星遥感数据(如NDVI植被指数、地表水体面积、夜间灯光等)构建蚊媒密度的空间代理指标,突破BI监测数据的地理覆盖限制。

\item \textbf{零膨胀模型}。针对低发病城市大量零病例月的特点,引入零膨胀泊松(ZIP)或零膨胀负二项(ZINB)模型,分别建模"是否发生传播"和"传播规模"两个过程。

\item \textbf{空间耦合网络模型}。在SEIR框架中引入城市间人口流动矩阵,建模病例的空间传播路径,提高对暴发扩散过程的预测能力。

\item \textbf{气候变化情景预测}。结合CMIP6气候模型的未来温度预测,评估不同排放情景下广东省登革热传播风险的时空变化趋势。

\item \textbf{多病原体扩展}。本框架的"物理先验+符号蒸馏+动力学"范式可应用于其他蚊媒传染病(寨卡、基孔肯雅热、疟疾等),通过重新估计Bri\`{e}re参数适应不同蚊媒传染病的建模需求。
\end{enumerate}

综合而言,本文提出的"物理先验+符号蒸馏+动力学"框架为传染病机制发现提供了兼顾数据适应性和物理可解释性的新路径。物理先验提供了稳健的函数骨架,符号回归提供了可解释的辅助公式,SEIR动力学提供了流行病学的因果结构,三者的有机结合使得模型在有限数据条件下仍能实现可靠的预测和有意义的机制解释。
