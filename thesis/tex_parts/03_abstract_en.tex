
\newpage
\begin{center}
{\sffamily\zihao{3} Abstract}
\end{center}
\addcontentsline{toc}{section}{Abstract}

Dengue fever is one of the most severe mosquito-borne infectious diseases globally. Southern China, especially Guangdong Province, is the primary endemic region in the country. Understanding how climatic factors drive mosquito population dynamics and dengue transmission is crucial for establishing early-warning systems and formulating targeted control strategies. However, traditional mechanistic models rely on \textit{a priori} assumptions to fix the functional form of the transmission rate, while purely data-driven machine learning approaches suffer from mode collapse under small-sample conditions.

This thesis proposes a hybrid modeling framework---``Symbolic Regression Mosquito Formula + Physics-Informed Bri\`{e}re SEIR Dynamics + Multi-City Joint Calibration''---that balances mechanistic interpretability with spatial generalizability.

The study is organized into three parts. Part~I uses 6~cities with Breteau Index (BI) monitoring data (218~samples) to discover the mosquito density formula $\hat{M} = \frac{7.36\sqrt{T}}{T_m} - \frac{12.80}{H} - \frac{64.08}{H_m + \sqrt{R_m + R}}$, incorporating temperature, humidity, and precipitation, through neural network + PySR symbolic regression, with leave-one-city-out cross-validation (LOCO~CV) mean Pearson~$r=0.493$ (5/6 cities $r>0.5$).

Part~II centers on Guangzhou (2005--2019 monthly data), employing the Bri\`{e}re function $\beta'(T) = cT(T-T_{\min})\sqrt{T_{\max}-T}$ to model the temperature dependence of the transmission rate, coupled with an SEIR compartmental model for case prediction. Differential evolution optimization yields optimal parameters ($T_{\min}=15.1^\circ$C, $T_{\max}=42.0^\circ$C, $T_{\text{opt}}=35.5^\circ$C), with Guangzhou fit Spearman~$\rho=0.814$ and log-scale $R^2=0.851$; leave-one-year-out CV (LOYO~CV) outbreak-year mean $\rho=0.806$. The model is then transferred to 16~cities via shared Bri\`{e}re parameters + per-city $\eta$ calibration, achieving monthly mean $\rho=0.531$, $R^2_{\log}=0.656$, and 2014 annual ranking $\rho=0.947$. Independent temporal-external validation using 2020--2026 Guangzhou BI data yields Bri\`{e}re $\beta'(T)$ vs.\ observed BI Pearson~$r=0.782$ and seasonal $r=0.920$, confirming temporal generalizability.

Part~III conducts a comparative experiment using a purely data-driven neural network $\beta$ approach, revealing mode collapse ($\beta_{\text{NN}}$ coefficient of variation only 1.17\%) under 180 monthly samples (64\% low/zero-case months), underscoring the necessity of physics-informed priors in small-sample infectious disease modeling.

Key innovations include: (1)~a ``PySR mosquito formula + Bri\`{e}re physical prior + SEIR dynamics'' interpretable hybrid framework; (2)~a two-stage spatial transfer strategy via shared physical parameters + per-city calibration across 16~cities; (3)~independent temporal-external validation using 2020--2026 data; (4)~systematic comparison of physics-informed vs.\ data-driven approaches, revealing the advantage of physical priors under small-sample conditions.

\vspace{1em}
\noindent \textbf{Keywords:} Dengue fever; SEIR model; Bri\`{e}re function; Symbolic regression; Mosquito density; Guangdong Province; Multi-city validation; Temporal-external validation
