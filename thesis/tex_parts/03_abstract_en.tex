
\newpage
\begin{center}
{\sffamily\zihao{3} Abstract}
\end{center}
\addcontentsline{toc}{section}{Abstract}

Dengue fever is one of the most severe mosquito-borne infectious diseases globally. Southern China, especially Guangdong Province, is the primary endemic region in the country. Understanding how climatic factors drive dengue transmission efficiency is crucial for establishing early-warning systems and formulating targeted control strategies. However, traditional mechanistic models rely on \textit{a priori} assumptions to fix the functional form of the transmission rate, while purely data-driven machine learning approaches suffer from mode collapse under small-sample conditions.

This thesis proposes a hybrid modeling framework---``Physics-Informed Bri\`{e}re Function + Symbolic Regression Mosquito Formula + Multi-City Joint Calibration''---that balances mechanistic interpretability with spatial generalizability. The framework consists of three core stages: (1)~learning the mapping from climate variables (temperature~$T$, precipitation~$R$, relative humidity~$H$) to mosquito density via a neural network, followed by symbolic regression distillation to obtain an interpretable mosquito density formula~$\hat{M}$; (2)~modeling the temperature dependence of the transmission rate using the Bri\`{e}re function $\beta'(T) = cT(T-T_{\min})\sqrt{T_{\max}-T}$, coupled with an SEIR compartmental model for case prediction; (3)~transferring to 16~cities via shared physical parameters and per-city input rate calibration.

The study is organized into two parts. Part~I focuses on single-city mechanism discovery centered on Guangzhou. First, 8~cities with Breteau Index (BI) monitoring data (306~samples) are used to discover the mosquito density formula $\hat{M} = \frac{7.36\sqrt{T}}{T_m} - \frac{12.80}{H} - \frac{64.08}{H_m + \sqrt{R_m + R}}$, incorporating temperature, humidity, and precipitation, through neural network + PySR symbolic regression, with leave-one-city-out cross-validation (LOCO~CV) mean $R^2=0.144$. The Bri\`{e}re function is then employed to model the temperature dependence of transmission rate. Differential evolution optimization yields optimal parameters ($T_{\min}=15.1^\circ$C, $T_{\max}=42.0^\circ$C, $T_{\text{opt}}=35.5^\circ$C). Guangzhou fitting achieves Spearman~$\rho=0.814$ and log-scale $R^2=0.851$; leave-one-year-out cross-validation (LOYO~CV) for outbreak years yields mean $\rho=0.806$. A purely data-driven neural network $\beta$ approach is presented as a comparative experiment, revealing mode collapse ($\beta_{\text{NN}}$ coefficient of variation = 1.17\%) under 180 monthly samples with 64\% low/zero-case months, highlighting the necessity of physics-informed approaches for small-sample infectious disease modeling.

Part~II performs multi-city spatial transfer and independent temporal-external validation. Transfer to 16~cities via shared Bri\`{e}re parameters + per-city $\eta$ calibration achieves monthly mean $\rho=0.531$, $R^2_{\log}=0.656$, WAPE${}=0.827$; 2014 annual ranking $\rho=0.947$. Independent temporal-external validation using new Guangzhou BI monitoring data from 2020--2026 shows that the Bri\`{e}re $\beta'(T)$ correlates with observed BI at Pearson~$r=0.782$ and seasonal $r=0.920$, confirming temporal generalizability.

Key innovations include: (1)~a ``Bri\`{e}re physics prior + PySR mosquito formula + SEIR dynamics'' interpretable hybrid framework; (2)~a two-stage transfer strategy with shared physical parameters and per-city calibration for 16-city spatial generalization; (3)~independent temporal-external validation using 2020--2026 data; (4)~systematic comparison between physics-informed and purely data-driven approaches, revealing the advantage of physical priors under small-sample conditions.

\vspace{1em}
\noindent \textbf{Keywords:} Dengue fever; SEIR model; Bri\`{e}re function; Symbolic regression; Mosquito density; Guangdong Province; Multi-city validation; Temporal-external validation
