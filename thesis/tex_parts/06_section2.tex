%% ==================================================================
\section{第一部分:气候驱动蚊媒密度公式发现}
\label{sec:part1}
%% ==================================================================

\subsection{引言}
\label{sec:p1-intro}

蚊媒密度是登革热传播动力学中的关键变量,直接决定了人群暴露于感染性叮咬的风险水平。然而,蚊媒密度的系统监测在大多数地区仍然缺乏——即使在广东省,也仅有少数城市开展了布雷图指数(BI)的常规监测,且监测频率和覆盖范围有限。因此,建立气候变量到蚊媒密度的定量映射关系,对于在缺乏直接监测数据的城市中估计蚊媒活动强度具有重要意义。

本章的目标是从广东省6个有BI监测数据的城市(共218个样本)中,通过"神经网络预训练+PySR符号回归蒸馏"的两阶段策略,发现一个可解释的蚊媒密度公式$\hat{M}$。该公式以月均温度$T$、相对湿度$H$和降水量$R$为输入,输出蚊媒密度的季节性相对变化,为后续SEIR动力学模型提供蚊媒密度输入。

%% ------------------------------------------------------------------
\subsection{数据与方法}
\label{sec:p1-data-method}
%% ------------------------------------------------------------------

\subsubsection{布雷图指数数据}
\label{sec:p1-bi-data}

蚊媒密度数据来源于广东省6个城市(广州、东莞、惠州、江门、茂名、汕头)的布雷图指数(BI)监测记录,共计218个有效月度样本。BI定义为每百户居民住宅中阳性容器(含伊蚊幼虫或蛹的积水容器)的数量,是世界卫生组织推荐的蚊媒密度标准化监测指标\cite{who2024}。各城市BI均值差异显著(表\ref{tab:bi-summary}),反映了不同城市在城市化水平、居住环境类型和防控力度等方面的系统性差异。图\ref{fig:bi-dist}展示了6城BI的分布特征和季节性模式。

\begin{figure}[H]
\centering
\includegraphics[width=0.95\textwidth]{../figures/fig0b_bi_distribution.png}
\caption{6城BI数据描述。左:各城市BI箱线图;右:BI月度季节性模式(均值$\pm$标准差)。}
\label{fig:bi-dist}
\end{figure}

\begin{table}[H]
\centering
\caption{6城市BI监测数据概况}
\label{tab:bi-summary}
\begin{tabular}{lcccc}
\toprule
城市 & 样本数 & BI均值 & BI标准差 & BI范围 \\
\midrule
东莞 & 24 & 2.6 & 1.3 & 0.2--5.4 \\
广州 & 87 & 8.7 & 7.8 & 0.1--38.5 \\
惠州 & 9 & 3.2 & 1.8 & 0.8--7.2 \\
江门 & 36 & 5.8 & 3.7 & 0.3--15.1 \\
茂名 & 18 & 2.8 & 1.6 & 0.4--7.5 \\
汕头 & 44 & 5.1 & 4.8 & 0.1--22.3 \\
\bottomrule
\end{tabular}
\end{table}

\subsubsection{气象数据}

月度气象数据来源于NOAA全球地面日观测数据集(GSOD),提取各城市对应站点的月均温度($T$,°C)、月均相对湿度($H$,\%)和月累计降水量($R$,mm)。为捕捉城市间气候基线差异,同时计算各城市的年均温度$T_m$、年均湿度$H_m$和年均降水量$R_m$作为代理特征。

\subsubsection{城市内归一化}
\label{sec:p1-normalization}

由于6城BI均值的最大/最小比达3.3倍,若直接以BI绝对值为建模目标,模型将不可避免地混淆气候效应与城市间的系统性尺度差异,导致跨城泛化困难。为此,采用城市内归一化策略:
\begin{equation}
\text{BI}_{\text{rel},i,t} = \frac{\text{BI}_{i,t}}{\overline{\text{BI}}_i}
\end{equation}
其中$\overline{\text{BI}}_i$为城市$i$的BI时间均值。归一化后,$\text{BI}_{\text{rel}}$反映的是各城市相对于自身基线水平的蚊媒密度季节性波动,消除了非气候因素导致的绝对水平差异,使模型聚焦于气候驱动的季节性变化。

建模目标设定为$y = \log(1 + \text{BI}_{\text{rel}})$,对数变换用于压缩BI的右偏分布,使其更接近正态分布,有利于回归模型的训练。

\subsubsection{神经网络预训练}
\label{sec:p1-nn}

第一阶段使用三层MLP神经网络(32-16-1,ReLU激活)学习气候变量到归一化蚊媒密度的映射。输入为6维向量:月度气象特征($T, H, R$)和城市年均气候特征($T_m, H_m, R_m$)。训练配置:Adam优化器,学习率$5\times10^{-3}$,5000个epoch,损失函数为Huber损失与Pearson相关性损失的加权组合($\mathcal{L} = \mathcal{L}_{\text{Huber}} + 0.3 \times (1 - r)$)。

NN的作用是作为"教师模型"——在连续的气候特征空间中学习一个平滑的映射函数,为后续符号回归提供高质量的蒸馏数据。

\subsubsection{PySR符号回归}
\label{sec:p1-pysr}

第二阶段利用训练好的NN在$6 \times 10{,}000$个网格点上生成蒸馏数据,然后运行PySR符号回归\cite{cranmer2023}搜索可解释的闭合公式。搜索空间包括基本算术运算($+, -, \times, \div$)、平方根和三角函数。PySR采用多种群遗传编程算法,在公式复杂度和拟合精度之间进行Pareto前沿搜索。

公式选择综合考虑拟合精度和物理可解释性:在Pareto前沿上,优先选择包含多个气候因素且各项具有明确生态学含义的公式。最终选定的公式在原始BI数据上进行独立验证,并通过留一城市交叉验证(LOCO CV)评估跨城泛化能力。

%% ------------------------------------------------------------------
\subsection{结果}
\label{sec:p1-results}
%% ------------------------------------------------------------------

\subsubsection{神经网络拟合}

神经网络在218个BI样本上的训练指标为:Pearson~$r = 0.503$,Spearman~$\rho = 0.518$。NN成功学习了温度、湿度和降水对蚊媒密度的非线性联合效应,为后续符号回归蒸馏提供了可靠的教师信号。

\subsubsection{符号回归公式发现}

PySR在Pareto前沿上输出了一系列候选公式,复杂度从3到25不等。在综合考虑精度和物理可解释性后,选定包含温度、湿度和降水三个气候因素的公式(复杂度19):
\begin{equation}
\label{eq:mhat}
\hat{M} = \frac{7.36\sqrt{T}}{T_m} - \frac{12.80}{H} - \frac{64.08}{H_m + \sqrt{R_m + R}}
\end{equation}
其中$T$为月均温度,$T_m$为城市年均温度,$H$为月均相对湿度,$H_m$为城市年均湿度,$R$为月降水量,$R_m$为城市年均降水量。该公式在真实数据上Pearson~$r = 0.462$,Spearman~$\rho = 0.455$。

公式包含三个物理含义明确的项:
\begin{itemize}[leftmargin=2em]
\item $\sqrt{T}/T_m$:温度对蚊虫活动的正向驱动,平方根形式体现了高温区域的边际递减效应;
\item $-1/H$:高湿度有利于蚊虫存活($H$越大,负贡献越小);
\item $-1/(H_m + \sqrt{R_m + R})$:降水对蚊虫孳生地的调节作用——适量降水增加积水容器,但该效应受城市基线湿度$H_m$的调制。
\end{itemize}

与NN教师模型相比,PySR公式以极小的精度损失($\Delta r = 0.041$)换取了完全的可解释性和零推理成本。

\begin{figure}[H]
\centering
\includegraphics[width=0.95\textwidth]{../figures/fig1_formula_vs_obs_8cities.png}
\caption{三变量蚊媒公式$\hat{M}$预测值与6城观测BI的散点对比。虚线为$y=x$参考线。}
\label{fig:formula-scatter}
\end{figure}

\subsubsection{响应曲面分析}

图\ref{fig:response-surface}展示了公式$\hat{M}$在三个气候因素两两组合下的响应曲面。温度是最主要的驱动因素,$\hat{M}$随温度升高而单调递增;湿度的效应次之,高湿度条件下蚊媒密度更高;降水的效应相对较弱,主要通过与基线湿度的交互作用发挥调节功能。

\begin{figure}[H]
\centering
\includegraphics[width=0.95\textwidth]{../figures/fig3_response_surface.png}
\caption{蚊媒公式$\hat{M}$的响应曲面。左:温度$\times$湿度($R=150$\,mm);中:温度$\times$降水($H=77\%$);右:湿度$\times$降水($T=25^\circ$C)。}
\label{fig:response-surface}
\end{figure}

\subsubsection{留一城市交叉验证}

留一城市交叉验证(LOCO CV)结果验证了蚊媒公式的跨城泛化能力。LOCO CV的均值Pearson~$r = 0.493$,6个城市中有4个$r > 0.5$(表\ref{tab:loco-cv}),表明公式在未见过的城市上能够较好地捕捉蚊媒密度的季节性变化趋势。Pearson~$r$衡量的是公式预测值与观测值的线性相关性(即趋势一致性),这正是蚊媒公式在下游SEIR模型中的核心功能——绝对尺度由逐城市$\eta$参数校准,公式只需提供正确的季节性变化模式。

\begin{table}[H]
\centering
\caption{留一城市交叉验证(LOCO CV)结果}
\label{tab:loco-cv}
\begin{tabular}{lcccc}
\toprule
测试城市 & $n$ & Pearson $r$ & Spearman $\rho$ & RMSE \\
\midrule
东莞 & 24 & 0.754 & 0.669 & 0.172 \\
广州 & 87 & 0.505 & 0.536 & 0.356 \\
惠州 & 9 & 0.755 & 0.533 & 0.096 \\
江门 & 36 & 0.559 & 0.524 & 0.240 \\
茂名 & 18 & 0.467 & 0.416 & 0.118 \\
汕头 & 44 & 0.542 & 0.381 & 0.247 \\
\midrule
\textbf{均值} & & \textbf{0.597} & \textbf{0.510} & \textbf{0.205} \\
\bottomrule
\end{tabular}
\end{table}

作为消融实验,若去除城市内归一化而直接以BI绝对值为目标,LOCO CV均值$r$降至$0.31$,$r>0.5$的城市从4个减少到1个,表明归一化策略对于跨城泛化不可或缺。

%% ------------------------------------------------------------------
\subsection{讨论}
\label{sec:p1-discussion}
%% ------------------------------------------------------------------

PySR发现的蚊媒公式$\hat{M}$包含温度、湿度和降水三个气候因素,各项的物理含义与已有的蚊媒生态学知识一致。$\sqrt{T}/T_m$项反映了温度对蚊虫发育和活动的正向驱动,平方根形式暗示高温区域的边际效应递减,这与Mordecai等\cite{mordecai2019}报道的蚊虫活动对温度的非线性响应一致。$-1/H$项表明湿度对蚊虫存活的正向作用,与Xu等\cite{xu2020}在广州的研究结论吻合。降水项$-1/(H_m + \sqrt{R_m + R})$的结构较为复杂,反映了降水通过增加积水孳生地促进蚊虫繁殖的效应,但该效应受城市基线湿度的调制——在本底湿度较高的城市,额外降水的边际贡献较小。公式的Pearson~$r = 0.462$表明气候因素能够解释蚊媒密度季节性变化的主要趋势,其余变异来源于城市化水平、防控措施、容器类型等未被公式捕捉的非气候因素。

城市内归一化策略的必要性源于蚊媒密度建模中一个常被忽视的问题:不同城市的BI绝对水平差异主要反映的是非气候因素的系统性差异,而非气候驱动的蚊媒密度变化。归一化到城市内相对变化后,模型得以聚焦于气候因素的季节性调控效应,这是实现跨城泛化的关键前提。消融实验表明,去除归一化后LOCO CV均值$r$从0.597降至0.31,进一步证实了这一判断。

"NN预训练+PySR蒸馏"的两阶段策略相比直接对原始数据运行符号回归具有两个优势:(1)NN在连续特征空间中学习的平滑映射函数为PySR提供了高质量的蒸馏数据,降低了符号搜索的噪声干扰;(2)通过在大量网格点上生成蒸馏数据,有效扩充了PySR的训练样本量,缓解了原始BI数据稀缺(仅218个样本)对符号搜索的限制。这一策略可推广至其他数据稀缺的生态学建模场景。
