%% ==================================================================
\section{第一部分:单城市机制发现(广州)}
\label{sec:part1}
%% ==================================================================

\subsection{引言}
\label{sec:p1-intro}

登革热传播是一个涉及病毒、蚊媒、人群和环境的复杂多因素过程。在构建跨城市的空间迁移模型之前,有必要首先在单一城市中深入理解传播机制,建立可靠的基础模型,再将其推广至更大的空间尺度。本章选择广州市作为核心研究案例,原因如下:第一,广州是中国大陆登革热负担最重的城市,2005--2019年累计报告病例数居全国首位,其中2014年单年暴发超过37,000例\cite{cheng2016},提供了丰富的疫情动态信息;第二,广州地处亚热带季风气候区,温度、降水和湿度的季节性变化显著,为研究气候驱动的传播机制提供了理想的自然实验条件;第三,广州及周边城市拥有相对完善的布雷图指数(BI)监测数据,为蚊媒密度建模提供了不可或缺的训练数据。

本章的核心任务包括三个层面:(1)从气候数据中发现可解释的蚊媒密度公式,建立气候变量到蚊媒活动强度的定量映射;(2)利用Bri\`{e}re物理先验函数建模传播率的温度依赖性,构建离散时间SEIR预测框架;(3)通过与纯数据驱动的神经网络方法进行系统对比,揭示物理先验在小样本传染病建模中的必要性。这三个层面相互关联、逐层递进,共同构成了从数据到机制、从拟合到验证的完整研究链条。

%% ------------------------------------------------------------------
\subsection{数据材料与方法}
\label{sec:p1-data-method}
%% ------------------------------------------------------------------

\subsubsection{研究区域与数据来源}
\label{sec:p1-study-area}

广州市(23.13°N, 113.26°E)位于广东省中南部、珠江三角洲北缘,属南亚热带季风气候,年均气温21.4--22.6°C,年均降水量1600--1900\,mm,年均相对湿度约77\%。全市常住人口约1,426万人($N_h = 1.426 \times 10^7$),城市化率超过86\%,人口密度高、建成区面积大,为伊蚊孳生和登革热传播提供了有利的生态环境。广州市白纹伊蚊(\textit{Aedes albopictus})为优势蚊种,全年活动期约为4--11月,高峰期集中在6--9月\cite{cheng2016}。

本研究使用的数据来源包括以下四个方面:

\textbf{(1)登革热病例数据。}广州市2005--2019年月度登革热报告病例数据来源于中国疾病预防控制中心传染病报告信息系统。该数据集共包含180个月度观测值,呈现出极端的右偏分布特征:约25\%的月份报告零病例,64.4\%的月份报告病例数$\leq$5例,而2014年暴发年全年累计报告37,382例,占15年总病例数的绝大部分。其余年份的年度病例数在9例(2005年)至1,690例(2013年)之间波动,年际变异系数极大。这种高度偏斜的分布特征对建模方法提出了严峻挑战——模型必须同时捕捉大量低发病月份的基线水平和偶发暴发年份的极端峰值。

\textbf{(2)气象数据。}月度气象数据来源于美国国家海洋和大气管理局(NOAA)全球地面日观测数据集(Global Surface Summary of the Day, GSOD),选取广州站(站号59287)。提取的气象变量包括月均温度($T$,°C)、月均相对湿度($H$,\%)和月累计降水量($R$,mm)。GSOD数据经过质量控制,缺测率低于2\%,缺测值采用相邻月份线性插值填补。

\textbf{(3)布雷图指数数据。}蚊媒密度数据来源于广东省8个城市(广州、深圳、珠海、佛山、东莞、中山、江门、揭阳)的布雷图指数(BI)监测记录,共计306个有效样本。各城市BI均值差异显著:深圳最低(均值2.2),揭阳最高(均值13.1),最大/最小比达5.9倍。这种城市间差异反映了不同城市在城市化水平、居住环境类型、蚊媒防控力度等方面的系统性差异,而非单纯的气候效应。

\textbf{(4)人口数据。}广州市常住人口数据来源于广东省统计年鉴,用于SEIR模型中易感人群池的初始化。考虑到登革热在中国大陆以输入性和散发性传播为主,人群免疫水平较低,本研究假设初始易感比例接近100\%。

\subsubsection{数据预处理}
\label{sec:p1-preprocess}

为确保不同量纲的气象变量在模型训练中具有可比性,本研究对所有气象输入变量进行Min-Max归一化处理。对于任意变量$x$,归一化公式为:
\begin{equation}
\label{eq:minmax}
x_{\text{norm}} = \frac{x - x_{\min}}{x_{\max} - x_{\min}}
\end{equation}
其中$x_{\min}$和$x_{\max}$分别为该变量在训练集中的最小值和最大值。归一化后所有变量的取值范围为$[0, 1]$,消除了量纲差异对梯度下降优化过程的不利影响。

对于布雷图指数(BI),本研究提出城市内归一化策略以消除城市间基线差异。如前所述,8个城市的BI均值从2.2到13.1不等,最大/最小比达5.9倍。若直接使用原始BI值作为建模目标,模型将不可避免地学习到城市间的系统性偏差,而非气候驱动的蚊媒密度变化规律。为此,定义城市内归一化BI(相对BI)为:
\begin{equation}
\label{eq:bi-rel}
\text{BI}_{\text{rel},i,t} = \frac{\text{BI}_{i,t}}{\overline{\text{BI}}_i}
\end{equation}
其中$\text{BI}_{i,t}$为城市$i$在时间$t$的原始BI值,$\overline{\text{BI}}_i$为城市$i$的BI时间均值。归一化后,$\text{BI}_{\text{rel}}$反映的是各城市相对于自身基线水平的蚊媒密度波动,消除了城市间因非气候因素(如城市化水平、防控力度、居住环境类型等)导致的绝对水平差异。由于8城BI均值的最大/最小比达5.9倍(深圳2.2 vs 揭阳13.1),若直接以BI绝对值为建模目标,模型将不可避免地混淆气候效应与城市间的系统性尺度差异,导致跨城泛化困难。归一化使模型聚焦于气候驱动的季节性相对变化,是实现跨城泛化的关键前提。

此外,考虑到登革热病例数据的极端右偏分布(64.4\%的月份病例数$\leq$5),本研究在涉及病例数的建模和评估中广泛采用对数变换$\log_{10}(1+C)$(即log1p变换),以压缩极端值的影响并改善残差的正态性。

\subsubsection{蚊媒密度公式发现}
\label{sec:p1-mosquito-formula}

蚊媒密度是连接气候因素与登革热传播的关键中间变量。由于BI监测数据在时间和空间上的稀缺性,本研究采用"神经网络预训练+符号回归蒸馏"的两阶段策略,从气候数据中发现可解释的蚊媒密度闭合公式。

\textbf{第一阶段:神经网络预训练。}构建一个三层多层感知机(MLP)网络,网络结构为32-16-1(即输入层到第一隐藏层32个神经元,第二隐藏层16个神经元,输出层1个神经元),激活函数采用ReLU。输入特征为6维向量,包括当月气象变量(温度$T$、相对湿度$H$、降水量$R$)及其对应的多年月均值($\bar{T}$、$\bar{H}$、$\bar{R}$)。引入多年月均值的目的是为模型提供气候学基线信息,使其能够区分"绝对气象条件"和"相对于气候态的异常程度"。预测目标为$\log(1 + \text{BI}_{\text{rel}})$,采用log1p变换以缓解目标变量的右偏分布。

训练配置如下:优化器采用Adam,学习率$\text{lr} = 5 \times 10^{-3}$,训练轮数5000个epoch,不使用早停策略。损失函数采用Huber损失与相关性损失的加权组合:
\begin{equation}
\label{eq:nn-loss}
\mathcal{L} = \mathcal{L}_{\text{Huber}}(\hat{y}, y) + \lambda \cdot (1 - r(\hat{y}, y))
\end{equation}
其中$\mathcal{L}_{\text{Huber}}$为Huber损失($\delta = 1.0$),$r(\hat{y}, y)$为预测值与观测值之间的Pearson相关系数,$\lambda$为相关性损失权重。Huber损失相比均方误差(MSE)对异常值更加鲁棒,适合处理BI数据中的极端值;相关性损失项鼓励模型捕捉预测值与观测值之间的线性趋势,而非仅最小化逐点误差。

\textbf{第二阶段:符号回归蒸馏。}利用训练好的神经网络在气象变量的均匀网格上生成8000个蒸馏数据点。具体而言,在温度、湿度和降水的归一化范围$[0, 1]$内均匀采样,通过神经网络前向传播获得对应的蚊媒密度预测值。这些蒸馏数据点相比原始306个BI样本具有以下优势:(1)数据量增加约26倍,为符号回归提供了充足的搜索信号;(2)数据分布均匀,避免了原始数据在特定气象条件下的聚集效应;(3)经过神经网络的平滑化处理,降低了观测噪声对公式搜索的干扰。

在蒸馏数据上运行PySR符号回归\cite{cranmer2023},搜索空间包括基本算术运算($+, -, \times, \div$)和三角函数($\sin, \cos$)。PySR采用多种群遗传编程算法,在公式复杂度和拟合精度之间进行Pareto前沿搜索。

\textbf{第三阶段:公式选择。}PySR输出一组位于Pareto前沿上的候选公式,按复杂度从低到高排列。公式选择遵循奥卡姆剃刀原则——在拟合精度可接受的前提下,选择复杂度最低的公式。具体选择标准为:在Pareto前沿上,选取复杂度增加但$R^2$提升不显著($\Delta R^2 < 0.02$)的拐点处的公式。最终选定的公式将在原始BI数据上进行独立验证,并通过留一城市交叉验证(Leave-One-City-Out, LOCO CV)评估其跨城泛化能力。

\subsubsection{Bri\`{e}re传播率模型}
\label{sec:p1-briere}

传播率$\beta'(T)$的温度依赖性采用Bri\`{e}re函数建模。Bri\`{e}re函数最初由Bri\`{e}re等\cite{briere1999}提出,用于描述昆虫发育速率对温度的非线性响应,其数学形式为:
\begin{equation}
\label{eq:briere}
\beta'(T) = cT(T - T_{\min})\sqrt{T_{\max} - T}, \quad T_{\min} \leq T \leq T_{\max}
\end{equation}
当$T < T_{\min}$或$T > T_{\max}$时,$\beta'(T) = 0$。该函数包含三个待估参数:尺度系数$c$、发育温度下限$T_{\min}$和发育温度上限$T_{\max}$。

Bri\`{e}re函数的生物学基础源于昆虫变温动物的温度生理学。在低温端,当环境温度低于$T_{\min}$时,蚊虫的代谢活动、飞行能力和叮咬行为基本停止,病毒在蚊体内的复制也趋于停滞,因此传播率为零。在高温端,当温度超过$T_{\max}$时,蚊虫因热应激导致死亡率急剧上升,种群密度骤降,传播同样中断。在$T_{\min}$和$T_{\max}$之间,传播率呈不对称的单峰形态,最优温度$T_{\text{opt}}$偏向$T_{\max}$一侧,这与实验室观测到的蚊虫叮咬率、产卵率和病毒外潜伏期的温度响应曲线一致\cite{mordecai2017,mordecai2019}。

选择Bri\`{e}re函数而非其他参数化形式(如高斯函数、多项式等)的理由在于:(1)Bri\`{e}re函数的不对称单峰形态与蚊媒传播的温度响应实验数据高度吻合,而对称的高斯函数无法捕捉这种不对称性;(2)三个参数均具有明确的生物学含义,可与独立的实验室数据进行交叉验证——例如,白纹伊蚊的发育温度下限在文献中报道为14--16°C\cite{mordecai2017,brady2013},可作为$T_{\min}$估计值的合理性检验标准;(3)仅需3个自由参数,在180个月度样本的小样本条件下不易过拟合,这一点在与7维神经网络的对比实验中将得到充分体现。

\subsubsection{SEIR离散时间预测框架}
\label{sec:p1-seir}

基于Bri\`{e}re传播率模型和蚊媒密度公式,本研究构建离散时间SEIR预测框架。与连续时间ODE模型不同,离散时间框架直接在月度时间步长上进行递推,避免了大时间步长下ODE数值积分的不稳定性问题。

每月新增病例的预测公式为:
\begin{equation}
\label{eq:seir-discrete}
\hat{C}_t = \beta'(T_t) \cdot \hat{M}_t \cdot S_t + \eta
\end{equation}
其中$\beta'(T_t)$为Bri\`{e}re温度响应函数在$t$月均温下的取值,$\hat{M}_t$为蚊媒密度公式的预测值,$S_t$为$t$月初的易感人群池规模,$\eta$为背景输入率(反映输入性病例和非气候因素的基线贡献)。

易感人群池的更新规则为:
\begin{equation}
\label{eq:susceptible-update}
S_{t+1} = S_t - \hat{C}_t + \delta \cdot N_h
\end{equation}
其中$\delta$为月度易感人群补充率,反映新生人口和免疫衰减的综合效应,$N_h$为总人口。在登革热的流行病学背景下,由于中国大陆人群对登革病毒的免疫水平较低(非地方性流行区),且登革热四个血清型之间仅存在短暂的交叉免疫保护,因此易感人群池的消耗主要发生在暴发年份,而在低发病年份$S_t$基本维持在接近$N_h$的水平。

模型参数通过差分进化(Differential Evolution, DE)算法进行全局优化。DE算法是一种基于种群的随机搜索算法,特别适合处理非凸、多模态的优化问题。优化目标为最小化预测病例数与观测病例数之间的对数尺度均方误差:
\begin{equation}
\label{eq:de-objective}
\min_{\theta} \sum_{t=1}^{T} \left[\log(1 + \hat{C}_t) - \log(1 + C_t)\right]^2
\end{equation}
其中$\theta = \{c, T_{\min}, T_{\max}, \eta, S_0, \delta\}$为待优化参数向量。采用对数尺度目标函数的原因在于:原始病例数跨越多个数量级(从0到数千),若在原始尺度上优化,损失函数将被少数极端暴发月份主导,而忽略大量低发病月份的拟合质量。DE算法的超参数设置为:种群规模为参数维度的15倍,最大迭代次数$\text{maxiter} = 800$,变异因子$F = 0.8$,交叉概率$CR = 0.7$。

\subsubsection{评估指标体系}
\label{sec:p1-metrics}

为全面评估模型的预测性能,本研究采用以下五个互补的评估指标:

\textbf{(1)Spearman秩相关系数($\rho$)。}衡量预测值与观测值之间的单调关联强度,不要求两者之间存在线性关系,对异常值具有鲁棒性:
\begin{equation}
\label{eq:spearman}
\rho = 1 - \frac{6\sum_{i=1}^{n} d_i^2}{n(n^2 - 1)}
\end{equation}
其中$d_i$为第$i$个样本的预测值秩次与观测值秩次之差。$\rho$的取值范围为$[-1, 1]$,$\rho = 1$表示完美的单调正相关。在登革热预测中,$\rho$特别适合评估模型是否正确捕捉了疫情的相对时序变化(如高峰月份和低谷月份的排序)。

\textbf{(2)Pearson相关系数($r$)。}衡量预测值与观测值之间的线性相关强度:
\begin{equation}
\label{eq:pearson}
r = \frac{\sum_{i=1}^{n}(\hat{y}_i - \bar{\hat{y}})(y_i - \bar{y})}{\sqrt{\sum_{i=1}^{n}(\hat{y}_i - \bar{\hat{y}})^2 \cdot \sum_{i=1}^{n}(y_i - \bar{y})^2}}
\end{equation}
$r$对极端值敏感,因此在登革热数据中通常在对数变换后计算。

\textbf{(3)对数尺度决定系数($R^2_{\log}$)。}在对数变换后的尺度上计算决定系数,综合评估模型的拟合优度:
\begin{equation}
\label{eq:r2log}
R^2_{\log} = 1 - \frac{\sum_{i=1}^{n}[\log(1+\hat{C}_i) - \log(1+C_i)]^2}{\sum_{i=1}^{n}[\log(1+C_i) - \overline{\log(1+C)}]^2}
\end{equation}
$R^2_{\log} = 1$表示完美拟合,$R^2_{\log} = 0$表示模型不优于均值预测,$R^2_{\log} < 0$表示模型劣于均值预测。对数变换使得该指标对低发病月份和高发病月份给予相对均衡的权重。

\textbf{(4)加权绝对百分比误差(WAPE)。}衡量预测值与观测值之间的相对偏差,以观测值为权重:
\begin{equation}
\label{eq:wape}
\text{WAPE} = \frac{\sum_{i=1}^{n}|\hat{C}_i - C_i|}{\sum_{i=1}^{n}C_i}
\end{equation}
WAPE的取值范围为$[0, +\infty)$,值越小表示预测越准确。与MAPE不同,WAPE避免了零病例月份导致的除零问题。

\textbf{(5)对数均方根误差(RMSLE)。}在对数尺度上计算均方根误差:
\begin{equation}
\label{eq:rmsle}
\text{RMSLE} = \sqrt{\frac{1}{n}\sum_{i=1}^{n}[\log(1+\hat{C}_i) - \log(1+C_i)]^2}
\end{equation}
RMSLE对预测值的低估和高估给予对称的惩罚,且在对数尺度上操作使其对极端值不过度敏感。

上述五个指标从不同角度评估模型性能:$\rho$和$r$评估相关性(趋势捕捉能力),$R^2_{\log}$评估整体拟合优度,WAPE评估绝对预测精度,RMSLE评估对数尺度上的预测误差。多指标综合评估可以避免单一指标可能带来的误导性结论。

\subsubsection{纯数据驱动对比实验}
\label{sec:p1-nn-baseline}

为验证物理先验的必要性,本研究设计了一组纯数据驱动的对比实验,用神经网络直接学习传播率$\beta$与气象变量的映射关系,替代Bri\`{e}re物理先验函数。

对比实验的具体流程如下:首先,利用SEIR连续时间ODE模型(公式\ref{eq:seir-ode}),通过二分法(bisection method)从观测病例数反演每月的有效传播率$\beta_t^{\text{obs}}$。具体而言,对于每个月$t$,在给定$S_t$、$I_t$和$N_h$的条件下,搜索使得ODE模型预测的新增病例数等于观测值$C_t$的$\beta$值。这一反演过程将连续时间ODE的动力学信息压缩为一个时间序列$\{\beta_t^{\text{obs}}\}_{t=1}^{T}$,为后续的神经网络学习提供了监督信号。

然后,构建一个与蚊媒密度NN结构相同的三层MLP网络(32-16-1,ReLU激活),输入为7维向量——6个气象特征($T, H, R, \bar{T}, \bar{H}, \bar{R}$)加上月份编码($\sin(2\pi m/12)$),预测目标为反演得到的$\beta_t^{\text{obs}}$。训练配置与蚊媒密度NN一致(Adam优化器,lr=$5\times10^{-3}$,5000 epochs,Huber+相关性损失)。

这一对比实验的设计逻辑在于:如果纯数据驱动方法能够从气象数据中有效学习传播率的时变模式,那么物理先验就不是必需的;反之,如果NN在相同数据条件下出现性能退化或模式坍缩,则说明物理先验在小样本条件下具有不可替代的正则化作用。通过将Bri\`{e}re方法和NN方法置于完全相同的数据条件和评估框架下进行对比,可以排除数据差异和评估标准差异带来的混淆因素,获得关于方法优劣的可靠结论。

%% ------------------------------------------------------------------
\subsection{结果}
\label{sec:p1-results}
%% ------------------------------------------------------------------

\subsubsection{蚊媒密度公式}
\label{sec:p1-res-mosquito}

PySR符号回归在Pareto前沿上输出了一系列候选公式,复杂度从3到25不等。在综合考虑精度和物理可解释性后,选定包含温度、湿度和降水三个气候因素的公式(复杂度19):
\begin{equation}
\label{eq:mhat}
\hat{M} = \frac{7.36\sqrt{T}}{T_m} - \frac{12.80}{H} - \frac{64.08}{H_m + \sqrt{R_m + R}}
\end{equation}
其中$T$为月均温度,$T_m$为城市年均温度,$H$为月均相对湿度,$H_m$为城市年均湿度,$R$为月降水量,$R_m$为城市年均降水量。该公式在真实数据上$R^2 = 0.205$,Pearson~$r = 0.462$。公式包含三个物理含义明确的项:第一项$\sqrt{T}/T_m$反映温度对蚊虫活动的正向驱动,平方根形式体现了高温区域的边际递减效应;第二项$-1/H$表明高湿度有利于蚊虫存活($H$越大,负贡献越小);第三项$-1/(H_m + \sqrt{R_m + R})$捕捉了降水对蚊虫孳生地的调节作用——适量降水增加积水容器,但该效应受城市基线湿度$H_m$的调制。

从物理意义上分析,该公式的三个项分别对应蚊媒生态学中的三个核心驱动因素。温度项$\sqrt{T}/T_m$的平方根形式意味着温度对蚊虫活动的促进效应在高温区域趋于饱和,这与Mordecai等\cite{mordecai2019}报道的蚊虫活动对温度的非线性响应一致。湿度项$-1/H$表明相对湿度通过影响成蚊存活率间接调控蚊媒密度。降水项中$\sqrt{R_m + R}$的平方根形式暗示降水对孳生地的贡献存在边际递减——初始降水显著增加积水容器,但持续强降水的边际效应减弱,甚至可能冲刷幼虫。图~\ref{fig:formula-scatter}展示了公式预测值与8城观测BI的散点对比。

\begin{figure}[H]
\centering
\includegraphics[width=0.95\textwidth]{../figures/fig1_formula_vs_obs_8cities.png}
\caption{三变量蚊媒公式$\hat{M}$预测值与8城观测BI的散点对比。虚线为$y=x$参考线。}
\label{fig:formula-scatter}
\end{figure}

作为公式发现的上游模型,神经网络在306个BI样本上的训练指标为:$R^2 = 0.251$,Pearson $r = 0.503$,Spearman $\rho = 0.518$。NN的$R^2$高于PySR公式(0.251 vs. 0.206),这是符合预期的——NN作为更灵活的非参数模型,在训练数据上的拟合能力必然优于简单的闭合公式。PySR公式的价值在于以极小的精度损失($\Delta R^2 = 0.045$)换取了完全的可解释性和零推理成本。

留一城市交叉验证(LOCO CV)结果验证了蚊媒公式的跨城泛化能力。LOCO CV的均值Pearson~$r = 0.493$,8个城市中有5个$r > 0.5$,表明公式在未见过的城市上能够较好地捕捉蚊媒密度的季节性变化趋势。值得注意的是,Pearson~$r$衡量的是公式预测值与观测值的线性相关性(即趋势一致性),这正是蚊媒公式在下游SEIR模型中的核心功能——绝对尺度由逐城市$\eta$参数校准,公式只需提供正确的季节性变化模式。3个$r$较低的城市(揭阳$r=0.14$、深圳$r=0.22$、茂名$r=0.47$)的样本量较少或BI分布特殊,泛化难度较大。作为消融实验,若去除城市内归一化而直接以BI绝对值为目标,LOCO CV均值$r$降至$0.31$,$r>0.5$的城市从5个减少到2个,表明归一化策略对于跨城泛化不可或缺。

\subsubsection{Bri\`{e}re传播率参数}
\label{sec:p1-res-briere}

差分进化算法在广州2005--2019年数据上优化得到的Bri\`{e}re函数参数为:尺度系数$c = 7.15 \times 10^{-4}$,发育温度下限$T_{\min} = 15.1$°C,发育温度上限$T_{\max} = 42.0$°C。由此计算的最优传播温度为$T_{\text{opt}} = 35.5$°C,背景输入率$\eta = 0.709$。

这些参数估计值与独立的实验室研究结果具有良好的一致性。$T_{\min} = 15.1$°C落在文献报道的白纹伊蚊发育温度下限范围(14--16°C)之内\cite{mordecai2017,brady2013},表明模型从流行病学数据中反演得到的温度阈值与蚊虫生理学实验数据吻合。$T_{\max} = 42.0$°C略高于部分实验室研究报道的上限(约38--40°C),这可能反映了以下因素:(1)实验室条件下的恒温暴露与自然环境中的日温度波动存在差异,蚊虫在自然条件下可能通过行为调节(如寻找阴凉微环境)部分规避极端高温;(2)$T_{\max}$在优化过程中的敏感性较低,因为广州月均温极少超过35°C,高温端的数据约束较弱。

$T_{\text{opt}} = 35.5$°C高于Mordecai等\cite{mordecai2019}报道的最优传播温度(约29°C),这一差异需要谨慎解读。Mordecai等的估计基于$R_0$的多个组分(叮咬率、存活率、外潜伏期等)的综合效应,而本研究的$\beta'(T)$仅反映传播率的温度依赖性,不包含蚊虫存活率等因素(这些因素已部分被蚊媒密度公式$\hat{M}$所捕捉)。因此,两者的最优温度不具有直接可比性。此外,$\beta'(T)$在实际应用中总是与$\hat{M}_t$相乘,两者的乘积效应决定了有效传播强度的温度响应,其峰值温度可能低于$T_{\text{opt}}$。

\begin{figure}[H]
\centering
\includegraphics[width=0.7\textwidth]{../figures/fig5_briere_curve.png}
\caption{Bri\`{e}re传播率函数$\beta'(T)$曲线。$T_{\min}=15.1^\circ$C, $T_{\max}=42.0^\circ$C, $T_{\text{opt}}=35.5^\circ$C。}
\label{fig:briere-curve}
\end{figure}

\subsubsection{广州拟合与交叉验证}
\label{sec:p1-res-gz}

表\ref{tab:gz-metrics}汇总了Bri\`{e}re模型在广州市不同评估场景下的预测性能。

\begin{table}[htbp]
\centering
\caption{Bri\`{e}re模型在广州市的预测性能}
\label{tab:gz-metrics}
\begin{tabular}{lcccc}
\hline
评估场景 & Spearman $\rho$ & $R^2_{\log}$ & Pearson $r$ & WAPE \\
\hline
全期拟合(2005--2019) & 0.814 & 0.851 & 0.714 & 0.679 \\
LOYO CV 全部年份 & $0.740 \pm 0.185$ & $0.442 \pm 0.534$ & --- & --- \\
LOYO CV 暴发年(病例$>$50) & $0.806 \pm 0.098$ & $0.660 \pm 0.218$ & --- & --- \\
LOYO CV 低发年 & $0.561 \pm 0.218$ & --- & --- & --- \\
\hline
\end{tabular}
\end{table}

在全期拟合中,模型取得了$\rho = 0.814$的Spearman秩相关系数和$R^2_{\log} = 0.851$的对数尺度决定系数,表明模型能够较好地捕捉广州登革热疫情的时序变化模式,包括季节性波动和年际差异。Pearson相关系数$r = 0.714$略低于$\rho$,反映了对数变换后预测值与观测值之间存在一定的非线性偏差。WAPE = 0.679表明模型的加权绝对预测误差约为观测总量的68\%,这一数值看似较高,但考虑到登革热病例数据的极端变异性(2014年单年病例数是其他年份均值的数十倍),该WAPE值在传染病预测领域属于可接受范围。

留一年交叉验证(Leave-One-Year-Out, LOYO CV)结果揭示了模型性能与疫情强度之间的密切关系。在所有15个留出年份中,LOYO CV的平均$\rho = 0.740 \pm 0.185$,$R^2_{\log} = 0.442 \pm 0.534$。将留出年份按疫情强度分组后,差异更加明显:暴发年份(年病例数$>$50)的平均$\rho = 0.806 \pm 0.098$,$R^2_{\log} = 0.660 \pm 0.218$,性能接近全期拟合水平;而低发年份的平均$\rho = 0.561 \pm 0.218$,性能显著下降。

进一步分析发现,LOYO CV中$R^2_{\log} < 0$的年份集中在极低发病年份:2005年(全年仅9例,58\%月份零病例)和2008年(全年仅10例,67\%月份零病例)。在这些年份中,观测病例数几乎全为零或个位数,信噪比极低,任何基于气候驱动的模型都难以产生有意义的预测——因为实际传播可能主要由随机输入事件而非系统性气候因素驱动。这一发现提示,气候驱动的传播模型在低发病背景下的适用性存在固有局限,模型的核心价值在于预测暴发年份的疫情动态。

\begin{figure}[H]
\centering
\includegraphics[width=0.85\textwidth]{../figures/fig6_guangzhou_fit.png}
\caption{广州市2005--2019年月度病例观测值与Bri\`{e}re模型预测值对比。}
\label{fig:gz-fit}
\end{figure}

\begin{figure}[H]
\centering
\includegraphics[width=0.75\textwidth]{../figures/fig7_loyo_cv.png}
\caption{广州市留一年交叉验证(LOYO CV)各年Spearman $\rho$。}
\label{fig:loyo-cv}
\end{figure}

\subsubsection{NN $\beta$对比实验}
\label{sec:p1-res-nn}

表\ref{tab:briere-vs-nn}对比了Bri\`{e}re物理先验方法和NN纯数据驱动方法在广州市的预测性能。

\begin{table}[htbp]
\centering
\caption{Bri\`{e}re物理先验方法与NN纯数据驱动方法的性能对比}
\label{tab:briere-vs-nn}
\begin{tabular}{lccccc}
\hline
方法 & 参数量 & Spearman $\rho$ & $R^2_{\log}$ & LOYO $\rho$ & $\beta$ CV \\
\hline
Bri\`{e}re先验 & 3 & 0.814 & 0.851 & $0.740 \pm 0.185$ & --- \\
NN数据驱动 & $\sim$600 & 0.706 & 0.230 & $0.570$ & 1.17\% \\
\hline
\end{tabular}
\end{table}

结果表明,Bri\`{e}re方法在所有评估指标上均显著优于NN方法。在全期拟合中,Bri\`{e}re方法的$\rho$(0.814 vs. 0.706)和$R^2_{\log}$(0.851 vs. 0.230)均大幅领先。在LOYO交叉验证中,Bri\`{e}re方法的优势进一步扩大($\rho$: 0.740 vs. 0.570),表明物理先验提供的正则化效应在样本外预测中尤为重要。

更值得关注的是NN方法出现的模式坍缩(mode collapse)现象。分析NN预测的$\beta$时间序列发现,NN输出的$\beta$值几乎坍缩为常数:均值约为0.102,标准差仅为0.0012,变异系数(CV)仅1.17\%。这意味着NN实质上放弃了学习$\beta$的时变模式,退化为预测一个与气象条件无关的常数传播率。

这一模式坍缩的形成机制可以从训练数据的分布特征和损失函数的交互作用来理解。广州180个月度样本中,64.4\%的月份病例数$\leq$5例,反演得到的$\beta_t^{\text{obs}}$在这些月份中接近零或极小值。在这种高度偏斜的目标分布下,Huber损失函数的最优策略是预测接近中位数的常数值——因为偏离中位数的惩罚(对大量低值样本)远大于准确预测少数高值样本带来的收益。当NN的预测值趋于常数时,预测值的标准差$\text{std}(\hat{y})$趋近于零,导致相关性损失项中的分母($\text{std}(\hat{y})$)趋近于零,Pearson相关系数变得数值不稳定。在实际实现中,当$\text{std}(\hat{y}) < \epsilon$(数值稳定性阈值)时,相关性损失被禁用(设为零),从而消除了唯一能够鼓励NN学习时变模式的梯度信号。这形成了一个自我强化的陷阱:Huber损失驱动预测趋于常数$\to$预测标准差趋于零$\to$相关性损失被禁用$\to$失去学习时变模式的梯度$\to$预测进一步趋于常数。

这一发现具有重要的方法论启示:在小样本、高偏斜的传染病数据条件下,纯数据驱动方法面临的不仅是过拟合风险,更是模式坍缩这一更为根本的失败模式。物理先验(如Bri\`{e}re函数)通过将模型的函数形态约束在生物学合理的范围内,从根本上避免了坍缩为常数的可能性——因为Bri\`{e}re函数的数学形式保证了$\beta'(T)$必然随温度变化,不可能退化为常数。

%% ------------------------------------------------------------------
\subsection{讨论}
\label{sec:p1-discussion}
%% ------------------------------------------------------------------

\subsubsection{物理先验的必要性}
\label{sec:p1-disc-physics}

本章的核心发现之一是:在广州180个月度样本的小样本条件下,Bri\`{e}re物理先验方法在所有评估维度上均显著优于NN纯数据驱动方法。这一结果的深层原因在于物理先验所提供的"免费午餐"——通过将数十年昆虫学实验研究积累的知识编码为函数形态约束,Bri\`{e}re函数仅用3个参数就捕捉了传播率温度依赖性的核心特征(不对称单峰、有限温度范围、偏向高温的最优值),而NN需要从有限的数据中同时学习函数形态和参数值,在约600个可训练参数的高维空间中搜索,不可避免地陷入模式坍缩。

从信息论的角度理解,Bri\`{e}re函数的3个参数对应约$\log_2(3!) \approx 2.6$比特的模型复杂度,而NN的约600个参数对应数千比特的模型复杂度。根据最小描述长度(Minimum Description Length, MDL)原则\cite{holm2019},当数据量不足以支撑高复杂度模型时,简单模型的泛化性能必然优于复杂模型。广州的180个月度样本(其中约116个为低/零病例月,有效信息量更少)显然不足以约束一个600参数的NN,但足以可靠地估计Bri\`{e}re函数的3个参数。

这一发现对传染病建模领域具有普遍意义。在许多传染病(如疟疾、寨卡、基孔肯雅热等)的建模中,研究者面临类似的小样本挑战——月度或周度分辨率的时间序列通常仅有数十到数百个样本,且分布高度偏斜。本研究的结果表明,在这种数据条件下,盲目追求模型的灵活性(如使用深度学习)可能适得其反,而利用已有的生物学知识约束模型结构是更为稳健的策略。

\subsubsection{蚊媒公式的改进与解释}
\label{sec:p1-disc-mosquito}

城市内归一化策略的必要性源于蚊媒密度建模中一个常被忽视的问题:不同城市的BI绝对水平差异(本研究中最大/最小比达5.9倍)主要反映的是非气候因素(城市化水平、居住环境类型、防控力度等)的系统性差异,而非气候驱动的蚊媒密度变化。归一化到城市内相对变化后,模型得以聚焦于气候因素的季节性调控效应,这是实现跨城泛化的关键前提。消融实验表明,去除归一化后LOCO CV均值$r$从0.493降至0.31,$r>0.5$的城市从5个减少到2个,进一步证实了这一判断。

PySR发现的蚊媒公式$\hat{M}$包含温度、湿度和降水三个气候因素,各项的物理含义与已有的蚊媒生态学知识一致。$\sqrt{T}/T_m$项反映了温度对蚊虫发育和活动的正向驱动,平方根形式暗示高温区域的边际效应递减,这与Mordecai等\cite{mordecai2019}报道的蚊虫活动对温度的非线性响应一致。$-1/H$项表明湿度对蚊虫存活的正向作用,与Xu等\cite{xu2020}在广州的研究结论吻合。降水项$-1/(H_m + \sqrt{R_m + R})$的结构较为复杂,反映了降水通过增加积水孳生地促进蚊虫繁殖的效应,但该效应受城市基线湿度的调制——在本底湿度较高的城市,额外降水的边际贡献较小。公式的$R^2 = 0.205$表明气候因素解释了蚊媒密度变异的约20\%,其余变异来源于城市化水平、防控措施、容器类型等未被公式捕捉的非气候因素。

\subsubsection{极端年份分析}
\label{sec:p1-disc-extreme}

2014年广州登革热大暴发(37,382例)是本研究时间序列中最显著的极端事件。值得注意的是,模型估计的2014年Bri\`{e}re传播率$\beta'(T)$与其他年份相比并无显著异常——2014年的月均温序列处于正常范围内,$\beta'(T)$的季节性模式与其他年份高度相似。这一发现表明,2014年暴发的驱动因素并非异常的气候条件,而更可能是非气候因素的叠加效应:(1)输入性病例的时间窗口恰好与蚊媒活动高峰重合;(2)当年防控响应的滞后导致早期传播链未被及时阻断;(3)人群免疫水平极低(广州非登革热地方性流行区)使得一旦传播链建立,疫情可以指数增长\cite{cheng2016,ccm14}。

从建模角度看,这一发现支持了将传播率分解为气候驱动组分($\beta'(T) \cdot \hat{M}_t$)和非气候基线组分($\eta$)的建模策略。气候驱动组分决定了传播的"时间窗口"(哪些月份具备传播条件),而暴发的实际规模还取决于输入强度、防控措施和人群免疫状态等非气候因素。这种分解使得模型能够正确预测暴发的季节性时机,即使无法精确预测暴发的绝对规模。

在LOYO CV中表现最差的2005年(9例)和2008年(10例)则代表了另一种极端——极低发病年份。在这些年份中,全年病例数仅为个位数,超过一半的月份报告零病例,疫情动态几乎完全由随机输入事件驱动,气候因素的系统性影响被随机噪声淹没。对于这类年份,任何基于气候驱动的确定性模型都无法产生有意义的预测,这是气候驱动建模方法的固有局限而非模型缺陷。

\subsubsection{方法论意义}
\label{sec:p1-disc-methodology}

本章提出的"NN+PySR蚊媒公式发现+Bri\`{e}re物理先验+SEIR离散框架"的混合建模策略,为传染病建模提供了一种兼顾可解释性和预测精度的新范式。与传统的纯机制模型相比,本方法通过符号回归从数据中发现蚊媒密度公式,避免了对蚊媒动力学的先验假设;与纯数据驱动方法相比,本方法通过Bri\`{e}re物理先验约束传播率的函数形态,避免了小样本条件下的模式坍缩。

NN模式坍缩的发现和机制分析(Huber损失驱动常数预测$\to$相关性损失失效$\to$自我强化陷阱)对机器学习在科学应用中的实践具有警示意义。在许多科学领域(如气候科学、生态学、流行病学),训练数据往往呈高度偏斜分布,且样本量有限。在这种条件下,标准的深度学习训练流程可能产生看似合理但实质上已坍缩的模型——模型的训练损失可能持续下降(因为预测常数值确实最小化了Huber损失),但模型已完全丧失了捕捉目标变量时变模式的能力。这提示研究者在使用深度学习处理偏斜分布数据时,应当常规检查预测值的变异性(如变异系数),而非仅关注损失函数的收敛行为。

此外,本章的对比实验设计——将物理先验方法和数据驱动方法置于完全相同的数据条件和评估框架下——为传染病建模中的方法选择提供了可复制的实证范式。未来的研究可以在不同传染病、不同地理区域和不同数据条件下重复这一对比实验,系统评估物理先验的适用边界。Li等\cite{li2019pnas}和Zhang等\cite{zhang2024plos}的工作已经表明,在蚊媒传染病建模中融合物理知识和数据驱动方法是一个富有前景的研究方向,本研究为这一方向提供了新的实证支持和方法论贡献。
