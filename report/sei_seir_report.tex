\documentclass[12pt,a4paper]{article}
\usepackage[utf8]{inputenc}
\usepackage[T1]{fontenc}
\usepackage{ctex}
\usepackage{amsmath,amssymb,amsfonts}
\usepackage{mathtools}
\usepackage{graphicx}
\usepackage{booktabs}
\usepackage{array}
\usepackage{geometry}
\usepackage{hyperref}
\usepackage{xcolor}
\usepackage{listings}
\usepackage{float}
\usepackage{subcaption}

\geometry{left=2.5cm,right=2.5cm,top=2.5cm,bottom=2.5cm}

\hypersetup{
    colorlinks=true,
    linkcolor=blue,
    filecolor=magenta,
    urlcolor=cyan,
}

\lstset{
    basicstyle=\ttfamily\small,
    breaklines=true,
    frame=single,
    language=Python,
    keywordstyle=\color{blue},
    commentstyle=\color{green!60!black},
    stringstyle=\color{red},
}

\title{\textbf{SEI-SEIR蚊媒-人群动力学模型} \\ \large 基于TCN神经网络的登革热传播预测}
\author{技术报告}
\date{\today}

\begin{document}

\maketitle

\begin{abstract}
本报告提出了一种基于SEI-SEIR微分方程和TCN(时间卷积网络)相结合的蚊媒-人群动力学模型,用于模拟和预测登革热的传播动态。模型使用广州2006-2014年的真实气象数据和布雷图指数(BI)数据进行训练和验证。实验结果表明,TCN网络能够有效地从气象数据预测蚊虫密度指标(R²=0.988),为登革热的早期预警提供了有效工具。
\end{abstract}

\tableofcontents
\newpage

%=============================================================================
\section{引言}
%=============================================================================

登革热是一种由登革病毒引起的急性传染病,主要通过伊蚊(特别是埃及伊蚊和白纹伊蚊)传播。全球每年约有3.9亿人感染登革热,其中约9600万人出现临床症状\cite{who2023}。中国南方地区,特别是广东省,是登革热的主要流行区域。

传统的传染病动力学模型(如SIR、SEIR)主要关注人群内部的传播,而忽视了媒介昆虫在疾病传播中的关键作用。本研究构建了一个\textbf{SEI-SEIR耦合模型},将蚊媒动力学与人群动力学结合,更准确地描述登革热的传播机制。

\subsection{研究目标}

\begin{enumerate}
    \item 建立SEI-SEIR蚊媒-人群耦合动力学模型
    \item 使用TCN神经网络从气象数据预测蚊虫承载力$\Lambda_v(t)$
    \item 用广州市布雷图指数(BI)数据验证模型预测性能
    \item 计算基本再生数$R_0$评估传播风险
\end{enumerate}

%=============================================================================
\section{模型框架}
%=============================================================================

\subsection{SEI-SEIR耦合模型结构}

模型将登革热传播分为两个相互作用的子系统:

\begin{itemize}
    \item \textbf{蚊媒部分(SEI)}:易感蚊(S$_v$) $\rightarrow$ 暴露蚊(E$_v$) $\rightarrow$ 感染蚊(I$_v$)
    \item \textbf{人群部分(SEIR)}:易感人群(S$_h$) $\rightarrow$ 暴露人群(E$_h$) $\rightarrow$ 感染人群(I$_h$) $\rightarrow$ 康复人群(R$_h$)
\end{itemize}

\begin{center}
\fbox{\parbox{0.9\textwidth}{
\centering
\textbf{模型流程}: \\[0.5em]
气象数据 $\xrightarrow{\text{TCN}}$ 蚊虫承载力$\Lambda_v(t)$ $\xrightarrow{\text{SEI}}$ 蚊媒动态 $\longleftrightarrow$ 人群动态(SEIR)
}}
\end{center}

\subsection{微分方程组}

\subsubsection{蚊媒动力学方程 (SEI)}

\begin{align}
    \frac{dS_v}{dt} &= \Lambda_v(t) - \mu_v S_v - \beta_v b \frac{I_h}{N_h} S_v \label{eq:sv}\\
    \frac{dE_v}{dt} &= \beta_v b \frac{I_h}{N_h} S_v - (\mu_v + \sigma_v) E_v \label{eq:ev}\\
    \frac{dI_v}{dt} &= \sigma_v E_v - \mu_v I_v \label{eq:iv}
\end{align}

其中:
\begin{itemize}
    \item $\Lambda_v(t)$: 蚊虫出生率(由TCN网络预测)
    \item $\mu_v$: 蚊虫死亡率
    \item $\beta_v$: 人$\rightarrow$蚊传播概率
    \item $b$: 叮咬率
    \item $\sigma_v$: 蚊虫潜伏期转化率($1/\text{EIP}$)
\end{itemize}

\subsubsection{人群动力学方程 (SEIR)}

\begin{align}
    \frac{dS_h}{dt} &= -\beta_h b \frac{I_v}{N_h} S_h \label{eq:sh}\\
    \frac{dE_h}{dt} &= \beta_h b \frac{I_v}{N_h} S_h - \sigma_h E_h \label{eq:eh}\\
    \frac{dI_h}{dt} &= \sigma_h E_h - \gamma I_h \label{eq:ih}\\
    \frac{dR_h}{dt} &= \gamma I_h \label{eq:rh}
\end{align}

其中:
\begin{itemize}
    \item $\beta_h$: 蚊$\rightarrow$人传播概率
    \item $\sigma_h$: 人潜伏期转化率
    \item $\gamma$: 康复率
    \item $N_h$: 总人口数
\end{itemize}

%=============================================================================
\section{TCN神经网络}
%=============================================================================

\subsection{网络架构}

时间卷积网络(TCN)是一种专为序列建模设计的深度学习架构,具有以下特点:

\begin{enumerate}
    \item \textbf{因果卷积}:确保模型只使用历史信息进行预测
    \item \textbf{膨胀卷积}:通过指数增长的膨胀因子扩大感受野
    \item \textbf{残差连接}:解决深层网络的梯度消失问题
\end{enumerate}

\begin{table}[H]
    \centering
    \caption{TCN网络结构}
    \begin{tabular}{lcc}
        \toprule
        \textbf{层} & \textbf{输出通道数} & \textbf{膨胀因子} \\
        \midrule
        输入层 & 3 (温度、湿度、降雨) & - \\
        TemporalBlock 1 & 64 & 1 \\
        TemporalBlock 2 & 128 & 2 \\
        TemporalBlock 3 & 64 & 4 \\
        全连接层 & 1 (BI预测) & - \\
        \bottomrule
    \end{tabular}
\end{table}

\subsection{训练配置}

\begin{itemize}
    \item 输入窗口:6个月历史气象数据
    \item 学习率:0.0005
    \item 优化器:Adam
    \item 损失函数:MSE
    \item 训练轮数:300
    \item 批次大小:8
    \item 验证集比例:20\%
\end{itemize}

%=============================================================================
\section{数据来源}
%=============================================================================

\subsection{气象数据}

气象数据来源于Open-Meteo API,包含广州市2006-2014年的月度数据:

\begin{table}[H]
    \centering
    \caption{气象数据统计}
    \begin{tabular}{lccc}
        \toprule
        \textbf{变量} & \textbf{最小值} & \textbf{最大值} & \textbf{均值} \\
        \midrule
        温度 (°C) & 10.3 & 29.0 & 22.3 \\
        相对湿度 (\%) & 50.4 & 88.7 & 75.7 \\
        降雨量 (mm/月) & 0.4 & 900.9 & 151.7 \\
        \bottomrule
    \end{tabular}
\end{table}

\subsection{布雷图指数数据}

布雷图指数(Breteau Index, BI)数据来自CCM14数据集\cite{ccm14},定义为:

\begin{equation}
    \text{BI} = \frac{\text{阳性容器数}}{\text{检查户数}} \times 100
\end{equation}

根据WS/T 784—2021标准\cite{wst784},BI的风险分级为:
\begin{itemize}
    \item BI $\leq$ 5:安全
    \item 5 $<$ BI $\leq$ 10:低风险
    \item 10 $<$ BI $\leq$ 20:中风险
    \item BI $>$ 20:高风险
\end{itemize}

\begin{table}[H]
    \centering
    \caption{广州BI数据统计 (2006-2014)}
    \begin{tabular}{lc}
        \toprule
        \textbf{指标} & \textbf{值} \\
        \midrule
        有效数据点 & 74个月 \\
        BI均值 & 6.64 \\
        BI标准差 & 7.21 \\
        BI最大值 & 38.71 \\
        BI最小值 & 0.00 \\
        \bottomrule
    \end{tabular}
\end{table}

%=============================================================================
\section{模型参数}
%=============================================================================

模型参数基于文献设定,部分参数具有温度依赖性:

\begin{table}[H]
    \centering
    \caption{SEI-SEIR模型参数}
    \begin{tabular}{llcl}
        \toprule
        \textbf{参数} & \textbf{符号} & \textbf{默认值} & \textbf{说明} \\
        \midrule
        蚊虫出生率 & $\Lambda_v(t)$ & TCN预测 & 气象驱动 \\
        蚊虫死亡率 & $\mu_v$ & 0.05 day$^{-1}$ & 约20天寿命 \\
        人$\rightarrow$蚊传播概率 & $\beta_v$ & 0.5 & 文献值 \\
        蚊$\rightarrow$人传播概率 & $\beta_h$ & 0.75 & 文献值 \\
        叮咬率 & $b$ & 0.5 day$^{-1}$ & 文献值 \\
        蚊虫潜伏期转化率 & $\sigma_v$ & 0.1 day$^{-1}$ & EIP$\approx$10天 \\
        人潜伏期转化率 & $\sigma_h$ & 0.2 day$^{-1}$ & 约5天 \\
        康复率 & $\gamma$ & 0.143 day$^{-1}$ & 约7天 \\
        总人口 & $N_h$ & 14,000,000 & 广州市人口 \\
        \bottomrule
    \end{tabular}
\end{table}

\subsection{基本再生数}

基本再生数$R_0$计算公式:

\begin{equation}
    R_0 = \sqrt{R_{0,vh} \times R_{0,hv}}
\end{equation}

其中:
\begin{align}
    R_{0,vh} &= \frac{\beta_h \cdot b \cdot \sigma_v \cdot N_v}{\mu_v \cdot (\mu_v + \sigma_v) \cdot N_h} \quad \text{(蚊$\rightarrow$人传播潜力)} \\
    R_{0,hv} &= \frac{\beta_v \cdot b}{\gamma} \quad \text{(人$\rightarrow$蚊传播潜力)}
\end{align}

当$R_0 > 1$时,疾病可能暴发流行。

%=============================================================================
\section{实验结果}
%=============================================================================

\subsection{TCN预测性能}

使用真实数据训练后,TCN模型在布雷图指数预测上取得了优异的性能:

\begin{table}[H]
    \centering
    \caption{TCN预测性能指标}
    \begin{tabular}{lc}
        \toprule
        \textbf{指标} & \textbf{值} \\
        \midrule
        决定系数 (R²) & \textbf{0.988} \\
        相关系数 & \textbf{0.997} \\
        平均绝对误差 (MAE) & 0.53 \\
        均方根误差 (RMSE) & 0.80 \\
        有效样本数 & 70 \\
        \bottomrule
    \end{tabular}
\end{table}

$R^2 = 0.988$表明模型解释了98.8\%的BI变异,预测效果非常好。

\subsection{基本再生数分析}

\begin{table}[H]
    \centering
    \caption{$R_0$统计结果 (2006-2014)}
    \begin{tabular}{lc}
        \toprule
        \textbf{指标} & \textbf{值} \\
        \midrule
        $R_0$均值 & 0.45 \\
        $R_0$最大值 & 0.76 \\
        $R_0$最小值 & 0.09 \\
        $R_0 > 1$的比例 & 0\% \\
        \bottomrule
    \end{tabular}
\end{table}

结果表明,在2006-2014年期间,广州市的$R_0$始终小于1,未达到大规模暴发的阈值。$R_0$最高值出现在2008年7月,对应BI峰值38.7。

\subsection{可视化结果}

\begin{figure}[H]
    \centering
    \includegraphics[width=\textwidth]{../results/real_data_results.png}
    \caption{SEI-SEIR + TCN模型综合结果}
    \label{fig:results}
\end{figure}

图\ref{fig:results}包含以下子图:
\begin{itemize}
    \item (a) TCN训练曲线
    \item (b) BI预测对比
    \item (c) 预测vs观测散点图
    \item (d) 广州月度温度和降雨
    \item (e) 估计的蚊虫出生率$\Lambda_v$
    \item (f) 基本再生数$R_0$时间序列
    \item (g) 蚊虫SEI动态
    \item (h) 人群SEIR动态
    \item (i) 感染蚊虫与BI对比
    \item (j) BI季节性模式
    \item (k) 温度-BI关系
    \item (l) 模型性能总结
\end{itemize}

%=============================================================================
\section{讨论}
%=============================================================================

\subsection{模型优势}

\begin{enumerate}
    \item \textbf{机理与数据结合}:SEI-SEIR提供生物学机理框架,TCN提供数据驱动的参数估计
    \item \textbf{气象驱动}:直接利用可获取的气象预报数据进行预测
    \item \textbf{高预测精度}:$R^2 = 0.988$的BI预测精度
    \item \textbf{可解释性}:通过$R_0$等指标评估传播风险
\end{enumerate}

\subsection{局限性}

\begin{enumerate}
    \item \textbf{病例数据缺失}:未使用实际登革热病例数据验证人群动态
    \item \textbf{$R_0$偏低}:可能与蚊虫密度估计或参数设置有关
    \item \textbf{地理局限}:仅使用广州数据,泛化能力待验证
\end{enumerate}

\subsection{未来工作}

\begin{enumerate}
    \item 获取广东省CDC的实际病例数据进行完整验证
    \item 扩展到其他登革热流行区域
    \item 结合气象预报实现提前预警
    \item 引入空间异质性建立时空模型
\end{enumerate}

%=============================================================================
\section{结论}
%=============================================================================

本研究成功构建了SEI-SEIR蚊媒-人群耦合动力学模型,并使用TCN神经网络从气象数据预测蚊虫承载力。主要结论如下:

\begin{enumerate}
    \item TCN网络能够有效捕捉气象因素与蚊虫密度之间的非线性关系,预测$R^2$达到0.988
    \item SEI-SEIR模型能够模拟蚊媒与人群之间的双向传播动态
    \item 广州市2006-2014年间$R_0 < 1$,表明该时期未达到大规模流行阈值
    \item 该框架为登革热的监测和早期预警提供了有效工具
\end{enumerate}

%=============================================================================
\section*{参考文献}
%=============================================================================

\begin{thebibliography}{99}

\bibitem{who2023}
World Health Organization. Dengue and severe dengue. WHO Fact Sheet, 2023.

\bibitem{ccm14}
xyyu001. CCM14: Mosquito surveillance data in China. GitHub Repository, 2024.
\url{https://github.com/xyyu001/CCM14}

\bibitem{wst784}
中华人民共和国卫生行业标准. WS/T 784—2021 登革热病媒生物应急监测与控制标准. 2021.

\bibitem{mordecai2017}
Mordecai EA, et al. Detecting the impact of temperature on transmission of Zika, dengue, and chikungunya using mechanistic models. \textit{PLOS Neglected Tropical Diseases}, 2017.

\bibitem{yang2009}
Yang HM, et al. Assessing the effects of temperature on the population of Aedes aegypti, the vector of dengue. \textit{Epidemiology \& Infection}, 2009.

\bibitem{openmeteo}
Open-Meteo. Free Weather API for non-commercial use.
\url{https://open-meteo.com/}

\end{thebibliography}

%=============================================================================
\appendix
\section{代码使用说明}
%=============================================================================

\subsection{安装依赖}

\begin{lstlisting}[language=bash]
pip install numpy scipy pandas torch matplotlib scikit-learn requests
\end{lstlisting}

\subsection{获取数据}

\begin{lstlisting}[language=bash]
python src/fetch_real_data.py
\end{lstlisting}

\subsection{训练模型}

\begin{lstlisting}[language=bash]
python src/train_with_real_data.py
\end{lstlisting}

\subsection{Python代码示例}

\begin{lstlisting}[language=Python]
from src.sei_seir_model import SEISEIRModel, SEISEIRParameters

# 创建参数
params = SEISEIRParameters(
    N_h=14_000_000,
    mu_v=0.05,
    beta_v=0.5,
    beta_h=0.75
)

# 创建模型
model = SEISEIRModel(params, lambda_v_func=lambda t: 1000)

# 运行模拟
y0 = model.get_initial_conditions(N_v0=10000, I_h0=10)
result = model.solve(y0, t_span=(0, 365))

# 计算R0
R0 = model.compute_R0(N_v=10000, temperature=25)
print(f"R0 = {R0:.2f}")
\end{lstlisting}

\end{document}
