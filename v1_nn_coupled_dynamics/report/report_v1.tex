\documentclass[12pt,a4paper]{article}
\usepackage[utf8]{inputenc}
\usepackage[T1]{fontenc}
\usepackage{ctex}
\usepackage{amsmath,amssymb,amsfonts}
\usepackage{mathtools}
\usepackage{graphicx}
\usepackage{booktabs}
\usepackage{array}
\usepackage{geometry}
\usepackage{hyperref}
\usepackage{xcolor}
\usepackage{float}
\usepackage{subcaption}
\usepackage{enumitem}
\usepackage{multirow}

\geometry{left=2.5cm,right=2.5cm,top=2.5cm,bottom=2.5cm}

\hypersetup{
    colorlinks=true,
    linkcolor=blue,
    citecolor=blue,
    urlcolor=cyan,
}

\title{
    \textbf{神经网络耦合动力学模型研究} \\[0.5em]
    \large 基于数据驱动的蚊媒传染病传播机制发现 \\[0.3em]
    \normalsize Version 1.0 — 排除2014暴发数据
}
\author{技术报告}
\date{\today}

\begin{document}

\maketitle

\begin{abstract}
本报告提出一种\textbf{神经网络耦合动力学模型}框架,用于研究登革热蚊媒传播的数学机理。
参照 Zhang, Wang \& Tang (2024, \textit{PLoS Computational Biology}) 的方法,
将未知的蚊虫产卵率函数用神经网络(NN)替代,嵌入微分方程系统内部,
通过ODE数值解与观测数据的误差反向传播来间接训练神经网络。
训练完成后,采用符号回归(Symbolic Regression)将NN的黑箱输入输出关系转化为显式的解析公式,
最终获得完全可解释的动力学模型。

使用广东省2006--2019年(排除2014年极端暴发)的布雷图指数(BI)和登革热病例数据进行训练验证。
结果显示:蚊虫种群与BI的相关系数 $r=0.28$,
\textbf{病例拟合相关系数 $r=0.59$($p<10^{-15}$),2019年暴发预测 $r=0.80$}。
符号回归发现产卵率的最优近似为三次多项式,拟合NN输出的 $R^2=0.45$。
\end{abstract}

\tableofcontents
\newpage

%=============================================================================
\section{引言}
%=============================================================================

\subsection{研究背景}

登革热是全球最重要的蚊媒传染病之一,主要通过伊蚊传播。
建立精确的数学模型对于理解传播机制、预测疫情趋势具有重要意义。
传统的动力学模型(如SIR、SEIR)在描述蚊媒传播时面临一个核心困难:
\textbf{蚊虫产卵率、发育率等关键参数与气象因素的函数关系形式未知}。
现有研究通常基于实验室数据预设函数形式(如高斯函数、Brière函数),
但这些形式在自然环境中的适用性尚未充分验证。

\subsection{参考方法}

本研究参照 Zhang, Wang \& Tang (2024) \cite{zhang2024} 在 \textit{PLoS Computational Biology} 发表的方法。
该工作的核心创新在于:
\begin{enumerate}[leftmargin=2em]
    \item 将\textbf{神经网络嵌入微分方程内部},替代未知的产卵率函数
    \item 通过ODE数值解与观测数据的误差\textbf{反向传播}来间接训练NN
    \item 训练后用\textbf{符号回归}解释NN,获得显式解析公式
\end{enumerate}

该方法的优势在于:动力学模型始终是主体框架,NN仅服务于发现未知机制,最终模型完全可解释。

\subsection{本研究的扩展}

在原方法基础上,本研究进行了以下扩展:
\begin{itemize}[leftmargin=2em]
    \item 从纯蚊虫种群模型扩展到\textbf{蚊虫--人群耦合模型}(加入SEIR疾病传播)
    \item 同时拟合\textbf{蚊虫监测数据(BI)和登革热病例数据}
    \item 采用\textbf{三步训练策略}:先蚊虫 → 后疾病 → 联合微调
\end{itemize}

%=============================================================================
\section{模型框架}
%=============================================================================

\subsection{整体架构}

模型分为两个阶段(图\ref{fig:framework}):

\begin{center}
\fbox{\parbox{0.92\textwidth}{
\centering
\textbf{Phase 1: 神经网络耦合动力学模型} \\[0.3em]
气象数据 $(T, H, R)$ $\xrightarrow{\text{NN}}$ 产卵率 $\xrightarrow{\text{ODE}}$ 蚊虫种群 + 病例 \\[0.3em]
ODE数值解 vs 观测数据 $\xrightarrow{\text{反向传播}}$ 训练NN \\[0.8em]
\textbf{Phase 2: 符号回归} \\[0.3em]
训练好的NN $\xrightarrow{\text{符号回归}}$ 解析公式 $f(T, H, R)$ \\[0.3em]
公式替代NN $\rightarrow$ 完全可解释的动力学模型
}}
\end{center}

\subsection{蚊虫种群动力学方程}

蚊虫种群分为未成熟期(卵、幼虫、蛹合并为 $P$)和成蚊期($A$),
建立如下微分方程系统:

\begin{align}
    \frac{dP}{dt} &= \underbrace{\text{NN}(T, H, R)}_{\text{产卵率(NN替代)}} \cdot A
                    - d_p(T) \cdot P
                    - m_p(T) \cdot P \cdot \left(1 + \frac{P}{K}\right) \label{eq:dP} \\[0.5em]
    \frac{dA}{dt} &= \sigma \cdot d_p(T) \cdot P - m_a(T) \cdot A \label{eq:dA}
\end{align}

其中:
\begin{itemize}[leftmargin=2em]
    \item $\text{NN}(T, H, R)$:产卵率函数,\textbf{由神经网络近似}(核心未知项)
    \item $d_p(T) = 0.08 \cdot e^{-((T-27)/9)^2}$:发育率(Sharpe \& DeMichele)
    \item $m_p(T) = 0.05 + 0.003(T-22)^2$:幼虫死亡率(Otero et al. 2006)
    \item $m_a(T) = 0.03 + 0.002(T-26)^2$:成蚊死亡率(Brady et al. 2013)
    \item $\sigma$:羽化存活率(可训练参数)
    \item $K$:环境承载力(可训练参数)
\end{itemize}

\subsection{疾病传播动力学方程}

在蚊虫种群动态基础上,建立简化的SEIR疾病传播方程:

\begin{align}
    \frac{dE_h}{dt} &= \beta \cdot b(T) \cdot \frac{\tilde{A}(t)}{N_h} \cdot S_h \cdot \alpha
                      + \text{imp} - \sigma_h \cdot E_h \label{eq:dEh} \\
    \frac{dI_h}{dt} &= \sigma_h \cdot E_h - \gamma \cdot I_h \label{eq:dIh} \\
    \frac{dR_h}{dt} &= \gamma \cdot I_h \label{eq:dRh}
\end{align}

其中 $\tilde{A}(t) = A(t) / \bar{A}$ 为归一化蚊虫密度,
$b(T) = 0.4 \cdot e^{-((T-27)/6)^2}$ 为温度依赖的传播效率,
$\beta$ 为传播系数(可训练),$\alpha$ 为放大因子(可训练),
$\text{imp}$ 为输入性病例率(可训练)。

\subsection{神经网络架构}

产卵率NN采用3层前馈网络,参照 Zhang et al. (2024):

\begin{table}[H]
    \centering
    \caption{产卵率神经网络架构}
    \begin{tabular}{lccc}
        \toprule
        \textbf{层} & \textbf{输入维度} & \textbf{输出维度} & \textbf{激活函数} \\
        \midrule
        输入层 & 3 (T, H, R) & 16 & Softplus \\
        隐藏层 & 16 & 16 & Softplus \\
        输出层 & 16 & 1 & Softplus \\
        \bottomrule
    \end{tabular}
    \label{tab:nn_arch}
\end{table}

网络输出经 Softplus 变换保证产卵率恒正。总计353个可训练参数。

\subsection{训练策略}

采用三步训练策略:

\begin{description}[leftmargin=2em]
    \item[Step A: 蚊虫ODE+NN $\rightarrow$ 拟合BI]
        固定疾病部分,仅训练蚊虫ODE和NN参数。
        损失函数包含BI拟合MSE、相关性损失、NN平滑性正则和变异性鼓励项。

    \item[Step B: 固定蚊虫 $\rightarrow$ SEIR拟合病例]
        固定蚊虫模型(包括NN),仅训练疾病参数 $\beta$, $\text{imp}$, $\alpha$。
        损失为log空间MSE + 相关性损失。

    \item[Step C: 联合微调]
        同时优化所有参数(蚊虫+疾病),平衡BI拟合和病例拟合。
\end{description}

所有训练使用 Adam 优化器,梯度裁剪 $\|\nabla\|_{\max}=5.0$,
ODE采用显式Euler积分(日步长),气象数据月度恒定。

%=============================================================================
\section{数据}
%=============================================================================

\begin{table}[H]
    \centering
    \caption{数据来源与描述}
    \begin{tabular}{llcl}
        \toprule
        \textbf{数据} & \textbf{来源} & \textbf{时间范围} & \textbf{说明} \\
        \midrule
        登革热病例 & CCM14 数据集 & 2006--2019 & 广东省月度,排除2014 \\
        布雷图指数 & CCM14 数据集 & 2006--2014 & 广州市月度,82个有效月 \\
        气象数据 & Open-Meteo API & 2006--2019 & 温度、湿度、降水 \\
        \bottomrule
    \end{tabular}
\end{table}

\textbf{排除2014年的原因}:
2014年广东省登革热大暴发(45,189例),占2006--2014年总病例数的90\%。
该极端事件可能由输入性病例大量增加、城市化等非气象因素导致,
严重干扰基于气象驱动的动力学模型训练。
排除后总病例18,315例(156个月),分布更均匀,利于模型学习常态传播规律。

%=============================================================================
\section{结果}
%=============================================================================

\subsection{Phase 1: 神经网络耦合动力学模型}

\subsubsection{整体性能}

\begin{table}[H]
    \centering
    \caption{Phase 1 模型性能}
    \begin{tabular}{lcc}
        \toprule
        \textbf{指标} & \textbf{含2014} & \textbf{排除2014} \\
        \midrule
        病例相关系数 $r$ & $-0.02$ & $\mathbf{0.59}$ \\
        $R^2$ (log空间) & $0.13$ & $\mathbf{0.25}$ \\
        $p$ 值 & $0.84$ & $\mathbf{9.4 \times 10^{-16}}$ \\
        BI相关系数 $r$ & $0.64$ & $0.28$ \\
        \bottomrule
    \end{tabular}
    \label{tab:phase1_perf}
\end{table}

排除2014后,病例相关系数从 $r=-0.02$ 大幅提升至 $r=0.59$($p<10^{-15}$),
表明模型成功捕捉了登革热传播的季节性和年际变化规律。

\subsubsection{综合可视化}

图\ref{fig:phase1}展示了Phase 1的完整结果,包括蚊虫种群拟合、病例拟合、
NN学到的产卵率模式以及训练过程。

\begin{figure}[H]
    \centering
    \includegraphics[width=0.95\textwidth]{../results/figures/phase1_coupled_model.png}
    \caption{Phase 1 神经网络耦合动力学模型综合结果。
    \textbf{第一行}:BI拟合(左)、病例拟合(中)、病例对数尺度(右)。
    \textbf{第二行}:NN产卵率时间序列(左)、产卵率vs温度散点(中)、蚊虫种群动态(右)。
    \textbf{第三行}:NN学到的产卵率热力图(左)、年度病例对比(中)、预测vs实际散点(右)。
    \textbf{第四行}:训练损失曲线(左)、BI散点(中)、模型框架总结(右)。}
    \label{fig:phase1}
\end{figure}

\subsubsection{分年度分析}

\begin{table}[H]
    \centering
    \caption{分年度拟合结果}
    \begin{tabular}{cccc}
        \toprule
        \textbf{年份} & \textbf{实际病例} & \textbf{模型预测} & \textbf{年内} $r$ \\
        \midrule
        2006 & 1,010 & 553 & 0.30 \\
        2007 & 397 & 133 & 0.08 \\
        2008 & 87 & 47 & 0.08 \\
        2009 & 19 & 45 & $-0.11$ \\
        2010 & 139 & 54 & 0.12 \\
        2011 & 49 & 41 & 0.42 \\
        2012 & 474 & 60 & 0.58 \\
        2013 & 2,894 & 159 & 0.49 \\
        2015 & 1,683 & 43 & 0.01 \\
        2016 & 544 & 45 & 0.33 \\
        2017 & 1,662 & 144 & 0.45 \\
        2018 & 3,315 & 49 & 0.52 \\
        \textbf{2019} & \textbf{6,042} & \textbf{5,710} & $\mathbf{0.80}$ \\
        \bottomrule
    \end{tabular}
    \label{tab:yearly}
\end{table}

2019年暴发(6,042例)模型预测为5,710例,年内相关系数 $r=0.80$,
说明模型对暴发年份的传播动态具有较好的捕捉能力。

\subsubsection{NN学到的产卵率模式}

\begin{table}[H]
    \centering
    \caption{NN学到的产卵率特征}
    \begin{tabular}{lc}
        \toprule
        \textbf{指标} & \textbf{值} \\
        \midrule
        产卵率范围 & $[0.21, 7.72]$ \\
        低温 ($<18$°C) 均值 & 4.83 \\
        高温 ($>25$°C) 均值 & 0.98 \\
        \bottomrule
    \end{tabular}
\end{table}

\subsubsection{校准参数}

\begin{table}[H]
    \centering
    \caption{模型校准参数}
    \begin{tabular}{llcl}
        \toprule
        \textbf{参数} & \textbf{符号} & \textbf{值} & \textbf{说明} \\
        \midrule
        传播系数 & $\beta$ & 0.49 & 蚊→人传播强度 \\
        输入病例 & imp & 2.80/月 & 外源输入 \\
        放大因子 & $\alpha$ & 9.81 & 蚊虫密度效应 \\
        \bottomrule
    \end{tabular}
\end{table}

\subsection{Phase 2: 符号回归}

\subsubsection{候选公式比较}

\begin{table}[H]
    \centering
    \caption{符号回归候选公式评估}
    \begin{tabular}{lcccc}
        \toprule
        \textbf{公式} & \textbf{类型} & $r$ & $R^2$ & \textbf{参数数} \\
        \midrule
        $a \cdot e^{-((T-T_0)/\sigma)^2}$ & 温度高斯 & 0.054 & $-0.004$ & 3 \\
        $a \cdot G(T) \cdot G(H)$ & 温度$\times$湿度 & 0.627 & 0.391 & 5 \\
        $a \cdot G(T) \cdot G(H) \cdot \text{rain}$ & 完整 & 0.627 & 0.391 & 7 \\
        $a \cdot T(T-T_{\min})\sqrt{T_{\max}-T}$ & Brière型 & $-0.656$ & $-0.742$ & 3 \\
        $\mathbf{a + bT + cT^2 + dT^3}$ & \textbf{三次多项式} & $\mathbf{0.668}$ & $\mathbf{0.447}$ & \textbf{4} \\
        \bottomrule
    \end{tabular}
    \label{tab:formulas}
\end{table}

\subsubsection{最优公式}

符号回归发现的最优近似为三次多项式:

\begin{equation}
    \boxed{f(T) = 9.50 - 0.311T - 0.00622T^2 + 0.000219T^3}
    \label{eq:best_formula}
\end{equation}

该公式拟合NN输出的 $R^2 = 0.447$,$r = 0.668$。
物理上,该函数在低温端产卵率较高(反映越冬卵储备),
在中高温区迅速下降(反映快速发育消耗),
与NN学到的非线性模式一致。

\subsubsection{符号回归综合可视化}

图\ref{fig:phase2}展示了Phase 2的完整符号回归结果。

\begin{figure}[H]
    \centering
    \includegraphics[width=0.95\textwidth]{../results/figures/phase2_formula_discovery.png}
    \caption{Phase 2 符号回归结果。
    \textbf{第一行}:候选公式R²对比(左)、最优公式vs NN输出散点(中)、温度响应曲线(右)。
    \textbf{第二行}:降水响应曲线(左)、公式热力图T$\times$R(中)、残差分布(右)。
    \textbf{第三行}:NN vs 公式时间序列(左)、动力学验证(中)、模型总结(右)。}
    \label{fig:phase2}
\end{figure}

%=============================================================================
\section{讨论}
%=============================================================================

\subsection{方法优势}

\begin{enumerate}[leftmargin=2em]
    \item \textbf{动力学模型为主体}:
        ODE框架保证了模型的生物学可解释性,NN仅替代未知的产卵率函数。

    \item \textbf{端到端训练}:
        NN通过ODE数值解间接训练,学到的函数自动满足动力学约束。

    \item \textbf{两阶段可解释性}:
        Phase 1用NN发现模式,Phase 2用符号回归转化为解析公式,
        最终模型完全可解释。

    \item \textbf{病例预测统计显著}:
        排除极端暴发后,$r=0.59$($p<10^{-15}$),2019年暴发 $r=0.80$。
\end{enumerate}

\subsection{当前局限}

\begin{enumerate}[leftmargin=2em]
    \item \textbf{病例绝对量级偏差}:
        模型季节性趋势正确($r=0.59$),但多数年份的绝对病例数预测偏低。
        这可能由于输入性病例的年际波动、报告率变化等非气象因素。

    \item \textbf{2014年极端暴发无法拟合}:
        该年45,189例(占总量90\%),
        可能由特殊的输入性病例激增、社会因素导致,超出气象驱动模型能力。

    \item \textbf{符号回归 $R^2$ 中等}:
        最优公式 $R^2=0.45$,说明NN学到的函数不完全是简单解析形式。
        可能需要引入更复杂的候选公式或使用PySR等自动化工具。

    \item \textbf{BI数据覆盖有限}:
        BI仅覆盖2006--2014年的82个月(广州),2015--2019年无BI,
        导致蚊虫模型在后期缺乏约束。
\end{enumerate}

\subsection{改进方向}

\begin{enumerate}[leftmargin=2em]
    \item 使用 \textbf{PySR} 等自动化符号回归工具扩大公式搜索空间
    \item 引入\textbf{可微分ODE求解器}(如 \texttt{torchdiffeq})替代Euler积分
    \item 构建\textbf{分城市模型}(广州、深圳等)利用空间异质性
    \item 考虑2014年极端暴发的\textbf{输入性病例模型}
\end{enumerate}

%=============================================================================
\section{结论}
%=============================================================================

本研究成功实现了\textbf{神经网络耦合动力学模型}框架,将蚊虫产卵率的未知函数用NN替代并嵌入ODE系统,通过端到端训练学习气象因素对蚊虫种群的影响机制。

主要结论:
\begin{enumerate}[leftmargin=2em]
    \item 动力学模型作为主体框架,NN嵌入其中替代未知的产卵率函数,
          训练后\textbf{病例相关系数 $r=0.59$}($p<10^{-15}$)。

    \item 三步训练策略(蚊虫拟合 → 疾病拟合 → 联合微调)有效平衡了
          蚊虫监测数据和病例数据的拟合。

    \item 符号回归发现产卵率的最优近似为三次多项式
          $f(T) = 9.50 - 0.311T - 0.006T^2 + 0.0002T^3$,$R^2=0.45$。

    \item 2014年极端暴发是模型拟合的主要障碍,排除后性能大幅改善。

    \item 该框架为\textbf{蚊媒传染病动力学建模中未知机制的数据驱动发现}
          提供了一种有效方法。
\end{enumerate}

%=============================================================================
\section*{参考文献}
%=============================================================================

\begin{thebibliography}{99}

\bibitem{zhang2024}
Zhang M, Wang X, Tang S (2024).
Integrating dynamic models and neural networks to discover the mechanism of meteorological factors on Aedes population.
\textit{PLoS Computational Biology}, 20(9): e1012499.
\url{https://doi.org/10.1371/journal.pcbi.1012499}

\bibitem{mordecai2017}
Mordecai EA, et al. (2017).
Detecting the impact of temperature on transmission of Zika, dengue, and chikungunya using mechanistic models.
\textit{PLoS Neglected Tropical Diseases}.

\bibitem{brady2013}
Brady OJ, et al. (2013).
Modelling adult Aedes aegypti and Aedes albopictus survival at different temperatures in laboratory and field settings.
\textit{Parasites \& Vectors}.

\bibitem{otero2006}
Otero M, Solari HG, Schweigmann N (2006).
A stochastic population dynamics model for Aedes aegypti.
\textit{Bulletin of Mathematical Biology}.

\bibitem{ccm14}
CCM14: Mosquito surveillance data in China.
\url{https://github.com/xyyu001/CCM14}

\end{thebibliography}

%=============================================================================
\appendix
\section{代码与数据}

所有代码和数据存放于 \texttt{v1\_nn\_coupled\_dynamics/} 目录:

\begin{verbatim}
v1_nn_coupled_dynamics/
├── code/
│   ├── phase1_coupled_model.py   # Phase 1: NN耦合动力学
│   └── phase2_formula_discovery.py # Phase 2: 符号回归
├── results/
│   ├── figures/                  # 可视化图片
│   └── data/                     # 预测数据、模型权重
└── report/
    └── report_v1.tex             # 本报告
\end{verbatim}

\end{document}
